\documentclass[12pt]{article}
\usepackage[utf8]{inputenc}

\usepackage{enumitem}
\usepackage[margin=2cm]{geometry}

\usepackage{amsmath, amsfonts, amssymb}
\usepackage{graphicx}
\usepackage{tikz}
\usepackage{pgfplots}
\usepackage{multicol}

\usepackage{comment}
\usepackage{url}
\usepackage{calc}

\usepackage{titlesec}
\titleformat{\section}[frame]
{\normalfont\scshape}
{\thesection}{8pt}{\centering}

\usepackage{array}

\pgfplotsset{compat=1.16}

%\usepackage[margin=1cm]{cloze}

\usepackage[thmmarks]{ntheorem}

% MATH commands
\newcommand{\bC}{\mathbb{C}}
\newcommand{\bR}{\mathbb{R}}
\newcommand{\bN}{\mathbb{N}}
\newcommand{\bQ}{\mathbb{Q}}
\newcommand{\bZ}{\mathbb{Z}}
\newcommand{\bT}{\mathbb{T}}
\newcommand{\bD}{\mathbb{D}}

\newcommand{\cL}{\mathcal{L}}
\newcommand{\cM}{\mathcal{M}}
\newcommand{\cP}{\mathcal{P}}
\newcommand{\cH}{\mathcal{H}}
\newcommand{\cB}{\mathcal{B}}
\newcommand{\cK}{\mathcal{K}}
\newcommand{\cJ}{\mathcal{J}}
\newcommand{\cU}{\mathcal{U}}
\newcommand{\cO}{\mathcal{O}}
\newcommand{\cA}{\mathcal{A}}
\newcommand{\cC}{\mathcal{C}}
\newcommand{\cT}{\mathcal{T}}

\newcommand{\fK}{\mathfrak{K}}
\newcommand{\fM}{\mathfrak{M}}

\newcommand{\ga}{\left\langle}
\newcommand{\da}{\right\rangle}
\newcommand{\oa}{\left\lbrace}
\newcommand{\fa}{\right\rbrace}
\newcommand{\oc}{\left[}
\newcommand{\fc}{\right]}
\newcommand{\op}{\left(}
\newcommand{\fp}{\right)}

\newcommand{\ra}{\rightarrow}
\newcommand{\Ra}{\Rightarrow}

\renewcommand{\Re}{\mathrm{Re}\,}
\renewcommand{\Im}{\mathrm{Im}\,}
\newcommand{\Arg}{\mathrm{Arg}\,}
\newcommand{\Arctan}{\mathrm{Arctan}\,}
\newcommand{\sech}{\mathrm{sech}\,}
\newcommand{\csch}{\mathrm{csch}\,}
\newcommand{\Log}{\mathrm{Log}\,}
\newcommand{\cis}{\mathrm{cis}\,}

\newcommand{\ran}{\mathrm{ran}\,}
\newcommand{\bi}{\mathbf{i}}
\newcommand{\Sp}{\mathrm{span}\,}
\newcommand{\Inv}{\mathrm{Inv}\,}
\newcommand\smallO{
  \mathchoice
    {{\scriptstyle\mathcal{O}}}% \displaystyle
    {{\scriptstyle\mathcal{O}}}% \textstyle
    {{\scriptscriptstyle\mathcal{O}}}% \scriptstyle
    {\scalebox{.7}{$\scriptscriptstyle\mathcal{O}$}}%\scriptscriptstyle
  }
\newcommand{\HOL}{\mathrm{Hol}}
\newcommand{\cl}{\mathrm{clos}}
\newcommand{\ve}{\varepsilon}

\tikzstyle{myboxT} = [draw=black, fill=black!0,line width = 1pt,
    rectangle, rounded corners = 0pt, inner sep=8pt, inner ysep=8pt]
    
\newcommand{\MyC}[1]{\begin{tikzpicture}
\node (boxIntro) at (0,0) {};
\node [myboxT](Intro) at (boxIntro){%
	\begin{minipage}{0.9\textwidth}
	#1
	\end{minipage}};
\end{tikzpicture}}

%%%%  Environnement exer et solutionnaire
{\theorembodyfont{}
\theoremstyle{plain}
\theoremseparator{\textbf{.}}
\theoremsymbol{}
\newtheorem{exer}{\textbf{Exercise}}}

{\theorembodyfont{\color{blue}}
\theoremstyle{plain}
\theoremseparator{\textbf{:}}
\theoremsymbol{$\square$}
\newtheorem*{sol}{\textbf{Solution}}}

{\theorembodyfont{\color{blue}}
\theoremstyle{plain}
\theoremseparator{\textbf{:}}
\theoremsymbol{$\qed$}
\newtheorem*{solWP}{\textbf{Solution}}}

{\theorembodyfont{\color{blue}}
\theoremstyle{plain}
\theoremseparator{\textbf{:}}
\theoremsymbol{$\qed$}
\newtheorem*{hint}{\textbf{Hints}}}

\renewcommand*{\theexer}{\arabic{exer}}
\renewcommand*{\thesol}{\arabic{sol}}
\renewcommand*{\thesolWP}{\arabic{solWP}}
\renewcommand*{\thehint}{\arabic{hint}}

%%% Ignorer les solutions
%\excludecomment{sol}
\excludecomment{solWP}
\excludecomment{hint}

\newcommand{\headHW}[4]{%
	\noindent \hrulefill \\
	MATH-#1 #2 \\
	#3 #4
	
}

\makeatletter
\DeclareFontFamily{U}{tipa}{}
\DeclareFontShape{U}{tipa}{m}{n}{<->tipa10}{}
\newcommand{\arc@char}{{\usefont{U}{tipa}{m}{n}\symbol{62}}}%

\newcommand{\arc}[1]{\mathpalette\arc@arc{#1}}

\newcommand{\arc@arc}[2]{%
  \sbox0{$\m@th#1#2$}%
  \vbox{
    \hbox{\resizebox{\wd0}{\height}{\arc@char}}
    \nointerlineskip
    \box0
  }%
}
\makeatother

\newcount\QO
\newcounter{QT}
\newcounter{QTh}
\newcounter{QF}
\newcounter{QFi}
\newcounter{QS}
\newcounter{QSe}
\newcounter{QE}
\newcounter{QN}
\newcounter{QTe}

\begin{document}
	\noindent \hrulefill \\
	MATH-331 Introduction to Real Analysis \hfill Created by P.-O. Paris{\'e}\\
	Homework 05 \hfill Fall 2021\\\vspace*{-0.7cm}
	
	\noindent\hrulefill
	
	\noindent Due date: November, 22${}^{\text{th}}$ 1:20pm \hfill Total: \hspace{0.3cm}/65.
	
\vspace*{0.5cm}

	\bgroup \renewcommand{\arraystretch}{1.5}
\begin{table}[h]
\centering
\begin{tabular}{|m{1.5cm}|>{\centering\arraybackslash}p{0.75cm}|>{\centering\arraybackslash}p{0.75cm}|>{\centering\arraybackslash}p{0.75cm}|>{\centering\arraybackslash}p{0.75cm}|>{\centering\arraybackslash}p{0.75cm}|>{\centering\arraybackslash}p{0.75cm}|>{\centering\arraybackslash}p{0.75cm}|>{\centering\arraybackslash}p{0.75cm}|>{\centering\arraybackslash}p{0.75cm}|>{\centering\arraybackslash}p{0.75cm}|}
\hline
Exercise & 1 (10) & 2 (10) & 3 (5) & 4 (5) & 5 (5) & 6 (10) & 7 (5) & 8 (5) & 9 (5) & 10 (5) \\
\hline
Score & & & & & & & & & &  \\\hline
\end{tabular}
\caption{Scores for each exercises}
\end{table}
\egroup
	
\vspace*{0.5cm}

{\bf Instructions:} You must answer all the questions below and send your solution by email (to \url{parisepo@hawaii.edu}). If you decide to not use {\LaTeX} to hand out your solutions, please be sure that after you scan your copy, it is clear and readable. Make sure that you attached a copy of the homework assignment to your homework. 

\noindent If you choose to use {\LaTeX}, you can use the template available on the course website.

\noindent No late homework will be accepted. No format other than PDF will be accepted. Name your file as indicated in the syllabus.

\section{Writing problems}
For each of the following problems, you will be asked to write a clear and detailed proof. You will have the chance to rewrite your solution in your semester project after receiving feedback from me.

%% ---------------------------------------------
\begin{exer}
(10 pts)
\begin{enumerate}[label=\textbf{\alph*)}]
\item Fix any $\delta > 0$ and let $[a, b]$ be an interval with $a < b$. Find a tagged partition $\cP$ of $[a, b]$ such that $\Vert \cP \Vert < \delta$.
\item Suppose that $f$ is Riemann integrable. Show that in the definition of the Riemann integral, the number $L$ is unique. [Remark: This is why we gave it the name $\int_a^b f$.]
\end{enumerate}
\end{exer}
\begin{sol}
\begin{enumerate}[label=\textbf{\alph*)}]
\item By the Archimedean Property, let $N \in \bN$ such that $\beta := \frac{b - a}{N} < \delta$. Define the points $x_i$ for $i = 0, 1, 2, \ldots , N$ as
	\begin{align*}
	x_i := a + i \beta .
	\end{align*}
Let $\cP := \{ (x_i , [x_{i - 1} , x_i ] \, : \, i = 1 , 2, \ldots , N \}$. 

Then, we see that $x_i - x_{i-1} = \frac{b - a}{N} < \delta$ for any $i = 1 , \ldots , N$. Thus, 
	\begin{align*}
	\Vert \cP \Vert = \max \{ x_i - x_{i-1} \, : \, i = 1 , 2, \ldots , N \} = \frac{b - a}{N} < \delta.
	\end{align*}
	The partition $\cP$ is the one we were searching for.
\item Suppose that we have two different numbers $L_1$ and $L_2$ for the Riemann integral of $f$. Let $\varepsilon > 0$. 

Then there is a number $\delta_1 > 0$ such that for any tagged partition $\cP$ of $[a, b]$, if $\Vert \cP \Vert < \delta_1$, then
	\begin{align*}
	| S (f , \cP ) - L_1 | < \varepsilon .
	\end{align*}
There is also another number $\delta_2 > 0$ such that for any tagged partition $\cP$ of $[a, b]$, if $\Vert \cP \Vert < \delta_2$, then
	\begin{align*}
	| S (f, \cP ) - L_2 | < \varepsilon .
	\end{align*}

Let $\delta := \min \{ \delta_1 , \delta_2 \}$. Let $\cP_0$ be a tagged partition such that $\Vert \cP_0 \Vert < \delta$. This partition exists from the first part of the exercise. Then, we have
	\begin{align*}
	| L_1 - L_2 | \leq |L_1 - S (f , \cP_0 ) | + | S (f , \cP_0 ) - L_2 | < 2 \varepsilon .
	\end{align*}
Thus, for any $\varepsilon > 0$, we have $0 \leq |L_1 - L_2 | < 2 \varepsilon$. Taking the limit as $\varepsilon \ra 0$, by the Squeeze Theorem for limits, we get $|L_1 - L_2| = 0$ which is equivalent to $L_1 = L_2$ from the property of the absolute value.
\end{enumerate}
\end{sol}

%-----------------------------------------------
\begin{exer}
(10 pts)
Suppose that $f$ and $g$ are Riemann integrable on the interval $[a, b]$.
	\begin{enumerate}[label=\textbf{\alph*)}]
	\item Show that $\int_a^b (f + g) = \int_a^b f + \int_a^b g$.
	\item Show that if $f(x) \leq g(x)$ for any $x \in [a, b]$, then $\int_a^b f \leq \int_a^b g$.
	\end{enumerate}
\end{exer}
\begin{sol}
\begin{enumerate}[label=\textbf{\alph*)}]
\item Let $\varepsilon > 0$ and let $\cP$ be a tagged partition of $[a, b]$. Since $f$ and $g$ are Riemann integrable on $[a, b]$, there are numbers $\delta_1 > 0$ and $\delta_2 > 0$ such that
	\begin{itemize}
	\item if $\Vert \cP \Vert < \delta_1$, then $|S (f , \cP ) - \int_a^b f | < \varepsilon/2$.
	\item if $\Vert \cP \Vert < \delta_2$, then $|S (g, \cP ) - \int_a^b g| < \varepsilon/2$.
	\end{itemize}
Let $\delta := \min \{ \delta_1 , \delta_2 \}$. Then if $\Vert \cP \Vert < \delta$, then
	\begin{align*}
	\Big| S (f + g , \cP ) - \int_a^b f - \int_a^b g \Big| &= \Big| S (f, \cP ) + S (g , \cP ) - \int_a^b f - \int_a^b g \Big| \\
	& \leq \Big| S (f , \cP ) - \int_a^b f \Big| + \Big| S (g , \cP ) - \int_a^b g \Big| \\
	& < \varepsilon/2 + \varepsilon/2 = \varepsilon .
	\end{align*}
	Thus, $f + g$ is Riemann integrable and the value is $\int_a^b (f + g) = \int_a^b f + \int_a^b g$.
\item Suppose that $f(x) \leq g(x)$ for any $x \in [a, b]$. Let $\cP$ be a tagged partition of $[a, b]$. By the assumption, we see that
	\begin{align*}
	S (f, \cP ) \leq S (g, \cP ) .
	\end{align*}
Let $\varepsilon > 0$. Our goal is to show that $\int_a^b f < \int_a^b g + \varepsilon$.

Since $f$ and $g$ are Riemann integrable on $[a , b]$, there are numbers $\delta_1, \delta_2 > 0$ such that
	\begin{itemize}
	\item if $\Vert \cP \Vert < \delta_1$, then $|S (f , \cP ) - \int_a^b f | < \varepsilon/2$.
	\item if $\Vert \cP \Vert < \delta_2$, then $|S (g, \cP ) - \int_a^b g| < \varepsilon/2$.
	\end{itemize}
Let $\cP_0$ be a tagged partition of $[a, b]$ such that $\Vert \cP_0 \Vert \leq \min \{ \delta_1 , \delta_2 \}$. Then we have that
	\begin{align*}
	- \varepsilon/2 < \int_a^b f - S (f , \cP_0 ) < \varepsilon/2
	\end{align*}
and
	\begin{align*}
	-\varepsilon /2< S (g, \cP_0 ) - \int_a^b g < \varepsilon/2 .
	\end{align*}
Thus, adding together these last inequalities, we get
	\begin{align*}
	- \varepsilon < \int_a^b f - \int_a^b g + S (g, \cP_0 ) - S (f, \cP_0 ) < \varepsilon
	\end{align*}
However, since $S (f , \cP_0 ) \leq S (g , \cP_0 )$, we have that $S (g, \cP_0 ) - S (f ,\cP_0 ) \geq 0$ and so
	\begin{align*}
	\int_a^b f - \int_a^b g \leq \int_a^b f - \int_a^b g + S(g , \cP_0 ) - S (f , \cP_0 ) < \int_a^b f - \int_a^b g + \varepsilon .
	\end{align*}
Thus, $\int_a^b f < \int_a^b g + \varepsilon$.

Since this is true for any $\varepsilon > 0$, we can take the limit as $\varepsilon \ra 0$ and get from the Squeeze Theorem for limits that $\int_a^b f \leq \int_a^b g$.
\end{enumerate}
\end{sol}

%-----------------------------------------------
\begin{exer}
(5 pts)
Let $f : [a, b] \ra \bR$ be Riemann integrable on $[a, b]$ and suppose that $|f(x)| \leq M$ for all $x \in [a, b]$. Show that $\int_a^b f \leq M (b - a)$.
\end{exer}
\begin{sol}
By Exercise 2 b), with $g (x) = M$ for any $x \in [a, b]$, we see that
	\begin{align*}
	\int_a^b f \leq \int_a^b M .
	\end{align*}
Since $M$ is a constant, from a property of the Riemann integral, we see that $\int_a^b M = M (b - a)$. Thus, we get $\int_a^b f \leq M (b - a)$.
\end{sol}

%-----------------------------------------------
\begin{exer}
(5 pts)
Suppose that $f$ is Riemann integrable on $[a, b]$. Let $( \cP_n )_{n =1}^\infty$ be a sequence of tagged partitions of $[a, b]$ such that the sequence $\lim_{n \ra \infty} \Vert \cP_n \Vert = 0$. Prove that the sequence $( S (f , \cP_n ))_{n = 1}^\infty$ converges to $\int_a^b f$.
\end{exer}
\begin{sol}
Suppose that $f$ is Riemann integral on $[a, b]$. Let $(\cP_n )_{n = 1}^\infty$ be a sequence of tagged partitions of $[a, b]$ such that $\lim_{n \ra \infty} \Vert \cP_n \Vert = 0$. The goal is to show that $\lim_{n \ra \infty} S (f , \cP_n ) = \int_a^b f$, that is for any $\varepsilon > 0$, there is an integer $N \in \bN$ such that if $n \geq N$, then $|S (f, \cP_n ) - \int_a^b f| < \varepsilon$.

Let $\varepsilon > 0$. Since $f$ is Riemann integrable on $[a, b]$, there is a number $\delta > 0$ such that if $\cP$ is a tagged partition of $[a, b]$ and $\Vert cP \Vert < \delta$, then
	\begin{align*}
	\Big| S (f , \cP ) - \int_a^b f \Big| < \varepsilon .
	\end{align*}
From the assumptions, we know that $\lim_{n \ra \infty} \Vert \cP_n \Vert = 0$. So, let $N \in \bN$ be an integer such that if $n \geq N$, then $\Vert \cP_n \Vert < \delta$. Then, of any $n \geq N$, we have
	\begin{align*}
	\Big| S (f, \cP_n ) - \int_a^b f \Big| < \varepsilon .
	\end{align*}
This is exactly what we wanted to prove. 
\end{sol}


%-----------------------------------------------
\begin{exer}
(5 pts)
Let $f : [a, b] \ra \bR$ be a bounded function. Suppose that $f$ is Riemann integrable on $[a, c]$ for any $c \in (a, b)$. Show that $f$ is Riemann integrable on $[a, b]$. [Hint: Use the Cauchy criterion for integrals.]
\end{exer}
\begin{sol}
Suppose $f$ is bounded in absolute value by $M$. Let $\varepsilon > 0$. Let $\cP_1$ and $\cP_1$ be two tagged partitions of $[a, b]$. We would like to prove that if $\Vert \cP_1 \Vert < \delta $ and $\Vert \cP_2 \Vert < \delta$ for some $\delta$ (to find out), then $|S (f , \cP_1 ) - S (f , \cP_2 ) | < \varepsilon$. 

Select $c \in (a, b)$ such that $b - c < \varepsilon$. We know that $f$ is Riemann integrable on $[a, c]$, so there exists a $\delta > 0$ such that for any tagged partition $\cT_1$, $\cT_2$ of $[a, c]$, if $\Vert \cT_1 \Vert < \delta$ and $\Vert \cT_2 \Vert < \delta$, then $|S (f, \cT_1 ) - S (f , \cT_2 )| < \varepsilon$. 

Let $\cP_1$ and $\cP_2$ be decribed as followed:
	\begin{align*}
	\cP_1 := \{ (c_i , [x_{i - 1} , x_i] ) \, : \, i = 1, 2, \ldots  , N_1 \}
	\end{align*}
and
	\begin{align*}
	\cP_2 := \{ (d_i , [y_{i - 1} , y_i ]) \, : \, i = 1 , 2, \ldots , N_2 \} .
	\end{align*}
Suppose that $\Vert \cP_1 \Vert < \delta$ and $\Vert \cP_2 \Vert < \delta$.
	
Define $\cP_1^a$, $\cP_1^b$, $\cP_2^a$, and $\cP_2^b$ as followed:
	\begin{align*}
	\cP_1^a &:= \{ (c_i , [x_{i - 1} , x_i ]) \in \cP_1 \, : \, [x_{i-1}, x_i] \subset [a, c] \}, \\
	\cP_1^b & := \{ (c_i , [x_{i - 1}, x_i]) \in \cP_1 \, : \, [x_{i-1}, x_i] \subset [c, b] \},\\
	\cP_2^a & := \{ (c_i , [y_{i - 1}, y_i]) \in \cP_2 \, : \, [y_{i-1}, y_i] \in [a, c] \} , \\
	\cP_2^b & := \{ (c_i , [y_{i - 1}, y_i]) \in \cP_2 \, : \, [y_{i-1} , y_i] \subset [c, b] \} .
	\end{align*}
Let $N_1^a$, $N_1^b$, $N_2^a$, and $N_2^b$ be the number of elements in the previous sets respectively. Now define $\tilde{\cP}_1^a$, $\tilde{\cP}_1^b$, $\tilde{\cP}_2^a$, and $\tilde{\cP}_2^b$ as followed:
	\begin{align*}
	\tilde{\cP}_1^a := \cP_1^a \cup \{(c , [x_{N_1^a} , c])\} \text{,} \quad	\tilde{\cP}_1^b := \cP_1^b \cup \{ (c, [c, x_{N_1^a + 1}] \}, \\
	\tilde{\cP}_2^a := \cP_2^a \cup \{ (c , [x_{N_2^a} , c] ) \} \text{, } \quad \tilde{\cP}_2^b := \cP_2^b \cup \{ (c , [c, x_{N_2^a + 1}]) \} .
	\end{align*}
Then we have that
	\begin{align*}
	|S (f, \cP_1 ) - S (f , \cP_2 ) | \leq |S (f, \tilde{\cP}_1^a) - S (f , \tilde{\cP}_2^a ) | + |S (f, \tilde{\cP}_1^b) - S (f, \tilde{\cP}_2^b) | .
	\end{align*}
Since $\Vert \cP_1 \Vert < \delta$, $\Vert \cP_2 \Vert < \delta$, $|x_{N_1^a} - c| < \delta$, and $|x_{N_2^a} - c| < \delta$, we see that the norms of $\tilde{\cP}_1^a$ and $\tilde{\cP}_2^a$ do not exceed $\delta$. So we have that $|S (f, \tilde{\cP}_1^a) - S (f , \tilde{\cP}_2^a ) | < \varepsilon$. Also, since $f$ is bounded in absolute value by $M$, we have
	\begin{align*}
	|S (f, \tilde{\cP}_1^b)| \leq \sum_{i = N_1^a + 1}^{N_1} |f(c_i)| (x_i - x_{i-1}| \leq M (b - c) < M \varepsilon .
	\end{align*}
With a similar argument, we obtain that $|S (f, \tilde{\cP}_2^b)| < M \varepsilon$. Thus, putting all this together, we get
	\begin{align*}
	|S (f, \cP_1 ) - S (f, \cP_2)| \leq (2M + 1) \varepsilon .
	\end{align*}
Change $\varepsilon$ for $\varepsilon/(2M + 1)$ and we obtain what we wanted to prove. Thus, from the Cauchy critirion, $f$ is Riemann integrable on $[a, b]$.
\end{sol}

\section{Homework problems}
Answer all the questions below. Make sure to show your work. When we are asking to show that a function is Riemann integrable on an interval $[a, b]$, you must use the definition or the properties of the Riemann integral presented in sections 6.1 and 6.2 respectively.

\begin{exer}
(10pts)
\begin{enumerate}[label=\textbf{\alph*)}]
\item Define the function $f : [a, b] \ra \bR$ by $f(x) = k$ for every $x \in [a, b]$ where $k \in \bR$ is a fixed constant. Show that $f$ is Riemann integrable on $[a, b]$ and that $\int_a^b k \, dx = k ( b-a)$.
\item Let $f(x) = \sin^2 (x)$ where $x \in [a, b]$ and assume that the function $g(x) := \cos (kx )$ is integrable on $[a, b]$ for any $k \in \bR$. Show that $f$ is Riemann integrable on $[a, b]$.
\end{enumerate}
\end{exer}
\begin{sol}
\begin{enumerate}[label=\textbf{\alph*)}]
\item Let $\varepsilon > 0$ and let $\cP$ be a tagged partition of $[a, b]$ and put $\cP := \{ (c_i , [x_{i - 1}, x_i] ) \, : \, i = 1 , 2, \ldots , N \}$. Then the Riemann sum of $f$ is
	\begin{align*}
	S (f, \cP ) = \sum_{i = 1}^N f(c_i) (x_i - x_{i-1}) = k \sum_{i = 1}^N (x_i - x_{i - 1}) = k (b - a) .
	\end{align*}
Thus, let $\delta = 1$. We then get that if $\Vert \cP \Vert < 1$, then
	\begin{align*}
	| S (f , \cP ) - k (b - a)| = 0 < \varepsilon .
	\end{align*}
Thus $f$ is Riemann integrable on $[a, b]$ and $\int_a^b f = k (b - a)$.
\item By a trigonometric identity, we have that
	\begin{align*}
	\sin^2 (x) = \frac{1 - \cos (2x)}{2} .
	\end{align*}
Thus we see that
	\begin{align*}
	\int_a^b \sin^2 (x) \, dx = \int_a^b \frac{1}{2} - \frac{\cos (2x )}{2} \, dx .
	\end{align*}
Since the function $h(x) = 1/2$ and $g(x) = \cos (2x)$ are integrable on $[a, b]$, from the properties of the Riemann integral, we obtain that $1/2 - (1/2)\cos (2x )$ is Riemann integrable on $[a, b]$. So $\sin^2$ is also Riemann integrable.
\end{enumerate}
\end{sol}

%------------------------------------------------
\begin{exer}
(5 pts)
Show that the function $f : [0, 1] \ra \bR$ defined by
	\begin{align*}
	f(x) := \begin{cases}
	1 & \text{, if } 0 \leq x < 1/2 \\
	0 & \text{, if } 1/2 \leq x \leq 1 
	\end{cases}
	\end{align*}
is Riemann integrable on $[0, 1]$.
\end{exer}
\begin{sol}
We will show that $\int_0^1 f = 1/2$. 

Let $\varepsilon > 0$. Let $\cP := \{ (c_i , [x_{i-1} , x_i ]) \, : \, i = 1 , 2, \ldots , N \}$ be a tagged partition of $[a, b]$. We will create two disjoint subfamily of tagged intervals. We define $\cP_1$ as the family of tagged intervals $(c_i , [x_{i - 1} , x_i] ) \in \cP$ such that $c_i < 1/2$ and we define $\cP_2$ as the family of tagged intervals $(c_i , [x_{i-1} , x_i]) \in \cP$ such that $c_i \geq 1/2$. Then $\cP = \cP_1 \cup \cP_2$ and $\cP_1 \cap \cP_2 = \varnothing$. This implies that
	\begin{align*}
	S (f , \cP ) = S (f, \cP_1) + S (f , \cP_2 ).
	\end{align*}
Let $N_1$ be the number of elements in $\cP_1$ and $N_2$ be the number of elements in $\cP_2$, then
	\begin{align*}
	S (f, \cP ) = \sum_{i = 1}^{N_1} f(c_i) (x_i - x_{i-1}) + \sum_{i = 1}^{N_2} f(c_{i + N_1}) (x_{i + N_1} - x_{i + N_1 -1}) .
	\end{align*}
For $i = 1 , 2, \ldots , N_1$, we have $f(c_i) = 1$ and for $i = 1, 2, \ldots , N_2$, $f(c_{i + N_1}) = 0$. So
	\begin{align*}
	S (f , \cP ) = \sum_{i = 1}^{N_1} (x_{i} - x_{i - 1}) = x_{N_1} - x_0 = x_{N_1} .
	\end{align*}
So we have
	\begin{align*}
	| S(f , \cP ) - 1/2| = |x_{N_1} - 1/2| .
	\end{align*}
We want that last expression to be less than $\varepsilon$. Let $\delta = \varepsilon/2$ and suppose that $\Vert \cP \Vert < \delta$. Then the distance from $x_{N_1}$ to $1/2$ is less than the distance from $x_{N_1}$ to $x_{N_1 + 2}$ because $1/2 \leq c_{N_1 + 1} \leq x_{N_1 + 1} < x_{N_1 + 2}$ and so
	\begin{align*}
	|x_{N_1} - 1/2| = 1/2 - x_{N_1} \leq x_{N_1 + 2} - x_{N_1} \leq x_{N_1 + 2} - x_{N_1 + 1} + x_{N_1 + 1} - x_{N_1} < 2 \delta = \varepsilon .
	\end{align*}
Thus, if $\Vert \cP \Vert < \delta$, then
	\begin{align*}
	| S (f, \cP ) - 1/2| = |x_{N_1} - 1/2| < \varepsilon .
	\end{align*}
This implies that $f$ is Riemann integrable on $[0, 1]$ and $\int_0^1 f = 1/2$.
\end{sol}

%-------------------------------------------------
\begin{exer}
(5 pts)
Let $f : [0, 1] \ra \bR$ be defined by $f(x) = 1$ if $x = 1/n$ where $n \in \bN$, and by $f(x) = 0$ if $x \neq 1/n$, $n \in \bN$. Show that $f$ is Riemann integrable on $[0, 1]$.
\end{exer}
\begin{sol}
We will show that $\int_c^1 f = 0$ on each interval $[c, 1]$ where $0 < c < 1$. Then from exercise 5 we can conclude that $f$ is Riemann integrable on $[0, 1]$.

Let $c \in (0, 1)$. There is a integer $n$ such that $1/n < c$ (by AP). Take the smallest integer with this property and call it $N$. This means that if $n > N$, then $1/n < c$ and if $n < N$, then $1/n > c$. 

Let $\varepsilon > 0$. Let $\cP = \{ (c_i , [x_{i - 1} , x_i]) \, : \, i = 1 , 2, \ldots , k \}$. Suppose that $\Vert \cP \Vert < \varepsilon/ N$. If no tag in $\cP$ is equal to $1/n$ for some $n < N$, then $S (f, \cP ) = 0$ because in this case $f(c_i) = 0$.

Suppose that there is at least one $c_i$ such that $f(c_i) = 1/n$ for some $n < N$. Let $\cP_1$ be the subfamily of tagged intervals from $\cP$ defined by the following relation:
	\begin{align*}
	(c_i , [x_{i- 1} , x_i]) \in \cP_1 \iff c_i = 1/k \text{ for some } k \in \bN.
	\end{align*}
From the definition of the integer $N$, the number of elements in $\cP_1$ is at most $N$ (for any given $\cP$). This comes from the fact that the only number of the form $1/k$ greater than $c$ are $1, 1/2 , 1/3 , \ldots , 1/(N-1)$. Then this implies that
	\begin{align*}
	|S(f, \cP )| = |S (f , \cP_1 ) |= \sum_{i = 1}^{k_1} 1 (x_{j_i} - x_{j_i - 1}) \leq \sum_{i = 1}^N (x_{j_i} - x_{j_i - 1}) < N (\varepsilon /N) = \varepsilon .
	\end{align*}
Thus, if $\Vert \cP \Vert < \delta$ with $\delta := \varepsilon /N$, then $|S (f, \cP )| < \varepsilon$. This means that $f$ is Riemann integrable on $[c, 1]$ and $\int_c^1 f = 0$.

Finally from exerise 5 we conclude that $f$ is Riemann integrable on $[0, 1]$.
\end{sol}

%------------------------------------------------
\begin{exer}
(5 pts)
Show that the function $f : [0, 1] \ra \bR$ defined by $f(x) = 0$ if $x \neq 0$ and $f(x) = 4$ if $x = 0$ is Riemann integrable on $[0, 1]$.
\end{exer}
\begin{sol}
We will show that $\int_0^1 f = 0$.

Let $\varepsilon > 0$ and let $\cP = \{ (c_i , [x_{i - 1} , x_i ]) \, : \, i = 1 , 2, \ldots , N \}$ be a tagged partition of $[0, 1]$ such that $\Vert \cP \Vert < \varepsilon/4$. 

The only tagged interval in $\cP$ that is important is $(c_1, [0, x_1])$ because the number $0$ belongs to $[0, x_1]$. So we have
	\begin{align*}
	S (f, \cP ) = \sum_{i = 1}^N f(c_i) (x_i - x_{i-1}) = f(c_0) (x_1 - x_0) .
	\end{align*}
	The value of $f(c_0)$ is less that $4$ for any choice of $c_0 \in [0, x_1]$. So
	\begin{align*}
	|S (f, \cP)| = |f(c_0)| (x_1 - x_0) \leq 4 \Vert \cP \Vert < \varepsilon .
	\end{align*}
Thus $f$ is Riemann integrable on $[0, 1]$ and $\int_0^1 f = 0$.
\end{sol}

%------------------------------------------------
\begin{exer}
(5 pts)
Let $\cP$ be the following tagged partition of $[-1, 2]$:
	\begin{align*}
	\cP &:= \{ (-9, [-1, -.8]) , (-.7, [-.8, -.3]), (-.1, [-.3, 0]), (.2, [0,0.2]), (.2, [.2, .4]), (.8, [.4, 1]), \\
	& \phantom{:= \{} (1.42, [1, 1.5]), (1.9, [1.5, 2]) \} .
	\end{align*}
Find another partition $\cP_0$ such that $\Vert \cP_0 \Vert \leq \Vert \cP \Vert/3$.
\end{exer}
\begin{sol}
The norm of $\cP$ is $0.6$. Let $\delta = 0.1 < 0.6/3$. Define $\cP_0 := \{ (c_i , [x_{i - 1} , x_i]) \, : \, i = 1, 2, 3, \ldots , 30 \}$ where
	\begin{align*}
	x_i = -1 + i 0.1 \quad (i = 0, 1, 2, \ldots , 30)
	\end{align*}
and
	\begin{align*}
	c_i = x_i \quad (i = 1, 2, \ldots , 30 ) .
	\end{align*}
Then we have $\Vert \cP_1 \Vert = 0.1 \leq \Vert \cP \Vert/3$.
\end{sol}


\end{document}