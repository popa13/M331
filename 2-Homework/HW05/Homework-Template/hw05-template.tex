\documentclass[12pt]{article}
\usepackage[utf8]{inputenc}

\usepackage{enumitem}
\usepackage[margin=2cm]{geometry}

\usepackage{amsmath, amsfonts, amssymb}
\usepackage{graphicx}
\usepackage{tikz}
\usepackage{pgfplots}
\usepackage{multicol}

\usepackage{comment}
\usepackage{url}

\usepackage{titlesec}
\titleformat{\section}[frame]
{\normalfont\scshape}
{\thesection}{8pt}{\centering}

\usepackage{array}

\pgfplotsset{compat=1.16}

%\usepackage[margin=1cm]{cloze}

\usepackage[thmmarks]{ntheorem}

% MATH commands
\newcommand{\bC}{\mathbb{C}}
\newcommand{\bR}{\mathbb{R}}
\newcommand{\bN}{\mathbb{N}}
\newcommand{\bZ}{\mathbb{Z}}
\newcommand{\bT}{\mathbb{T}}
\newcommand{\bD}{\mathbb{D}}
\newcommand{\bQ}{\mathbb{Q}}

\newcommand{\cL}{\mathcal{L}}
\newcommand{\cM}{\mathcal{M}}
\newcommand{\cP}{\mathcal{P}}
\newcommand{\cH}{\mathcal{H}}
\newcommand{\cB}{\mathcal{B}}
\newcommand{\cK}{\mathcal{K}}
\newcommand{\cJ}{\mathcal{J}}
\newcommand{\cU}{\mathcal{U}}
\newcommand{\cO}{\mathcal{O}}
\newcommand{\cA}{\mathcal{A}}
\newcommand{\cC}{\mathcal{C}}

\newcommand{\fK}{\mathfrak{K}}
\newcommand{\fM}{\mathfrak{M}}

\newcommand{\ga}{\left\langle}
\newcommand{\da}{\right\rangle}
\newcommand{\oa}{\left\lbrace}
\newcommand{\fa}{\right\rbrace}
\newcommand{\oc}{\left[}
\newcommand{\fc}{\right]}
\newcommand{\op}{\left(}
\newcommand{\fp}{\right)}

\newcommand{\ra}{\rightarrow}
\newcommand{\Ra}{\Rightarrow}

\renewcommand{\Re}{\mathrm{Re}\,}
\renewcommand{\Im}{\mathrm{Im}\,}
\newcommand{\Arg}{\mathrm{Arg}\,}
\newcommand{\Arctan}{\mathrm{Arctan}\,}
\newcommand{\sech}{\mathrm{sech}\,}
\newcommand{\csch}{\mathrm{csch}\,}
\newcommand{\Log}{\mathrm{Log}\,}
\newcommand{\cis}{\mathrm{cis}\,}

\newcommand{\ran}{\mathrm{ran}\,}
\newcommand{\bi}{\mathbf{i}}
\newcommand{\Sp}{\mathrm{span}\,}
\newcommand{\Inv}{\mathrm{Inv}\,}
\newcommand\smallO{
  \mathchoice
    {{\scriptstyle\mathcal{O}}}% \displaystyle
    {{\scriptstyle\mathcal{O}}}% \textstyle
    {{\scriptscriptstyle\mathcal{O}}}% \scriptstyle
    {\scalebox{.7}{$\scriptscriptstyle\mathcal{O}$}}%\scriptscriptstyle
  }
\newcommand{\HOL}{\mathrm{Hol}}
\newcommand{\cl}{\mathrm{clos}}
\newcommand{\ve}{\varepsilon}

\tikzstyle{myboxT} = [draw=black, fill=black!0,line width = 1pt,
    rectangle, rounded corners = 0pt, inner sep=8pt, inner ysep=8pt]
    
\newcommand{\MyC}[1]{\begin{tikzpicture}
\node (boxIntro) at (0,0) {};
\node [myboxT](Intro) at (boxIntro){%
	\begin{minipage}{0.9\textwidth}
	#1
	\end{minipage}};
\end{tikzpicture}}

%%%%  Environnement exer et solutionnaire
{\theorembodyfont{}
\theoremstyle{plain}
\theoremseparator{\textbf{.}}
\theoremsymbol{}
\newtheorem{exer}{\textbf{Exercise}}}

{\theorembodyfont{\color{black}}
\theoremstyle{plain}
\theoremseparator{\textbf{:}}
\theoremsymbol{$\square$}
\newtheorem*{sol}{\textbf{Solution}}}


\renewcommand*{\theexer}{\arabic{exer}}
\renewcommand*{\thesol}{\arabic{sol}}


\newcommand{\headHW}[4]{%
	\noindent \hrulefill \\
	MATH-#1 #2 \\
	#3 #4
	
}

\begin{document}
	\noindent \hrulefill \\
	MATH-331 Introduction to Real Analysis \hfill YOUR FULL NAME\\
	Homework 05 \hfill Fall 2021\\\vspace*{-0.7cm}
	
	\noindent\hrulefill
	
	\noindent Due date: November 8${}^{\text{th}}$ 1:20pm \hfill Total: \hspace{0.3cm}/70.
	
\vspace*{0.5cm}

	\bgroup \renewcommand{\arraystretch}{1.5}
\begin{table}[h]
\centering
\begin{tabular}{|m{1.5cm}|>{\centering\arraybackslash}p{0.75cm}|>{\centering\arraybackslash}p{0.75cm}|>{\centering\arraybackslash}p{0.75cm}|>{\centering\arraybackslash}p{0.75cm}|>{\centering\arraybackslash}p{0.75cm}|>{\centering\arraybackslash}p{0.75cm}|>{\centering\arraybackslash}p{0.75cm}|>{\centering\arraybackslash}p{0.75cm}|>{\centering\arraybackslash}p{0.75cm}|>{\centering\arraybackslash}p{0.75cm}|}
\hline
Exercise & 1 (5) & 2 (5) & 3 (5) & 4 (5) & 5 (10) & 6 (10) & 7 (5) & 8 (5) & 9 (5) & 10 (10) \\
\hline
Score & & & & & & & & & &  \\\hline
\end{tabular}
\caption{Scores for each exercises}
\end{table}
\egroup
	
\vspace*{0.5cm}

{\bf Instructions:} You must answer all the questions below and send your solution by email (to \url{parisepo@hawaii.edu}). If you decide to not use {\LaTeX} to hand out your solutions, please be sure that after you scan your copy, it is clear and readable. Make sure that you attached a copy of the homework assignment to your homework. 

\noindent If you choose to use {\LaTeX}, you can use the template available on the course website.

\noindent No late homework will be accepted. No format other than PDF will be accepted. Name your file as indicated in the syllabus.

\section{Writing problems}
For each of the following problems, you will be asked to write a clear and detailed proof. You will have the chance to rewrite your solution in your semester project after receiving feedback from me.

%% ---------------------------------------------
\begin{exer}
(5 pts)
Let $f : \bR \ra \bR$ and suppose that there exists a positive constant $M$ such that $|f(y) - f(x)| \leq M |y - x|$ for all $x, y \in \bR$. Prove that $f$ is uniformly continuous on $\bR$.
\end{exer}
\begin{sol}

\end{sol}

%-----------------------------------------------
\begin{exer}
(5 pts)
Let $f : [0, \infty ) \ra \bR$ be nonnegative and continuous such that $\lim_{x \ra \infty} f(x) = 0$. Prove that $f$ attains its maximum at some point in $[0, \infty )$.
\end{exer}
\begin{sol}

\end{sol}

%-----------------------------------------------
\begin{exer}
Suppose that $f: [a, b] \ra \bR$ is a continuous function such that $f([a, b]) \subseteq [a, b]$. Prove that there is a $c \in [a, b]$ such that $f(c) = c$. [This one of the many fixed point Theorem.]
\end{exer}
\begin{sol}

\end{sol}

%-----------------------------------------------
\begin{exer}
(5 pts)
Suppose that $f: (a, b) \ra \bR$ is twice differentiable on $(a,b)$ and there are two points $c  < d$ in $(a ,b)$ such that $f'(c) = f'(d)$. Show that there is a point $x \in (c, d)$ such that $f''(x) = 0$.
\end{exer}
\begin{sol}

\end{sol}

%-----------------------------------------------
\begin{exer}
(10 pts)
Suppose that $f : (a, b) \ra \bR$ is differentiable at $x_0 \in (a, b)$. 
	\begin{enumerate}[label=\textbf{\alph*)}]
	\item Prove that
	\begin{align}
	\lim_{h \ra 0} \frac{f (x_0 + h) - f(x_0 - h)}{2h} \label{eq:SymmetricDerivative}\tag{$\star$}
	\end{align}
exists and equals $f'(x_0)$.
	\item Find a continuous function $f : \bR \ra \bR$ and a point $x_0 \in \bR$ such that $f$ is not differentiable at $x_0$, but the limit \eqref{eq:SymmetricDerivative} exists.
	\end{enumerate}
\end{exer}
\begin{sol}

\end{sol}



\section{Homework problems}
Answer all the questions below. Make sure to show your work.

\begin{exer}
(10pts)
\begin{enumerate}[label=\textbf{\alph*)}]
\item Suppose $r > 0$. Prove that $f : (0, \infty ) \ra \bR$ defined by $f(x) = x^r$ is differentiable on $(0, \infty )$ and compute its derivative. [Hint: take for granted that $e^x$ and $\ln x$ are differentiable with $(e^x)' = e^x$ and $(\ln x)' = 1/x$. Rewrite then $x^r$ in terms of a composition of these two differentiable functions.]
\item Define $f(x) = \sqrt{x^2 + \sin x + \cos x}$ where $x \in [0, \pi/2 ]$. Show that $f$ is a differentiable function.
\end{enumerate}
\end{exer}
\begin{sol}

\end{sol}

%------------------------------------------------
\begin{exer}
(5 pts)
Show that $S \subseteq \bR$ is closed if and only if $\bR \backslash S$ is open.
\end{exer}
\begin{sol}

\end{sol}

%-------------------------------------------------
\begin{exer}
(5 pts)
Let $f : \bR \ra \bR$ be a differentiable function and define $g(x) = x^2 f(x^3)$. Show that $g$ is differentiable and compute its derivative.
\end{exer}
\begin{sol}

\end{sol}

%------------------------------------------------
\begin{exer}
(5 pts)
Prove that $f(x) = \arcsin x$ is differentiable on its domain and find a formula for the derivative of $f$ (justify all your steps!).
\end{exer}
\begin{sol}

\end{sol}

%------------------------------------------------
\begin{exer}
(10 pts)
Use the Mean-Value Theorem to show the following inequalities.
	\begin{enumerate}[label=\textbf{\alph*)}]
	\item $n y^{n-1} (x - y) \leq x^n - y^n \leq n x^{n-1} (x - y)$ if $n \in \bN$ and $0 \leq y \leq x$.
	\item $\sqrt{1 + x} < 1 + \frac{1}{2} x$ for $x > 0$.
	\end{enumerate}
\end{exer}
\begin{sol}

\end{sol}


\end{document}