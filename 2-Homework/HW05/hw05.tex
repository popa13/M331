\documentclass[12pt]{article}
\usepackage[utf8]{inputenc}

\usepackage{enumitem}
\usepackage[margin=2cm]{geometry}

\usepackage{amsmath, amsfonts, amssymb}
\usepackage{graphicx}
\usepackage{tikz}
\usepackage{pgfplots}
\usepackage{multicol}

\usepackage{comment}
\usepackage{url}
\usepackage{calc}

\usepackage{titlesec}
\titleformat{\section}[frame]
{\normalfont\scshape}
{\thesection}{8pt}{\centering}

\usepackage{array}

\pgfplotsset{compat=1.16}

%\usepackage[margin=1cm]{cloze}

\usepackage[thmmarks]{ntheorem}

% MATH commands
\newcommand{\bC}{\mathbb{C}}
\newcommand{\bR}{\mathbb{R}}
\newcommand{\bN}{\mathbb{N}}
\newcommand{\bQ}{\mathbb{Q}}
\newcommand{\bZ}{\mathbb{Z}}
\newcommand{\bT}{\mathbb{T}}
\newcommand{\bD}{\mathbb{D}}

\newcommand{\cL}{\mathcal{L}}
\newcommand{\cM}{\mathcal{M}}
\newcommand{\cP}{\mathcal{P}}
\newcommand{\cH}{\mathcal{H}}
\newcommand{\cB}{\mathcal{B}}
\newcommand{\cK}{\mathcal{K}}
\newcommand{\cJ}{\mathcal{J}}
\newcommand{\cU}{\mathcal{U}}
\newcommand{\cO}{\mathcal{O}}
\newcommand{\cA}{\mathcal{A}}
\newcommand{\cC}{\mathcal{C}}

\newcommand{\fK}{\mathfrak{K}}
\newcommand{\fM}{\mathfrak{M}}

\newcommand{\ga}{\left\langle}
\newcommand{\da}{\right\rangle}
\newcommand{\oa}{\left\lbrace}
\newcommand{\fa}{\right\rbrace}
\newcommand{\oc}{\left[}
\newcommand{\fc}{\right]}
\newcommand{\op}{\left(}
\newcommand{\fp}{\right)}

\newcommand{\ra}{\rightarrow}
\newcommand{\Ra}{\Rightarrow}

\renewcommand{\Re}{\mathrm{Re}\,}
\renewcommand{\Im}{\mathrm{Im}\,}
\newcommand{\Arg}{\mathrm{Arg}\,}
\newcommand{\Arctan}{\mathrm{Arctan}\,}
\newcommand{\sech}{\mathrm{sech}\,}
\newcommand{\csch}{\mathrm{csch}\,}
\newcommand{\Log}{\mathrm{Log}\,}
\newcommand{\cis}{\mathrm{cis}\,}

\newcommand{\ran}{\mathrm{ran}\,}
\newcommand{\bi}{\mathbf{i}}
\newcommand{\Sp}{\mathrm{span}\,}
\newcommand{\Inv}{\mathrm{Inv}\,}
\newcommand\smallO{
  \mathchoice
    {{\scriptstyle\mathcal{O}}}% \displaystyle
    {{\scriptstyle\mathcal{O}}}% \textstyle
    {{\scriptscriptstyle\mathcal{O}}}% \scriptstyle
    {\scalebox{.7}{$\scriptscriptstyle\mathcal{O}$}}%\scriptscriptstyle
  }
\newcommand{\HOL}{\mathrm{Hol}}
\newcommand{\cl}{\mathrm{clos}}
\newcommand{\ve}{\varepsilon}

\tikzstyle{myboxT} = [draw=black, fill=black!0,line width = 1pt,
    rectangle, rounded corners = 0pt, inner sep=8pt, inner ysep=8pt]
    
\newcommand{\MyC}[1]{\begin{tikzpicture}
\node (boxIntro) at (0,0) {};
\node [myboxT](Intro) at (boxIntro){%
	\begin{minipage}{0.9\textwidth}
	#1
	\end{minipage}};
\end{tikzpicture}}

%%%%  Environnement exer et solutionnaire
{\theorembodyfont{}
\theoremstyle{plain}
\theoremseparator{\textbf{.}}
\theoremsymbol{}
\newtheorem{exer}{\textbf{Exercise}}}

{\theorembodyfont{\color{blue}}
\theoremstyle{plain}
\theoremseparator{\textbf{:}}
\theoremsymbol{$\square$}
\newtheorem*{sol}{\textbf{Solution}}}

{\theorembodyfont{\color{blue}}
\theoremstyle{plain}
\theoremseparator{\textbf{:}}
\theoremsymbol{$\qed$}
\newtheorem*{solWP}{\textbf{Solution}}}

{\theorembodyfont{\color{blue}}
\theoremstyle{plain}
\theoremseparator{\textbf{:}}
\theoremsymbol{$\qed$}
\newtheorem*{hint}{\textbf{Hints}}}

\renewcommand*{\theexer}{\arabic{exer}}
\renewcommand*{\thesol}{\arabic{sol}}
\renewcommand*{\thesolWP}{\arabic{solWP}}
\renewcommand*{\thehint}{\arabic{hint}}

%%% Ignorer les solutions
\excludecomment{sol}
\excludecomment{solWP}
\excludecomment{hint}

\newcommand{\headHW}[4]{%
	\noindent \hrulefill \\
	MATH-#1 #2 \\
	#3 #4
	
}

\makeatletter
\DeclareFontFamily{U}{tipa}{}
\DeclareFontShape{U}{tipa}{m}{n}{<->tipa10}{}
\newcommand{\arc@char}{{\usefont{U}{tipa}{m}{n}\symbol{62}}}%

\newcommand{\arc}[1]{\mathpalette\arc@arc{#1}}

\newcommand{\arc@arc}[2]{%
  \sbox0{$\m@th#1#2$}%
  \vbox{
    \hbox{\resizebox{\wd0}{\height}{\arc@char}}
    \nointerlineskip
    \box0
  }%
}
\makeatother

\newcount\QO
\newcounter{QT}
\newcounter{QTh}
\newcounter{QF}
\newcounter{QFi}
\newcounter{QS}
\newcounter{QSe}
\newcounter{QE}
\newcounter{QN}
\newcounter{QTe}

\begin{document}
	\noindent \hrulefill \\
	MATH-331 Introduction to Real Analysis \hfill Created by P.-O. Paris{\'e}\\
	Homework 05 \hfill Fall 2021\\\vspace*{-0.7cm}
	
	\noindent\hrulefill
	
	\noindent Due date: November, 8${}^{\text{th}}$ 1:20pm \hfill Total: \hspace{0.3cm}/70.
	
\vspace*{0.5cm}

	\bgroup \renewcommand{\arraystretch}{1.5}
\begin{table}[h]
\centering
\begin{tabular}{|m{1.5cm}|>{\centering\arraybackslash}p{0.75cm}|>{\centering\arraybackslash}p{0.75cm}|>{\centering\arraybackslash}p{0.75cm}|>{\centering\arraybackslash}p{0.75cm}|>{\centering\arraybackslash}p{0.75cm}|>{\centering\arraybackslash}p{0.75cm}|>{\centering\arraybackslash}p{0.75cm}|>{\centering\arraybackslash}p{0.75cm}|>{\centering\arraybackslash}p{0.75cm}|>{\centering\arraybackslash}p{0.75cm}|}
\hline
Exercise & 1 (5) & 2 (5) & 3 (5) & 4 (5) & 5 (10) & 6 (10) & 7 (5) & 8 (5) & 9 (5) & 10 (10) \\
\hline
Score & & & & & & & & & &  \\\hline
\end{tabular}
\caption{Scores for each exercises}
\end{table}
\egroup
	
\vspace*{0.5cm}

{\bf Instructions:} You must answer all the questions below and send your solution by email (to \url{parisepo@hawaii.edu}). If you decide to not use {\LaTeX} to hand out your solutions, please be sure that after you scan your copy, it is clear and readable. Make sure that you attached a copy of the homework assignment to your homework. 

\noindent If you choose to use {\LaTeX}, you can use the template available on the course website.

\noindent No late homework will be accepted. No format other than PDF will be accepted. Name your file as indicated in the syllabus.

\section{Writing problems}
For each of the following problems, you will be asked to write a clear and detailed proof. You will have the chance to rewrite your solution in your semester project after receiving feedback from me.

%% ---------------------------------------------
\begin{exer}
(5 pts)
Let $f : \bR \ra \bR$ and suppose that there exists a positive constant $M$ such that $|f(y) - f(x)| \leq M |y - x|$ for all $x, y \in \bR$. Prove that $f$ is uniformly continuous on $\bR$.
\end{exer}
\begin{sol}
Let $\varepsilon > 0$ and $\delta := \frac{\varepsilon}{M}$. Let $x, y \in \bR$ such that $|x - y| < \delta$. Then, we have
	\begin{align*}
	|f(x) - f(y)| \leq M |x - y| < M (\delta ) = \varepsilon .
	\end{align*}
So $f$ is uniformly continuous.
\end{sol}

%-----------------------------------------------
\begin{exer}
(5 pts)
Let $f : [0, \infty ) \ra \bR$ be nonnegative and continuous such that $\lim_{x \ra \infty} f(x) = 0$. Prove that $f$ attained its maximum at some point in $[0 , \infty )$.
\end{exer}
\begin{sol}
If $f(x) = 0$ for every $x \in [0, \infty )$, then the result is obvious. Just take $x = 0$ and $f(0) = 0$ is the maximum value of $f$.

Suppose that $f$ is not identically zero. Then there is some $t \in [0, \infty )$ such that $f(t) > 0$. Let $\varepsilon := f(t)/2$. Since $\lim_{x \ra \infty} f(x) = 0$, there is a positive number $M > 0$ such that $|f(x)| < f(t)/2$ if $x > M$. So the maximum value can't occur if $x > M$, that is in the interval $(M, \infty )$.

Since $f$ is continuous on $[0, \infty )$, it is also continuous on the compact set $[0, M]$. We know, from the Extreme Value Theorem, that $f$ attains a maximum at some point, say $c$, in $[0, M]$. 
\end{sol}

%-----------------------------------------------
\begin{exer}
Suppose that $f: [a, b] \ra \bR$ is a continuous function such that $f([a, b]) \subseteq [a, b]$. Prove that there is a $c \in [a, b]$ such that $f(c) = c$. [This is one of the many fixed point Theorem.]
\end{exer}
\begin{sol}
Let $g(x) = f(x) - x$ for $x \in [a, b]$. Then $g$ is a continuous function because it is the sum of two continuous function.

If $f(a) = a$ or $f(b) = b$, we are done. 

Suppose then that $f(a) \neq a$ and $f(b) \neq b$. Since $f(a), f(b) \in [a, b]$, the only possible case is when $f(a) > a$ and $f(b) < b$. So, at those points, we have $g(a) > 0$ and $g(b) < 0$. Since $f$ is continuous, by the IVT, there is a point $c \in (a, b)$ such that $g(c) = 0$, so $f(c) = c$.
\end{sol}

%-----------------------------------------------
\begin{exer}
(5 pts)
Suppose that $f: (a, b) \ra \bR$ is twice differentiable on $(a,b)$ and there are two points $c  < d$ in $(a ,b)$ such that $f'(c) = f'(d)$. Show that there is a point $x \in (c, d)$ such that $f''(x) = 0$.
\end{exer}
\begin{sol}
We have that $f'$ and $f''$ exists on $(a, b)$. From a Theorem in the lecture notes, the fact that $f''$ exists implies that $f'$ is continuous on $(a, b)$. Thus, from Rolle's Theorem, we know that there is some $x \in (c, d)$ such that $f''(x) = 0$.
\end{sol}

%-----------------------------------------------
\begin{exer}
(10 pts)
Suppose that $f : (a, b) \ra \bR$ is differentiable at $x_0 \in (a, b)$. 
	\begin{enumerate}[label=\textbf{\alph*)}]
	\item Prove that
	\begin{align}
	\lim_{h \ra 0} \frac{f (x_0 + h) - f(x_0 - h)}{2h} \label{eq:SymmetricDerivative}\tag{$\star$}
	\end{align}
exists and equals $f'(x_0)$.
	\item Find a continuous function $f : \bR \ra \bR$ and a point $x_0 \in \bR$ such that $f$ is not differentiable at $x_0$, but the limit \eqref{eq:SymmetricDerivative} exists.
	\end{enumerate}
\end{exer}
\begin{sol}
\begin{enumerate}[label=\textbf{\alph*)}]
\item Since $f$ is differentiable at $x_0$, we have that the limit defining $f'(x_0)$ exists. We can rewrite the fraction in the limit \eqref{eq:SymmetricDerivative} in the following way:
	\begin{align*}
	\frac{f(x_0 + h) + f (x_0 - h)}{2h} = \frac{f (x_0 + h) - f(x_0)}{2h} + \frac{ f(x_0 - h) - f(x_0)}{2(-h)} .
	\end{align*}
So, as $h \ra 0$, the first term on the right-hand side exists and is $f'(x_0)/2$ and the second term on the right-hand side exists and is $f'(x_0)/2$ (use the change of variable $t = -h$). So by the sum rule for limit, the limit \eqref{eq:SymmetricDerivative} exists and is equal to $f'(x_0)/2 + f'(x_0)/2 = f'(x_0)$.
\item Take $f(x) = |x|$. Then, $f$ is continuous and $f'(0)$ doesn't exists. However, we have
	\begin{align*}
	\lim_{h \ra 0} \frac{f (h ) - f(-h)}{2h} = \lim_{h \ra 0} \frac{|h| - |h|}{2h} = 0 .
	\end{align*}
\end{enumerate}
\end{sol}



\section{Homework problems}
Answer all the questions below. Make sure to show your work.

\begin{exer}
(10pts)
\begin{enumerate}[label=\textbf{\alph*)}]
\item Suppose $r > 0$. Prove that $f : (0, \infty ) \ra \bR$ defined by $f(x) = x^r$ is differentiable on $(0, \infty )$ and compute its derivative. [Hint: take for granted that $e^x$ and $\ln x$ are differentiable with $(e^x)' = e^x$ and $(\ln x)' = 1/x$. Rewrite then $x^r$ in terms of a composition of two differentiable functions.]
\item Define $f(x) = \sqrt{x^2 + \sin x + \cos x}$ where $x \in [0, \pi/2 ]$. Show that $f$ is a differentiable function and find a formula for its derivative.
\end{enumerate}
\end{exer}
\begin{sol}
\begin{enumerate}[label=\textbf{\alph*)}]
\item We write $f(x) = e^{r\ln x}$. Since $x > 0$, then $\ln x$ is well-defined. Since the exponential function and the logarithmic function are differentiable by assumption, then $f$ is differentiable by the chain rule. We get
	\begin{align*}
	f'(x) = e^{r\ln x} \Big( \frac{r}{x} \Big) = rx^r/x = rx^{r-1} .
	\end{align*}
\item The function $f$ is differentiable on $[0, \pi/2]$ by the chain rule, the sum rule, and the facts that $x^2$, $\cos x$, and $\sin x$ are differentiable functions. By the chain rule, we get
	\begin{align*}
	f'(x) = \frac{2x + \cos x - \sin x}{2 \sqrt{x^2 + \sin x + \cos x}}
	\end{align*}
\end{enumerate}
\end{sol}

%------------------------------------------------
\begin{exer}
(5 pts)
Show that $S \subseteq \bR$ is closed if and only if $\bR \backslash S$ is open.
\end{exer}
\begin{sol}
Suppose that $S$ is closed. Let $x \not\in S$. We want to show that there is a $\delta > 0$ such that $(x - \delta , x + \delta ) \subset S$. Suppose that this is not the case for some $x \not\in S$. Then, for any $\delta > 0$, $(x - \delta , x + \delta ) \nsubseteq \bR \backslash S$. So, there is a $y_{\delta} \in (x + \delta , x - \delta )$ such that $y_\delta \not\in \bR\backslash S$, so in $S$. Take $\delta = 1/n$. Then $|y_n - x| < 1/n$ and $y_n \in S$ for any $n \geq 1$. Then $y_n \ra x$ and since $S$ is closed, we must have that $x \in S$, a contradiction. So $\bR \backslash S$ is open.

Suppose that $\bR \backslash S$ is open. Suppose, if possible, that there is a sequence $(x_n)_{n =1}^\infty \subseteq S$ such that $x_n \ra x$, but $x \not\in S$. Since $x \not\in S$ and $\bR\backslash S$ is open, there is a $\delta > 0$ such that $(x - \delta , x + \delta ) \subset \bR \backslash S$. Since $x_n \ra x$, there is a $N \in \bN$ such that for any $n \geq N$, $x_n \in (x - \delta , x + \delta )$. So, for any $n \geq N$, $x_n \not\in S$, which is a contradiction with the fact that $x_n \in S$ for any $n \geq 1$. So $S$ is closed.
\end{sol}

%-------------------------------------------------
\begin{exer}
(5 pts)
Let $f : \bR \ra \bR$ be a differentiable function and define $g(x) = x^2 f(x^3)$. Show that $g$ is differentiable and compute its derivative.
\end{exer}
\begin{sol}
We know that $x^n$ ($n \geq 0$) is differentiable. By the chain rule and the product rule, so is $g$. Then, we get
	\begin{align*}
	g'(x) = 2x f(x^3) + x^2 f'(x^3) 3x^2 = 2x f(x^3) + 3x^4 f(x^3) .
	\end{align*}
\end{sol}

%------------------------------------------------
\begin{exer}
(5 pts)
Prove that $f(x) = \arcsin x$ is differentiable on its domain and find a formula, involving no trigonometric functions, for the derivative of $f$ (justify all your steps!).
\end{exer}
\begin{sol}
The function $f$ is the inverse function of $\sin$ when $x$ is restricted to the interval $[-1, 1]$. So, since $\sin$ is strictly increasing on $[-\pi/2, \pi/2]$ and differentiable on $(-\pi /2 , \pi /2 )$, the function $\arcsin$ is also strictly increasing and differentiable on $[-1, 1]$ and $(-1 , 1)$ respectively. For every $x \in (-1, 1)$, we have
	\begin{align*}
	f'(x) = \frac{1}{\cos (\arcsin (x))} .
	\end{align*}
But, $\arcsin (x)$ is the angle of the right-triangle formed by the sides $\cos (\arcsin (x))$, $x$ and $1$. So, we get, from Pythagorean Theorem, that $\cos (\arcsin (x)) = \sqrt{1 - x^2}$. Thus, 
	\begin{align*}
	f'(x)= \frac{1}{\sqrt{1 - x^2}} .
	\end{align*}
\end{sol}

%------------------------------------------------
\begin{exer}
(10 pts)
Use the Mean-Value Theorem to show the following inequalities.
	\begin{enumerate}[label=\textbf{\alph*)}]
	\item $n y^{n-1} (x - y) \leq x^n - y^n \leq n x^{n-1} (x - y)$ if $n \in \bN$ and $0 \leq y \leq x$.
	\item $\sqrt{1 + x} < 1 + \frac{1}{2} x$ for $x > 0$.
	\end{enumerate}
\end{exer}
\begin{sol}
\begin{enumerate}[label=\textbf{\alph*)}]
\item If $n = 1$, then the inequalities reduce to $1 \leq 1 \leq 1$. 

When $n \in \bN$ and $x = y$, then the inequalities reduce to $0 \leq 0 \leq 0$. 

Suppose that $n > 1$ and let $0 \leq y < x$. Put $f(t) = t^n$ defined on $[x, y]$. The function $f$ is differentiable on $(x, y)$. By the MVT, there is a $c \in (x, y)$ such that
	\begin{align*}
	f'(c) = \frac{f(x) - f(y)}{y - x} .
	\end{align*}
Now, $f'(t) = n t^{n-1}$. Since $n - 1 \geq 1$ and $c \in (x, y)$, we have $y^{n-1} \leq c^{n-1} \leq x^{n-1}$. So, we get $n y^{n-1} \leq f'(c) \leq n x^{n-1}$ and so
	\begin{align*}
	n y^{n-1} \leq \frac{f(x) - f(y)}{x - y} \leq n x^{n-1}
	\end{align*}
and since $y < x$, we get
	\begin{align*}
	n y^{n-1} (x- y) \leq f(x) - f(y) \leq nx^{n-1} (x - y) .
	\end{align*}
	\item Suppose $x > 0$ and consider the function $f(t) = \sqrt{1 + t}$ for $t \in (0, x)$. By the exercise 6 a) and the chain rule, we know that $f$ is differentiable on $(0, x]$. By the MVT, there is a $c \in (0, x)$ such that
		\begin{align*}
		f'(c) = \frac{f(x) - f(0)}{x} = \frac{\sqrt{1 + x} - 1}{x} .
		\end{align*}
	We have $f'(t) = \frac{1}{2\sqrt{1 + t}}$. Since $c \in (0, x)$, we have $1 + c > 1$ and so since the square root function is strictly increasing, we get $\sqrt{1 + c} > 1$. Then
		\begin{align*}
		f'(c) = \frac{1}{2 \sqrt{1 + c}} < \frac{1}{2} .
		\end{align*}
	So, we get
		\begin{align*}
		\frac{\sqrt{1 + x} - 1}{x} < \frac{1}{2}
		\end{align*}
	which can be rewritten as $\sqrt{1 + x} < 1 + x/2$.
\end{enumerate}
\end{sol}


\end{document}