\documentclass[12pt]{article}
\usepackage[utf8]{inputenc}

\usepackage{enumitem}
\usepackage[margin=2cm]{geometry}

\usepackage{amsmath, amsfonts, amssymb}
\usepackage{graphicx}
\usepackage{tikz}
\usepackage{pgfplots}
\usepackage{multicol}

\usepackage{comment}
\usepackage{url}

\usepackage{titlesec}
\titleformat{\section}[frame]
{\normalfont\scshape}
{\thesection}{8pt}{\centering}

\usepackage{array}

\pgfplotsset{compat=1.16}

%\usepackage[margin=1cm]{cloze}

\usepackage[thmmarks]{ntheorem}

% MATH commands
\newcommand{\bC}{\mathbb{C}}
\newcommand{\bR}{\mathbb{R}}
\newcommand{\bN}{\mathbb{N}}
\newcommand{\bZ}{\mathbb{Z}}
\newcommand{\bT}{\mathbb{T}}
\newcommand{\bD}{\mathbb{D}}

\newcommand{\cL}{\mathcal{L}}
\newcommand{\cM}{\mathcal{M}}
\newcommand{\cP}{\mathcal{P}}
\newcommand{\cH}{\mathcal{H}}
\newcommand{\cB}{\mathcal{B}}
\newcommand{\cK}{\mathcal{K}}
\newcommand{\cJ}{\mathcal{J}}
\newcommand{\cU}{\mathcal{U}}
\newcommand{\cO}{\mathcal{O}}
\newcommand{\cA}{\mathcal{A}}
\newcommand{\cC}{\mathcal{C}}

\newcommand{\fK}{\mathfrak{K}}
\newcommand{\fM}{\mathfrak{M}}

\newcommand{\ga}{\left\langle}
\newcommand{\da}{\right\rangle}
\newcommand{\oa}{\left\lbrace}
\newcommand{\fa}{\right\rbrace}
\newcommand{\oc}{\left[}
\newcommand{\fc}{\right]}
\newcommand{\op}{\left(}
\newcommand{\fp}{\right)}

\newcommand{\ra}{\rightarrow}
\newcommand{\Ra}{\Rightarrow}

\renewcommand{\Re}{\mathrm{Re}\,}
\renewcommand{\Im}{\mathrm{Im}\,}
\newcommand{\Arg}{\mathrm{Arg}\,}
\newcommand{\Arctan}{\mathrm{Arctan}\,}
\newcommand{\sech}{\mathrm{sech}\,}
\newcommand{\csch}{\mathrm{csch}\,}
\newcommand{\Log}{\mathrm{Log}\,}
\newcommand{\cis}{\mathrm{cis}\,}

\newcommand{\ran}{\mathrm{ran}\,}
\newcommand{\bi}{\mathbf{i}}
\newcommand{\Sp}{\mathrm{span}\,}
\newcommand{\Inv}{\mathrm{Inv}\,}
\newcommand\smallO{
  \mathchoice
    {{\scriptstyle\mathcal{O}}}% \displaystyle
    {{\scriptstyle\mathcal{O}}}% \textstyle
    {{\scriptscriptstyle\mathcal{O}}}% \scriptstyle
    {\scalebox{.7}{$\scriptscriptstyle\mathcal{O}$}}%\scriptscriptstyle
  }
\newcommand{\HOL}{\mathrm{Hol}}
\newcommand{\cl}{\mathrm{clos}}
\newcommand{\ve}{\varepsilon}

\tikzstyle{myboxT} = [draw=black, fill=black!0,line width = 1pt,
    rectangle, rounded corners = 0pt, inner sep=8pt, inner ysep=8pt]
    
\newcommand{\MyC}[1]{\begin{tikzpicture}
\node (boxIntro) at (0,0) {};
\node [myboxT](Intro) at (boxIntro){%
	\begin{minipage}{0.9\textwidth}
	#1
	\end{minipage}};
\end{tikzpicture}}

%%%%  Environnement exer et solutionnaire
{\theorembodyfont{}
\theoremstyle{plain}
\theoremseparator{\textbf{.}}
\theoremsymbol{}
\newtheorem{exer}{\textbf{Exercise}}}

{\theorembodyfont{\color{blue}}
\theoremstyle{plain}
\theoremseparator{\textbf{:}}
\theoremsymbol{$\square$}
\newtheorem*{sol}{\textbf{Solution}}}


\renewcommand*{\theexer}{\arabic{exer}}
\renewcommand*{\thesol}{\arabic{sol}}


\newcommand{\headHW}[4]{%
	\noindent \hrulefill \\
	MATH-#1 #2 \\
	#3 #4
	
}

\begin{document}
	\noindent \hrulefill \\
	MATH-331 Introduction to Real Analysis \hfill YOUR FULL NAME\\
	Homework 02 \hfill Fall 2021\\\vspace*{-0.7cm}
	
	\noindent\hrulefill
	
	\noindent Due date: 20-09-2021 1:20pm \hfill Total: \hspace{0.3cm}/70.
	
\vspace*{0.5cm}

	\bgroup \renewcommand{\arraystretch}{1.5}
\begin{table}[h]
\centering
\begin{tabular}{|m{1.5cm}|>{\centering\arraybackslash}p{0.75cm}|>{\centering\arraybackslash}p{0.75cm}|>{\centering\arraybackslash}p{0.75cm}|>{\centering\arraybackslash}p{0.75cm}|>{\centering\arraybackslash}p{0.75cm}|>{\centering\arraybackslash}p{0.75cm}|>{\centering\arraybackslash}p{0.75cm}|>{\centering\arraybackslash}p{0.75cm}|>{\centering\arraybackslash}p{0.75cm}|>{\centering\arraybackslash}p{0.75cm}|}
\hline
Exercise & 1 (10) & 2 (5) & 3 (5) & 4 (5) & 5 (5) & 6 (10) & 7 (5) & 8 (10) & 9 (5) & 10 (10) \\
\hline
Score & & & & & & & & & &  \\\hline
\end{tabular}
\caption{Scores for each exercises}
\end{table}
\egroup
	
\vspace*{0.5cm}

{\bf Instructions:} You must answer all the questions below and send your solution by email (to \url{parisepo@hawaii.edu}). If you decide to not use {\LaTeX} to hand out your solutions, please be sure that after you scan your copy, it is clear and readable. Make sure that you attached a copy of the homework assignment to your homework. No late homework will be accepted. No format other than PDF will be accepted. Name your file as indicated in the syllabus.

\section{Writing problems}
For each of the following problems, you will be asked to write a clear and detailed proof. You will have the chance to rewrite your solution in your semester project after receiving feedback from me.

\begin{exer}
(10 pts)
\begin{enumerate}[label=\textbf{\alph*)}]
\item Let $\{ [a_n , b_n ] \, : \, n \geq 1\}$ be a family of closed intervals such that $[a_1 , b_1] \supset [a_2 , b_2] \supset [a_3, b_3] \supset \cdots$. Show that there is a $c \in \bR$ such that $c \in [a_n , b_n ]$ for all $n \geq \bN$. Follow the following steps to prove it:
	\begin{enumerate}[label=\textbf{(\roman*)}]
	\item Prove that for any $n, m \geq 1$, $a_n \leq b_m$. [hint: put $M := \max \{ n , m \}$.]
	\item Show that $\sup \{ a_n\, : \, n \geq 1 \}$ exists.
	\item Show that $c = \sup \{ a_n \, : \, n \geq 1\}$ satisfies the requirement.	
	\end{enumerate}
\item Use this last result to prove that the set $\bR$ is uncountable. [Hint: Show that any function $f : \bN \ra \bR$ can't be surjective. To do so, construct a sequence of closed intervals such that $f(n) \not\in [a_n , b_n]$ with $a_n < b_n$.]
\end{enumerate}
\end{exer}
\begin{sol}

\end{sol}


\begin{exer}
(5 pts)
Prove that if $a_n \ra A$, then $|a_n| \ra |A|$.
\end{exer}
\begin{sol}

\end{sol}

\begin{exer}
(5 pts)
Let $(a_n)$, $(b_n)$, and $(c_n)$ be sequences of real numbers. Prove that if $a_n \ra L$, $b_n \ra L$, and $a_n \leq c_n \leq b_n$, then $c_n \ra L$.
\end{exer}
\begin{sol}

\end{sol}

\begin{exer}
(5 pts)
Prove that if $a_n \ra A$ and $a_n \geq 0$ for all $n \geq 1$, then $\sqrt{a_n} \ra \sqrt{A}$. Follow the following steps to prove it:
	\begin{enumerate}
	\item Consider the case $A = 0$.
	\item Suppose that $A \neq 0$. Show that there is a $N_1 \in \bN$ such that if $n \geq N_1$, then $\sqrt{a_n} \geq \sqrt{|A|/2}$. [Hint: use the definition of convergence of $(a_n)_{n \geq 0}$ with a clever choice of $\varepsilon$ and use the properties of the absolute value.]
	\item Use the convergence of $(a_n)$ again to find a $N_2$ such that $|a_n - A| < \frac{3}{4} \frac{\varepsilon}{\sqrt{|A|}}$. 
	\item Express $\sqrt{a_n} - A$ as $\frac{a_n - A}{\sqrt{a_n} + \sqrt{A}}$ and put $N = \max \{ N_1 , N_2 \}$. Conclude.
	\end{enumerate}
\end{exer}
\begin{sol}

\end{sol}


\begin{exer}
(5 pts)
For each sequence $(a_n)_{n = 1}^\infty$, define the sequence $(\sigma_n)_{n = 1}^\infty$ by
	\begin{align*}
	\sigma_n := \frac{a_1 + a_2 + \cdots + a_n}{n} \quad (n \geq 1 ) .
	\end{align*}
Prove that if $a_n \ra A$, then $\sigma_n \ra A$. Find an example of a divergent sequence $(a_n)$ such that $(\sigma_n)_{n = 1}^\infty$ converges.
\end{exer}
\begin{sol}

\end{sol}

\section{Homework problems}
\begin{exer}
(10 pts)
Use the definition of convergence to prove that each of the following sequences converges.
	\begin{enumerate}[label=\textbf{\alph*)}]
	\item $(a_n )_{n = 1}^\infty$ given by $a_n = 5 + 1/n$ for $n \geq 1$.
	\item $(a_n)_{n = 1}^\infty$ given by $a_n = \frac{3n}{2n + 1}$ for $n \geq 1$.
	\end{enumerate}
\end{exer}
\begin{sol}

\end{sol}

\begin{exer}
(5 pts)
Prove that the sequence $(a_n)_{n = 1}^\infty = \Big( \frac{2n + 1}{n} \Big)_{n = 1}^\infty$ is a Cauchy sequence.
\end{exer}
\begin{sol}

\end{sol}

\begin{exer}
(10 pts)
Prove that each of the following sequence diverges.
	\begin{enumerate}[label=\textbf{\alph*)}]
	\item $(a_n)_{n = 1}^\infty = ( (-1)^n )_{n = 1}^\infty$.
	\item $(a_n)_{n = 1}^\infty = ( \sin (\frac{4n + 1}{2} \pi) )_{n = 1}^{\infty}$.
	\end{enumerate}
\end{exer}
\begin{sol}

\end{sol}

\begin{exer}
(5 pts)
Give an examples of two sequences $(a_n)$ and $(b_n)$ such that $(a_n)$ and $(b_n)$ don't converge, but $(a_n + b_n)$ converge.
\end{exer}
\begin{sol}

\end{sol}

\begin{exer}
(10 pts)
With the limit operations and the writing problems, find the limit of the following sequence with general term
	\begin{enumerate}[label=\textbf{\alph*)}]
	\item $\frac{n^2 + 4n}{n^2 - 5}$.
	\item $\frac{n}{n^2 - 3}$.
	\item $\frac{\cos n}{n}$. [You can use what you know on the cosine function.]
	\item $\Big( \sqrt{4 - \frac{1}{n}} - 2 \Big) n$.
	\end{enumerate}
\end{exer}
\begin{sol}

\end{sol}

\end{document}