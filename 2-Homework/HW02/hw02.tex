\documentclass[12pt]{article}
\usepackage[utf8]{inputenc}

\usepackage{enumitem}
\usepackage[margin=2cm]{geometry}

\usepackage{amsmath, amsfonts, amssymb}
\usepackage{graphicx}
\usepackage{tikz}
\usepackage{pgfplots}
\usepackage{multicol}

\usepackage{comment}
\usepackage{url}

\usepackage{titlesec}
\titleformat{\section}[frame]
{\normalfont\scshape}
{\thesection}{8pt}{\centering}

\usepackage{array}

\pgfplotsset{compat=1.16}

%\usepackage[margin=1cm]{cloze}

\usepackage[thmmarks]{ntheorem}

% MATH commands
\newcommand{\bC}{\mathbb{C}}
\newcommand{\bR}{\mathbb{R}}
\newcommand{\bN}{\mathbb{N}}
\newcommand{\bZ}{\mathbb{Z}}
\newcommand{\bT}{\mathbb{T}}
\newcommand{\bD}{\mathbb{D}}

\newcommand{\cL}{\mathcal{L}}
\newcommand{\cM}{\mathcal{M}}
\newcommand{\cP}{\mathcal{P}}
\newcommand{\cH}{\mathcal{H}}
\newcommand{\cB}{\mathcal{B}}
\newcommand{\cK}{\mathcal{K}}
\newcommand{\cJ}{\mathcal{J}}
\newcommand{\cU}{\mathcal{U}}
\newcommand{\cO}{\mathcal{O}}
\newcommand{\cA}{\mathcal{A}}
\newcommand{\cC}{\mathcal{C}}

\newcommand{\fK}{\mathfrak{K}}
\newcommand{\fM}{\mathfrak{M}}

\newcommand{\ga}{\left\langle}
\newcommand{\da}{\right\rangle}
\newcommand{\oa}{\left\lbrace}
\newcommand{\fa}{\right\rbrace}
\newcommand{\oc}{\left[}
\newcommand{\fc}{\right]}
\newcommand{\op}{\left(}
\newcommand{\fp}{\right)}

\newcommand{\ra}{\rightarrow}
\newcommand{\Ra}{\Rightarrow}

\renewcommand{\Re}{\mathrm{Re}\,}
\renewcommand{\Im}{\mathrm{Im}\,}
\newcommand{\Arg}{\mathrm{Arg}\,}
\newcommand{\Arctan}{\mathrm{Arctan}\,}
\newcommand{\sech}{\mathrm{sech}\,}
\newcommand{\csch}{\mathrm{csch}\,}
\newcommand{\Log}{\mathrm{Log}\,}
\newcommand{\cis}{\mathrm{cis}\,}

\newcommand{\ran}{\mathrm{ran}\,}
\newcommand{\bi}{\mathbf{i}}
\newcommand{\Sp}{\mathrm{span}\,}
\newcommand{\Inv}{\mathrm{Inv}\,}
\newcommand\smallO{
  \mathchoice
    {{\scriptstyle\mathcal{O}}}% \displaystyle
    {{\scriptstyle\mathcal{O}}}% \textstyle
    {{\scriptscriptstyle\mathcal{O}}}% \scriptstyle
    {\scalebox{.7}{$\scriptscriptstyle\mathcal{O}$}}%\scriptscriptstyle
  }
\newcommand{\HOL}{\mathrm{Hol}}
\newcommand{\cl}{\mathrm{clos}}
\newcommand{\ve}{\varepsilon}

\tikzstyle{myboxT} = [draw=black, fill=black!0,line width = 1pt,
    rectangle, rounded corners = 0pt, inner sep=8pt, inner ysep=8pt]
    
\newcommand{\MyC}[1]{\begin{tikzpicture}
\node (boxIntro) at (0,0) {};
\node [myboxT](Intro) at (boxIntro){%
	\begin{minipage}{0.9\textwidth}
	#1
	\end{minipage}};
\end{tikzpicture}}

%%%%  Environnement exer et solutionnaire
{\theorembodyfont{}
\theoremstyle{plain}
\theoremseparator{\textbf{.}}
\theoremsymbol{}
\newtheorem{exer}{\textbf{Exercise}}}

{\theorembodyfont{\color{blue}}
\theoremstyle{plain}
\theoremseparator{\textbf{:}}
\theoremsymbol{$\square$}
\newtheorem*{sol}{\textbf{Solution}}}

{\theorembodyfont{\color{blue}}
\theoremstyle{plain}
\theoremseparator{\textbf{:}}
\theoremsymbol{$\square$}
\newtheorem*{solWP}{\textbf{Solution}}}

{\theorembodyfont{\color{blue}}
\theoremstyle{plain}
\theoremseparator{\textbf{:}}
\theoremsymbol{$\square$}
\newtheorem*{hint}{\textbf{Hints}}}

\renewcommand*{\theexer}{\arabic{exer}}
\renewcommand*{\thesol}{\arabic{sol}}
\renewcommand*{\thesolWP}{\arabic{solWP}}
\renewcommand*{\thehint}{\arabic{hint}}

%%% Ignorer les solutions
\excludecomment{sol}
\excludecomment{solWP}
\excludecomment{hint}

\newcommand{\headHW}[4]{%
	\noindent \hrulefill \\
	MATH-#1 #2 \\
	#3 #4
	
}

\begin{document}
	\noindent \hrulefill \\
	MATH-331 Introduction to Real Analysis \hfill Pierre-Olivier Paris{\'e}\\
	Homework 02 \hfill Fall 2021\\\vspace*{-0.7cm}
	
	\noindent\hrulefill
	
	\noindent Due date: 20-09-2021 1:20pm \hfill Total: \hspace{0.3cm}/70.
	
\vspace*{0.5cm}

	\bgroup \renewcommand{\arraystretch}{1.5}
\begin{table}[h]
\centering
\begin{tabular}{|m{1.5cm}|>{\centering\arraybackslash}p{0.75cm}|>{\centering\arraybackslash}p{0.75cm}|>{\centering\arraybackslash}p{0.75cm}|>{\centering\arraybackslash}p{0.75cm}|>{\centering\arraybackslash}p{0.75cm}|>{\centering\arraybackslash}p{0.75cm}|>{\centering\arraybackslash}p{0.75cm}|>{\centering\arraybackslash}p{0.75cm}|>{\centering\arraybackslash}p{0.75cm}|>{\centering\arraybackslash}p{0.75cm}|}
\hline
Exercise & 1 (10) & 2 (5) & 3 (5) & 4 (5) & 5 (5) & 6 (10) & 7 (5) & 8 (10) & 9 (5) & 10 (10) \\
\hline
Score & & & & & & & & & &  \\\hline
\end{tabular}
\caption{Scores for each exercises}
\end{table}
\egroup
	
\vspace*{0.5cm}

{\bf Instructions:} You must answer all the questions below and send your solution by email (to \url{parisepo@hawaii.edu}). If you decide to not use {\LaTeX} to hand out your solutions, please be sure that after you scan your copy, it is clear and readable. Make sure that you attached a copy of the homework assignment to your homework. No late homework will be accepted. No format other than PDF will be accepted. Name your file as indicated in the syllabus.

\section{Writing problems}
For each of the following problems, you will be asked to write a clear and detailed proof. You will have the chance to rewrite your solution in your semester project after receiving feedback from me.

\begin{exer}
(10 pts)
\begin{enumerate}[label=\textbf{\alph*)}]
\item Let $\{ [a_n , b_n ] \, : \, n \geq 1\}$ be a family of closed intervals such that $[a_1 , b_1] \supset [a_2 , b_2] \supset [a_3, b_3] \supset \cdots$. Show that there is a $c \in \bR$ such that $c \in [a_n , b_n ]$ for all $n \geq \bN$. Follow the following steps to prove it:
	\begin{enumerate}[label=\textbf{(\roman*)}]
	\item Prove that for any $n, m \geq 1$, $a_n \leq b_m$. [hint: put $M := \max \{ n , m \}$.]
	\item Show that $\sup \{ a_n\, : \, n \geq 1 \}$ exists.
	\item Show that $c = \sup \{ a_n \, : \, n \geq 1\}$ satisfies the requirement.	
	\end{enumerate}
\item Use this last result to prove that the set $\bR$ is uncountable. [Hint: Show that any function $f : \bN \ra \bR$ can't be surjective. To do so, construct a sequence of closed intervals such that $f(n) \not\in [a_n , b_n]$ with $a_n < b_n$.]
\end{enumerate}
\end{exer}
\begin{sol}
\begin{enumerate}[label=\textbf{\alph*)}]
\item Let $A := \{ a_n \, : \, n \geq 1 \}$. By the hypothesis, we know that $a_1 \leq a_2 \leq a_3 \leq \cdots$, $b_1 \geq b_2 \geq b_3 \geq \cdots$ and $a_n \leq b_n$ for any $n \geq 1$. In fact, if $n, m \geq 1$ and $M := \max \{ n, m \}$, then
	\begin{align*}
	a_n \leq a_M \leq b_M \leq b_m .
	\end{align*}
So we have $a_n \leq b_m$ for any $n, m \geq 1$. This implies that for any $m \geq 1$, the number $b_m$ is an upper bound for $A$. So, by AC, $\sup A$ exists. Put $c := \sup A$. We will verify that $c$ satisfy all the requirements. Since $c$ is the supremum of $A$, we have $a_n \leq c$ for any $n \geq 1$. Also, since $b_m$ is an upper bound of $A$ for any $m \geq 1$, by the definition of the supremum, we get that $c \leq b_m$ for any $m \geq 1$. Thus, for any $n \geq 1$, we get $a_n \leq c \leq b_n$. In other words, this means $c \in [a_n , b_n]$ for any $n \geq 1$.
\item Let $f : \bN \ra \bR$ be a function. Since $f(1) \in \bR$, there are real numbers $a_1$ and $b_1$ such that $a_1 < b_1 < f(1)$ (just take $a_1 = f(1) - \varepsilon$ and $b_1 = f(1) - \varepsilon/2$ for $\varepsilon > 0$). Now let $a_2, b_2 \in \bR$ such that $a_2 < b_2$ and $[a_2, b_2] \subset [a_1 , b_1]$ and $f(2) \not\in [a_2 , b_2]$. This is possible because
	\begin{itemize}
	\item if $f(2) \not\in (a_1 , b_1)$, then take $a_2 = a_1/2$ and $b_2 = b_1/2$.
	\item if $f(2) \in (a_1 , b_1)$, then by the density of the rational numbers, there are rational numbers $r, \tilde{r}$ such that $a_1 < r < \tilde{r} < f(2) < b_1$. Set $a_2 = r$ and $b_2 = \tilde{r}$.
	\end{itemize}
Continue in this fashion to construct a sequence of intervals $[a_n , b_n]$ such that 
	\begin{itemize}
	\item $[a_1 , b_1] \supset [a_2 , b_2] \supset [a_3, b_3] \supset \cdots$ and
	\item $f(n) \not\in [a_n , b_n]$ for every $n \geq 1$.
	\end{itemize}
By a), there is some $c \in \bR$ such that $c \in [a_n , b_n ]$ for every $n \geq 1$. This implies that $c \neq f(n)$ for every $n \geq 1$ and so $f$ is not surjective.
\end{enumerate}
\end{sol}

\begin{exer}
(5 pts)
Prove that if $a_n \ra A$, then $|a_n| \ra |A|$.
\end{exer}
\begin{sol}
Let $a_n \ra A$. This means that for any $\varepsilon > 0$, there exists a $N \in \bN$ such that if $n \geq N$, then $|a_n - A| < \varepsilon$. Let $\varepsilon > 0$ be arbitrary. We know, from the definition of convergence, that there is a $N \in \bN$ such that if $n \geq N$, then $|a_n - A| < \varepsilon$. Let $n \geq N$, then, by the properties of the absolute value, we have
	\begin{align*}
	| |a_n | - |A|| \leq |a_n - A| < \varepsilon .
	\end{align*}
So, for any $\varepsilon > 0$, there is a $N \in \bN$ such that if $n \geq N$, then $||a_n| - |A|| < \varepsilon$. Since $\varepsilon > 0$ was arbitrary, we conclude that $|a_n| \ra |A|$.
\end{sol}

\begin{exer}
(5 pts)
Let $(a_n)$, $(b_n)$, and $(c_n)$ be sequences of real numbers. Prove that if $a_n \ra L$, $b_n \ra L$, and $a_n \leq c_n \leq b_n$, then $c_n \ra L$.
\end{exer}
\begin{sol}
Let $(a_n)$, $(b_n)$ and $(c_n)$ be sequences such that $a_n \ra L$ and $b_n \ra L$. Suppose also that $a_n \leq c_n \leq b_n$ for any $n \geq 1$. We want to prove that $c_n \ra L$. Let $\varepsilon > 0$. Then, from the definition of convergence, there are $N_A , N_B \in \bN$ such that
	\begin{itemize}
	\item if $n \geq N_A$, then $|a_n - L| < \varepsilon$ and;
	\item if $n \geq N_B$, then $|b_n - L| < \varepsilon$.
	\end{itemize}
These last inequalities are equivalent to $-\varepsilon < a_n - L < \varepsilon$ for $n \geq N_A$ and $-\varepsilon < b_n - L < \varepsilon$ for $n \geq N_B$. The goal is to prove that $|c_n - L| < \varepsilon$. Now, from the hypothesis, we know that $a_n \leq c_n \leq b_n$ for any $n \geq 1$. So, for such $n$, we have
	\begin{align*}
	a_n - L \leq c_n - L \leq b_n - L
	\end{align*}
Take $N := \max \{ N_A , N_B \}$. Then, if $n \geq N \geq N_A$, we get
	\begin{align*}
	a_n - L > -\varepsilon \quad \Ra \quad c_n - L > -\varepsilon .
	\end{align*}
Also, if $n \geq N \geq N_B$, we get
	\begin{align*}
	b_n - L < \varepsilon \quad \Ra \quad c_n - L < \varepsilon .
	\end{align*}
So, combining these last two inequalities, if $n \geq N$, then $-\varepsilon < c_n - L < \varepsilon$. This is the same thing as $|c_n - L| < \varepsilon$. Thus, we have just shown that if $\varepsilon > 0$, then there is a $N \in \bN$ such that if $n \geq N$, then $|c_n - L| < \varepsilon$. Since $\varepsilon$ was arbitrary, we conclude that $c_n \ra L$.
\end{sol}


\begin{exer}
(5 pts)
Prove that if $a_n \ra A$ and $a_n \geq 0$ for all $n \geq 1$, then $\sqrt{a_n} \ra \sqrt{A}$. Follow the following steps to prove it:
	\begin{enumerate}
	\item Consider the case $A = 0$.
	\item Suppose that $A \neq 0$. Show that there is a $N_1 \in \bN$ such that if $n \geq N_1$, then $\sqrt{a_n} \geq \sqrt{|A|/2}$. [Hint: use the definition of convergence of $(a_n)_{n \geq 0}$ with a clever choice of $\varepsilon$ and use the properties of the absolute value.]
	\item Use the convergence of $(a_n)$ again to find a $N_2$ such that $|a_n - A| < \frac{3}{4} \frac{\varepsilon}{\sqrt{|A|}}$. 
	\item Express $\sqrt{a_n} - A$ as $\frac{a_n - A}{\sqrt{a_n} + \sqrt{A}}$ and put $N = \max \{ N_1 , N_2 \}$. Conclude.
	\end{enumerate}
\end{exer}
\begin{sol}
Let $A= 0$ and $\varepsilon > 0$. Then, $a_n \ra 0$ and this implies that there is a $N \in \bN$ such that if $n \geq N$, then $|a_n| < \varepsilon^2$. We know that $a_n \geq 0$, then $|a_n| = a_n$. Taking the square root in the last expression gives $\sqrt{a_n} < \varepsilon$ if $n \geq N$. So, for $\varepsilon > 0$, there is a $N \in \bN$ such that if $n \geq N$, then $|\sqrt{a_n}| < \varepsilon$. In other words, $\sqrt{a_n} \ra \sqrt{0} = 0$.

Let $A \neq 0$ and $\varepsilon > 0$. For $n \geq 1$, we have
	\begin{equation}
	\sqrt{a_n} - \sqrt{A} = \frac{a_n - A}{\sqrt{a_n} + \sqrt{A}} .  \label{eq:sqrt}
	\end{equation}
Since $a_n \ra A$ with $A \neq 0$, there is a $N_1 \in \bN$ such that if $n \geq N$, then $|a_n - A| < \frac{|A|}{2}$ (take $\varepsilon = |A|/2 > 0$ in the definition of convergence). Now, by the properties of the absolute value, we have $|A| - |a_n| \leq ||a_n| - |A|| \leq |a_n - A|$. So, if $n \geq N_1$, then
	\begin{align*}
	|A| - |a_n| < \frac{|A|}{2} \quad \Ra \quad \frac{|A|}{2} < |a_n| .
	\end{align*}
Taking the square root on each side of the inequality, we obtain $\sqrt{a_n} > \sqrt{\frac{|A|}{2}}$ if $n \geq N_1$. Since $\sqrt{2} < 2$, we also see that $\sqrt{|A|/2} \geq \sqrt{|A|}{2}$. So, 
	\begin{equation}
	n \geq N_1 \quad \Ra \quad \sqrt{a_n} \geq \frac{\sqrt{|A|}}{2} . \label{eq:lowerbound}
	\end{equation}

By the definition of the convergence of the sequence $(a_n)$, there is a $N_2 \in \bN$ such that if $n \geq N_2$, then $|a_n - A| < \frac{3}{4} \sqrt{|A|} \varepsilon$. Put $N := \max \{ N_1 , N_2 \}$. Then, using \eqref{eq:sqrt}, if $n \geq N$, then
	\begin{align*}
	\left| \sqrt{a_n} - \sqrt{A} \right| = \left| \frac{a_n - A}{\sqrt{a_n} + \sqrt{A}} \right| = \frac{|a_n - A|}{\sqrt{a_n} + \sqrt{A}}. 
	\end{align*}
Now, since $n \geq N \geq N_1$, we know that $\sqrt{a_n} \geq \sqrt{|A|}/2$ and so $\sqrt{a_n} + \sqrt{A} \geq \frac{3}{4} \sqrt{|A|}$. Using this, we can bound $|\sqrt{a_n} - \sqrt{A}|$ by
	\begin{align*}
	\frac{|a_n - A|}{\frac{3}{4} \sqrt{|A|}}
	\end{align*}
and since $n \geq N \geq N_2$, we also have
	\begin{align*}
	\frac{|a_n - A|}{\frac{3}{4} \sqrt{|A|}} < \frac{\frac{3}{4} |A| \varepsilon}{\frac{3}{4} |A|} = \varepsilon .
	\end{align*}
Thus, we have just shown that if $\varepsilon > 0$ is arbitrary, then there exists a $N \in \bN$ such that $|\sqrt{a_n} - \sqrt{A}| < \varepsilon$. We then conclude that $\sqrt{a_n} \ra \sqrt{A}$.
\end{sol}

\begin{exer}
(5 pts)
For each sequence $(a_n)_{n = 1}^\infty$, define the sequence $(\sigma_n)_{n = 1}^\infty$ by
	\begin{align*}
	\sigma_n := \frac{a_1 + a_2 + \cdots + a_n}{n} \quad (n \geq 1 ) .
	\end{align*}
Prove that if $a_n \ra A$, then $\sigma_n \ra A$. Find an example of a divergent sequence $(a_n)$ such that $(\sigma_n)_{n = 1}^\infty$ converges.
\end{exer}
\begin{sol}
Let $a_n \ra A$. We want to prove that $\sigma_n \ra A$. This means that for any $\varepsilon > 0$, there is a $N \in \bN$ such that if $n \geq N$, then $|\sigma_n - A| < \varepsilon$. Let $\varepsilon > 0$ be arbitrary. From the definition of the convergence of $(a_n)$, there exists a $N_1 \in \bN$ such that if $n \geq N_1$, then $|a_n - A| < \varepsilon/2$. So, we get
	\begin{align*}
	|\sigma_n - A| = \left| \frac{\sum_{k = 1}^n a_k}{n} - A \right| = \left| \frac{\sum_{k = 1}^n a_k - nA}{n} \right|
	\end{align*}
Separate the sum from $k= 1$ to $k= N_1 - 1$ and from $k= N_1$ to $k = n$ and use triangle inequality twice to obtain
	\begin{align*}
	\left| \frac{\sum_{k = 1}^n a_k - nA}{n} \right| \leq \frac{\sum_{k = 1}^{N_1 - 1} |a_k - A|}{n} + \frac{\sum_{k = N_1}^n |a_k - A|}{n} .
	\end{align*}

We know that a convergence sequence is bounded. So, there is a $M > 0$ such that $|a_k| \leq M$ for any $k \geq 1$. Then 
	\begin{align*}
	|a_k - A| \leq |a_k| + |A| \leq M + |A| \quad \forall k \geq 1 .
	\end{align*}
Also, when $k \geq N_1$, then $|a_k - A| < \varepsilon/2$. Also, by the AP, there is a natural number $N_2 \in \bN$ such that $N_2 (\varepsilon /2) > (N_1 - 1) (M + |A|)$. 

Take $N := \max \{ N_1 , N_2 \}$ and $n \geq N$. Putting everything together, we obtain
	\begin{align*}
	| \sigma_n - A | \leq \frac{(N_1 - 1) (M + |A|)}{n} + \sum_{k = N_1}^n \frac{\varepsilon/2}{n} < \frac{(N_1 - 1) (M + |A|)}{N_2} + (n - N_1)(\varepsilon/2) /n  < \varepsilon/2 + \varepsilon/2 .
	\end{align*}
Then, for any $\varepsilon$, we just proved that there is a $N \in \bN$ such that if $n \geq N$, then $|\sigma_n - A| < \varepsilon$. We conclude that $\sigma_n \ra A$.

Take $a_n = (-1)^n$. Then $(a_n)$ diverge, but $\sigma_n \ra 0$.
\end{sol}

\section{Homework problems}
%\begin{exer}
%Prove that
	%\begin{enumerate}[label=\textbf{\alph*)}]
	%\item $\bigcap_{n \geq 1} [-1/n , 1/n ] = \{ 0 \}$.
	%\item $\bigcup {n \geq 1} [-n , n ] = \bR$.
	%\end{enumerate}
%\end{exer}

\begin{exer}
(10 pts)
Use the definition of convergence to prove that each of the following sequences converges.
	\begin{enumerate}[label=\textbf{\alph*)}]
	\item $(a_n )_{n = 1}^\infty$ given by $a_n = 5 + 1/n$ for $n \geq 1$.
	\item $(a_n)_{n = 1}^\infty$ given by $a_n = \frac{3n}{2n + 1}$ for $n \geq 1$.
	\end{enumerate}
\end{exer}
\begin{sol}
\begin{enumerate}[label=\textbf{\alph*)}]
\item Take $A = 5$. Let $\varepsilon > 0$ be arbitrary. Then, for any $n \geq 1$, we have
	\begin{align*}
	|5 - 1/n - 5| = |-1/n| = 1/n .
	\end{align*}
By the AP ($x = \varepsilon$ and $y= 1$), there is a $N_0 \in \bN$ such that $N_0 \varepsilon > 1$ and so $1/N_0 < \varepsilon$. Take $N = N_0$, so, if $n \geq N_0$, we have
	$$
	|5 - 1/n - 5| = 1/n \leq 1/N_0 < \varepsilon .
	$$
Since $\varepsilon > 0$ was arbitrary, we just proved that for any $\varepsilon > 0$, there is a $N \in \bN$ such that $|a_n - 5| < \varepsilon$. Thus, $a_n \ra 5$.
\item Take $A = 3/2$. Let $\varepsilon > 0$. We have
	\begin{align*}
	\left| \frac{3n}{2n + 1} - \frac{3}{2} \right| = \left| \frac{6n - 6n - 3}{2 (2n + 1)} \right| = \frac{3}{2 (2n + 1)} .
	\end{align*}
By the AP ($x = \varepsilon$ and $y = 1/2$), there is a $N_0 \in \bN$ such that $(2N_0 + 1)\varepsilon > \frac{1}{2}$ and so $\frac{1}{2 (2N_0 + 1)} < \varepsilon$. Take $N = N_0$, so if $n \geq N_0$, then we have $2 (2n + 1) \geq 2(2N_0 + 1)$ and
	\begin{align*}
	\left| \frac{3n}{2n + 1} - \frac{3}{2} \right| = \frac{3}{2 (2n + 1)}  \leq \frac{1}{2 (2N_0 + 1)} < \varepsilon .
	\end{align*}
Since $\varepsilon > 0$ was arbitrary, we just proved that for any $\varepsilon > 0$, there is a $N \in \bN$ such that $|a_n - 3/2| < \varepsilon$. Thus $a_n \ra 3/2$.
\end{enumerate}
\end{sol}

\begin{exer}
(5 pts)
Prove that the sequence $(a_n)_{n = 1}^\infty = \Big( \frac{2n + 1}{n} \Big)_{n = 1}^\infty$ is a Cauchy sequence.
\end{exer}
\begin{sol}
Let $\varepsilon > 0$ be arbitrary. For $n , m \geq 1$, we have
	\begin{align*}
	\left| \frac{2n + 1}{n} - \frac{2m + 1}{m} \right| = \left| \frac{2nm + m - 2nm - n}{nm} \right| = \frac{|m - n|}{nm} .
	\end{align*}
By the triangle inequality, $|m - n|/mn \leq (m + n)/mn = \frac{1}{n} + \frac{1}{m}$ and so
	\begin{align*}
	\left| \frac{2n + 1}{n} - \frac{2m + 1}{m} \right| \leq \frac{1}{n} + \frac{1}{m} .
	\end{align*}
By the AP ($x = \varepsilon $ and $y = 2$), there is a $N_0 \in \bN$ such that $N_0 \varepsilon > 2$, so that $\frac{1}{N_0} < \varepsilon /2$. Take $N = N_0$ and let $n, m \geq N_0$. Then, we have
	\begin{align*}
	\left| \frac{2n + 1}{n} - \frac{2m + 1}{m} \right| \leq 1/n + 1/m < \varepsilon / 2 + \varepsilon / 2 = \varepsilon .
	\end{align*}
Since $\varepsilon > 0$ was arbitrary, we just proved that for any $\varepsilon > 0$, there exists a $N \in \bN$ such that if $n, m > N$, then $|a_n - a_m| < \varepsilon$. Thus the sequence is Cauchy.
\end{sol}

\begin{exer}
(10 pts)
Prove that each of the following sequence diverges.
	\begin{enumerate}[label=\textbf{\alph*)}]
	\item $(a_n)_{n = 1}^\infty = ( (-1)^n )_{n = 1}^\infty$.
	\item $(a_n)_{n = 1}^\infty = ( \sin (\frac{2n + 1}{2} \pi) )_{n = 1}^{\infty}$.
	\end{enumerate}
\end{exer}
\begin{sol}
\begin{enumerate}[label=\textbf{\alph*)}]
\item Suppose that the sequence converges to $A$. Let $\varepsilon = |A|$, if $A \neq 0$ and $\varepsilon = 1/2$, if $A = 0$, in the definition of convergence of a sequence. Then there exists a $N \in \bN$ such that if $n \geq N$, then $|a_n - A| < \varepsilon$. 

If $A \neq 0$ and if $n \geq N$, then, by the properties of the absolute value, we have
	\begin{align*}
	|a_n - A| < \varepsilon \iff -|A| < a_n - A < |A| \iff A - |A| < (-1)^n < A + |A| .
	\end{align*}
	\begin{itemize}
	\item If $A > 0$, then $A - |A| = 0$ and for any $n \geq N$, $0 < (-1)^n$ which is false if $n$ is odd.
	\item If $A < 0$, then $A + |A| = 0$ and for any $n \geq N$, $(-1)^n < 0$ which is false if $n$ is even.
	\end{itemize}
	
	If $A = 0$, then $|(-1)^n| < 1/2$ and so $1 < 1/2$ a contradiction.
	
Thus, the sequence $(a_n)_{n = 1}^\infty$ is not convergent.
\item We see that $\sin (4n + 1)\pi/2 = (-1)^n$. So, from a), it is not a convergent sequence.
\end{enumerate}
\end{sol}


\begin{exer}
(5 pts)
Give an examples of two sequences $(a_n)$ and $(b_n)$ such that $(a_n)$ and $(b_n)$ don't converge, but $(a_n + b_n)$ converge.
\end{exer}
\begin{sol}
Let $a_n = (-1)^n$ and $b_n = (-1)^{n + 1}$, then $a_n + b_n = 0$. The sequences $(a_n)_{n = 1}^\infty$ and $(b_n)_{n = 1}^\infty$ both diverge, but $(a_n + b_n)_{n = 1}^\infty$ converge.
\end{sol}

\begin{exer}
(10 pts)
With the limit operations and the writing problems, find the limit of the following sequence with general term
	\begin{enumerate}[label=\textbf{\alph*)}]
	\item $\frac{n^2 + 4n}{n^2 - 5}$.
	\item $\frac{n}{n^2 - 3}$.
	\item $\frac{\cos n}{n}$. [You can use what you know on the cosine function.]
	\item $\Big( \sqrt{4 - \frac{1}{n}} - 2 \Big) n$.
	\end{enumerate}
\end{exer}
\begin{sol}
\begin{enumerate}[label=\textbf{\alph*)}]
\item We can't use the limit rules directly. We rearrange the expression:
	\begin{align*}
	\frac{n^2 + 4n}{n^2 - 5} = \frac{1 + 4/n}{1 - 5/n^2} .
	\end{align*}
Now, $1/n \ra 0$, so $1 + 4/n \ra 1$. Also, $1/n^2 \ra 0$ according to the product rule and so $1 - 5/n^2 \ra 1$. Thus, by the quotient rule, we get
	\begin{align*}
	\lim_{n \ra \infty} \frac{n^2 + 4n}{n^2 - 5} = \frac{1}{1} = 1.
	\end{align*}
\item Again, we can't use the limit rules directly. We rearrange the expression:
	\begin{align*}
	\frac{n}{n^2 - 3} = \Big( \frac{1}{n}\Big)  \Big( \frac{1}{1 - 3/n^2} \Big) .
	\end{align*}
We know that $1/n \ra 0$ and $1/n^2 \ra 0$. So, by the limit rules, $1 - 3/n^2 \ra 1$. Thus, by the product rule, we get
	\begin{align*}
	\lim_{n \ra \infty} \frac{n}{n^2 - 3} = \Big( \lim_{n \ra \infty} \frac{1}{n} \Big) \Big( \lim_{n \ra \infty} \frac{1}{1 - 3/n^2} \Big) = 0 .
	\end{align*}
\item We know that $|\cos (x) | \leq 1$ for any $x \in \bR$. So, use a Theorem in the lecture notes with $(a_n) = (1/n)_{n = 1}^\infty$ and $(b_n) = (\cos n )_{n = 1}^\infty$, we have that $a_n b_n \ra 0$ since $1/n \ra 0$. In other words, $\cos (n) /n \ra 0$.
\item Here we can't apply the limit rules directly. We have to rearrange the expression. We have
	\begin{align*}
	\Big( \sqrt{4 - \frac{1}{n}} - 2 \Big) n = \frac{(4 - 1/n -4)n}{\sqrt{4 - 1/n} + 2} = \frac{-1}{\sqrt{4 - 1/n} + 2} .
	\end{align*}
Now, from the working problems, since $1/n \ra 0$ and so $4 - 1/n \ra 4$, we get that $\sqrt{4 - 1/n} \ra \sqrt{4} = 2$. Thus, from the quotient rule,
	\begin{align*}
	\lim_{n \ra \infty} \Big( \sqrt{4 - \frac{1}{n}} - 2 \Big) n = -\frac{1}{2 + 2} = -\frac{1}{4} .
	\end{align*}
\end{enumerate}
\end{sol}

\end{document}