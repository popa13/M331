\documentclass[12pt]{article}
\usepackage[utf8]{inputenc}

\usepackage{enumitem}
\usepackage[margin=2cm]{geometry}

\usepackage{amsmath, amsfonts, amssymb}
\usepackage{graphicx}
\usepackage{tikz}
\usepackage{pgfplots}
\usepackage{multicol}

\usepackage{comment}
\usepackage{url}

\usepackage{titlesec}
\titleformat{\section}[frame]
{\normalfont\scshape}
{\thesection}{8pt}{\centering}

\usepackage{array}

\pgfplotsset{compat=1.16}

%\usepackage[margin=1cm]{cloze}

\usepackage[thmmarks]{ntheorem}

% MATH commands
\newcommand{\bC}{\mathbb{C}}
\newcommand{\bR}{\mathbb{R}}
\newcommand{\bN}{\mathbb{N}}
\newcommand{\bZ}{\mathbb{Z}}
\newcommand{\bT}{\mathbb{T}}
\newcommand{\bD}{\mathbb{D}}
\newcommand{\bQ}{\mathbb{Q}}

\newcommand{\cL}{\mathcal{L}}
\newcommand{\cM}{\mathcal{M}}
\newcommand{\cP}{\mathcal{P}}
\newcommand{\cH}{\mathcal{H}}
\newcommand{\cB}{\mathcal{B}}
\newcommand{\cK}{\mathcal{K}}
\newcommand{\cJ}{\mathcal{J}}
\newcommand{\cU}{\mathcal{U}}
\newcommand{\cO}{\mathcal{O}}
\newcommand{\cA}{\mathcal{A}}
\newcommand{\cC}{\mathcal{C}}

\newcommand{\fK}{\mathfrak{K}}
\newcommand{\fM}{\mathfrak{M}}

\newcommand{\ga}{\left\langle}
\newcommand{\da}{\right\rangle}
\newcommand{\oa}{\left\lbrace}
\newcommand{\fa}{\right\rbrace}
\newcommand{\oc}{\left[}
\newcommand{\fc}{\right]}
\newcommand{\op}{\left(}
\newcommand{\fp}{\right)}

\newcommand{\ra}{\rightarrow}
\newcommand{\Ra}{\Rightarrow}

\renewcommand{\Re}{\mathrm{Re}\,}
\renewcommand{\Im}{\mathrm{Im}\,}
\newcommand{\Arg}{\mathrm{Arg}\,}
\newcommand{\Arctan}{\mathrm{Arctan}\,}
\newcommand{\sech}{\mathrm{sech}\,}
\newcommand{\csch}{\mathrm{csch}\,}
\newcommand{\Log}{\mathrm{Log}\,}
\newcommand{\cis}{\mathrm{cis}\,}

\newcommand{\ran}{\mathrm{ran}\,}
\newcommand{\bi}{\mathbf{i}}
\newcommand{\Sp}{\mathrm{span}\,}
\newcommand{\Inv}{\mathrm{Inv}\,}
\newcommand\smallO{
  \mathchoice
    {{\scriptstyle\mathcal{O}}}% \displaystyle
    {{\scriptstyle\mathcal{O}}}% \textstyle
    {{\scriptscriptstyle\mathcal{O}}}% \scriptstyle
    {\scalebox{.7}{$\scriptscriptstyle\mathcal{O}$}}%\scriptscriptstyle
  }
\newcommand{\HOL}{\mathrm{Hol}}
\newcommand{\cl}{\mathrm{clos}}
\newcommand{\ve}{\varepsilon}

\tikzstyle{myboxT} = [draw=black, fill=black!0,line width = 1pt,
    rectangle, rounded corners = 0pt, inner sep=8pt, inner ysep=8pt]
    
\newcommand{\MyC}[1]{\begin{tikzpicture}
\node (boxIntro) at (0,0) {};
\node [myboxT](Intro) at (boxIntro){%
	\begin{minipage}{0.9\textwidth}
	#1
	\end{minipage}};
\end{tikzpicture}}

%%%%  Environnement exer et solutionnaire
{\theorembodyfont{}
\theoremstyle{plain}
\theoremseparator{\textbf{.}}
\theoremsymbol{}
\newtheorem{exer}{\textbf{Exercise}}}

{\theorembodyfont{\color{black}}
\theoremstyle{plain}
\theoremseparator{\textbf{:}}
\theoremsymbol{$\square$}
\newtheorem*{sol}{\textbf{Solution}}}


\renewcommand*{\theexer}{\arabic{exer}}
\renewcommand*{\thesol}{\arabic{sol}}


\newcommand{\headHW}[4]{%
	\noindent \hrulefill \\
	MATH-#1 #2 \\
	#3 #4
	
}

\begin{document}
	\noindent \hrulefill \\
	MATH-331 Introduction to Real Analysis \hfill YOUR FULL NAME\\
	Homework 04 \hfill Fall 2021\\\vspace*{-0.7cm}
	
	\noindent\hrulefill
	
	\noindent Due date: October 25${}^{\text{th}}$ 1:20pm \hfill Total: \hspace{0.3cm}/70.
	
\vspace*{0.5cm}

	\bgroup \renewcommand{\arraystretch}{1.5}
\begin{table}[h]
\centering
\begin{tabular}{|m{1.5cm}|>{\centering\arraybackslash}p{0.75cm}|>{\centering\arraybackslash}p{0.75cm}|>{\centering\arraybackslash}p{0.75cm}|>{\centering\arraybackslash}p{0.75cm}|>{\centering\arraybackslash}p{0.75cm}|>{\centering\arraybackslash}p{0.75cm}|>{\centering\arraybackslash}p{0.75cm}|>{\centering\arraybackslash}p{0.75cm}|>{\centering\arraybackslash}p{0.75cm}|>{\centering\arraybackslash}p{0.75cm}|}
\hline
Exercise & 1 (5) & 2 (5) & 3 (5) & 4 (5) & 5 (10) & 6 (10) & 7 (5) & 8 (5) & 9 (5) & 10 (10) \\
\hline
Score & & & & & & & & & &  \\\hline
\end{tabular}
\caption{Scores for each exercises}
\end{table}
\egroup
	
\vspace*{0.5cm}

{\bf Instructions:} You must answer all the questions below and send your solution by email (to \url{parisepo@hawaii.edu}). If you decide to not use {\LaTeX} to hand out your solutions, please be sure that after you scan your copy, it is clear and readable. Make sure that you attached a copy of the homework assignment to your homework. 

\noindent If you choose to use {\LaTeX}, you can use the template available on the course website.

\noindent No late homework will be accepted. No format other than PDF will be accepted. Name your file as indicated in the syllabus.

\section{Writing problems}
For each of the following problems, you will be asked to write a clear and detailed proof. You will have the chance to rewrite your solution in your semester project after receiving feedback from me.

%% ---------------------------------------------
\begin{exer}
(5 pts)
Prove that, if $0 < x < \pi/2$, then $0 \leq \sin x \leq x$ with a geometric argument. [Hint: View $\sin x$ as a point on the unit circle in the first quadrant.]
\end{exer}
\begin{sol}

\end{sol}

%-----------------------------------------------
\begin{exer}
(5 pts)
Let $f : A \ra \bR$ and $g : B \ra A$ be two functions where $A, B \subset \bR$. Let $a$ be an accumulation point of $A$ and $b$ be an accumulation point of $B$. Suppose that
	\begin{itemize}
	\item $\lim_{t \ra b} g(t) = a$.
	\item there is a $\eta > 0$ such that for any $t \in B \cap (b - \eta ,  b + \eta )$, $g(t) \neq a$.
	\item $f$ has a limit at $a$.
	\end{itemize}
Prove that $f \circ g$ has a limit at $b$ and $\lim_{x \ra a} f(x) = \lim_{t \ra b} f(g(t))$. [This is the change of variable rule for limits.]
\end{exer}
\begin{sol}

\end{sol}

%-----------------------------------------------
\begin{exer}
(5 pts)
Let $f : [a, b] \ra \bR$ be continuous on $[a, b]$ and suppose that $f(x) = 0$ for each rational number $x$ in $[a, b]$. Prove that $f(x) = 0$ for all $x \in [a, b]$.
\end{exer}
\begin{sol}

\end{sol}

%-----------------------------------------------
\begin{exer}
(5 pts)
Let $f: [a, b] \ra \bR$ be continuous on $[a, b]$ and suppose that $f(c) > 0$ for some $c \in [a, b]$. Prove that there exist a number $\eta$ and an interval $[u, v] \subset [a, b]$ such that $f(x) \geq \eta$ for all $x \in [u, v]$.
\end{exer}
\begin{sol}

\end{sol}

%-----------------------------------------------
\begin{exer}
(10 pts)
Let $f : \bR \ra \bR$ be a function that satisfies $f(x + y) = f(x) + f(y)$ for any real number $x$ and $y$.
	\begin{enumerate}[label=\textbf{\alph*)}]
	\item Suppose that $f$ is continuous at some point $c$. Prove that $f$ is continuous on $\bR$.
	\item Suppose that $f$ is continuous on $\bR$ and that $f(1) = k$. Prove that $f(x) = kx$ for all $x \in \bR$. [Hint: start with $x$ integer, then $x$ rational, and finally use Exercise 3.]
	\end{enumerate}
\end{exer}
\begin{sol}
\begin{enumerate}[label=\textbf{\alph*)}]
\item 
\item 
\end{enumerate}
\end{sol}



\section{Homework problems}
Answer all the questions below. Make sure to show your work.

\begin{exer}
(10pts)
For each of the functions below, say if the limit exists or doesn't exist at the given point. Justify your answer (in other words, prove it!)
	\begin{enumerate}[label=\textbf{\alph*)}]
	\item $f(x) = \sin (1/x)$ and $x_0 = 0$.
	\item $f(x) = x \sin (1/x)$ adn $x_0 = 0$.
	\end{enumerate}
\end{exer}
\begin{sol}
\begin{enumerate}[label=\textbf{\alph*)}]
\item 
\item 
\end{enumerate}
\end{sol}

%------------------------------------------------
\begin{exer}
(5 pts)
Let $c \in (a, b)$ and let $f$ be a function defined on $(a, b)$ except at $c$. Suppose that $f(x) > 0$ for any $x \in (a, b) \backslash \{ c \}$, that $\lim_{x \ra c} f(x)$ exists, and that
	\begin{align*}
	\lim_{x \ra c} f(x) = \lim_{x \ra c} \oc (f(x))^2 - f(x) - 3 \fc .
	\end{align*} 
Find the value of $\lim_{x \ra c} f(x)$. Explain each step carefully.
\end{exer}
\begin{sol}

\end{sol}

%------------------------------------------------
\begin{exer}
(5 pts)
Prove that the function $f : \bR \ra \bR$ defined by
	\begin{align*}
	f(x) := \begin{cases}
	x & \text{, } x \in \bQ \\
	-x &  \text{, } x \not\in \bQ .
	\end{cases}
	\end{align*}
is discontinuous at any point of $\bR \backslash \{ 0 \}$ and continuous at $0$.
\end{exer}
\begin{sol}

\end{sol}

%------------------------------------------------
\begin{exer}
(5 pts)
Let $p(x) = x^2 + 2$. Find an interval where $p$ is strictly decreasing and find a formula for its inverse.
\end{exer}
\begin{sol}

\end{sol}

%-------------------------------------------------
\begin{exer}
(10 pts)
Let $p(x) = ax^3 + bx^2 + cx + d$ be a polynomial of degree $3$ and $a > 0$. Prove that $p$ has at least one real root by following these steps:
	\begin{enumerate}[label=\textbf{\alph*)}]
	\item Prove that $\lim_{x \ra \infty} p(x) = \infty$.
	\item Prove that $\lim_{x \ra -\infty} p (x) = -\infty$.
	\item Conclude.
	\end{enumerate}
[Hint for a): write your polynomial $p(x) = ax^3 + bx^2 + cx + d$ as $x^3 (a + b/x + c/x^2 + d/x^3)$ and use the fact that $\lim_{x \ra \infty} 1/x^n = 0$ for every $n \geq 1$.]
\end{exer}
\begin{sol}
\begin{enumerate}[label=\textbf{\alph*)}]
\item 
\item 
\item
\end{enumerate}
\end{sol}

\end{document}