\documentclass[12pt]{article}
\usepackage[utf8]{inputenc}

\usepackage{enumitem}
\usepackage[margin=2cm]{geometry}

\usepackage{amsmath, amsfonts, amssymb}
\usepackage{graphicx}
\usepackage{tikz}
\usepackage{pgfplots}
\usepackage{multicol}

\usepackage{comment}
\usepackage{url}
\usepackage{calc}

\usepackage{titlesec}
\titleformat{\section}[frame]
{\normalfont\scshape}
{\thesection}{8pt}{\centering}

\usepackage{array}

\pgfplotsset{compat=1.16}

%\usepackage[margin=1cm]{cloze}

\usepackage[thmmarks]{ntheorem}

% MATH commands
\newcommand{\bC}{\mathbb{C}}
\newcommand{\bR}{\mathbb{R}}
\newcommand{\bN}{\mathbb{N}}
\newcommand{\bQ}{\mathbb{Q}}
\newcommand{\bZ}{\mathbb{Z}}
\newcommand{\bT}{\mathbb{T}}
\newcommand{\bD}{\mathbb{D}}

\newcommand{\cL}{\mathcal{L}}
\newcommand{\cM}{\mathcal{M}}
\newcommand{\cP}{\mathcal{P}}
\newcommand{\cH}{\mathcal{H}}
\newcommand{\cB}{\mathcal{B}}
\newcommand{\cK}{\mathcal{K}}
\newcommand{\cJ}{\mathcal{J}}
\newcommand{\cU}{\mathcal{U}}
\newcommand{\cO}{\mathcal{O}}
\newcommand{\cA}{\mathcal{A}}
\newcommand{\cC}{\mathcal{C}}
\newcommand{\cT}{\mathcal{T}}

\newcommand{\fK}{\mathfrak{K}}
\newcommand{\fM}{\mathfrak{M}}

\newcommand{\ga}{\left\langle}
\newcommand{\da}{\right\rangle}
\newcommand{\oa}{\left\lbrace}
\newcommand{\fa}{\right\rbrace}
\newcommand{\oc}{\left[}
\newcommand{\fc}{\right]}
\newcommand{\op}{\left(}
\newcommand{\fp}{\right)}

\newcommand{\ra}{\rightarrow}
\newcommand{\Ra}{\Rightarrow}

\renewcommand{\Re}{\mathrm{Re}\,}
\renewcommand{\Im}{\mathrm{Im}\,}
\newcommand{\Arg}{\mathrm{Arg}\,}
\newcommand{\Arctan}{\mathrm{Arctan}\,}
\newcommand{\sech}{\mathrm{sech}\,}
\newcommand{\csch}{\mathrm{csch}\,}
\newcommand{\Log}{\mathrm{Log}\,}
\newcommand{\cis}{\mathrm{cis}\,}

\newcommand{\ran}{\mathrm{ran}\,}
\newcommand{\bi}{\mathbf{i}}
\newcommand{\Sp}{\mathrm{span}\,}
\newcommand{\Inv}{\mathrm{Inv}\,}
\newcommand\smallO{
  \mathchoice
    {{\scriptstyle\mathcal{O}}}% \displaystyle
    {{\scriptstyle\mathcal{O}}}% \textstyle
    {{\scriptscriptstyle\mathcal{O}}}% \scriptstyle
    {\scalebox{.7}{$\scriptscriptstyle\mathcal{O}$}}%\scriptscriptstyle
  }
\newcommand{\HOL}{\mathrm{Hol}}
\newcommand{\cl}{\mathrm{clos}}
\newcommand{\ve}{\varepsilon}

\tikzstyle{myboxT} = [draw=black, fill=black!0,line width = 1pt,
    rectangle, rounded corners = 0pt, inner sep=8pt, inner ysep=8pt]
    
\newcommand{\MyC}[1]{\begin{tikzpicture}
\node (boxIntro) at (0,0) {};
\node [myboxT](Intro) at (boxIntro){%
	\begin{minipage}{0.9\textwidth}
	#1
	\end{minipage}};
\end{tikzpicture}}

%%%%  Environnement exer et solutionnaire
{\theorembodyfont{}
\theoremstyle{plain}
\theoremseparator{\textbf{.}}
\theoremsymbol{}
\newtheorem{exer}{\textbf{Exercise}}}

{\theorembodyfont{\color{blue}}
\theoremstyle{plain}
\theoremseparator{\textbf{:}}
\theoremsymbol{$\square$}
\newtheorem*{sol}{\textbf{Solution}}}

{\theorembodyfont{\color{blue}}
\theoremstyle{plain}
\theoremseparator{\textbf{:}}
\theoremsymbol{$\qed$}
\newtheorem*{solWP}{\textbf{Solution}}}

{\theorembodyfont{\color{blue}}
\theoremstyle{plain}
\theoremseparator{\textbf{:}}
\theoremsymbol{$\qed$}
\newtheorem*{hint}{\textbf{Hints}}}

\renewcommand*{\theexer}{\arabic{exer}}
\renewcommand*{\thesol}{\arabic{sol}}
\renewcommand*{\thesolWP}{\arabic{solWP}}
\renewcommand*{\thehint}{\arabic{hint}}

%%% Ignorer les solutions
\excludecomment{sol}
\excludecomment{solWP}
\excludecomment{hint}

\newcommand{\headHW}[4]{%
	\noindent \hrulefill \\
	MATH-#1 #2 \\
	#3 #4
	
}

\makeatletter
\DeclareFontFamily{U}{tipa}{}
\DeclareFontShape{U}{tipa}{m}{n}{<->tipa10}{}
\newcommand{\arc@char}{{\usefont{U}{tipa}{m}{n}\symbol{62}}}%

\newcommand{\arc}[1]{\mathpalette\arc@arc{#1}}

\newcommand{\arc@arc}[2]{%
  \sbox0{$\m@th#1#2$}%
  \vbox{
    \hbox{\resizebox{\wd0}{\height}{\arc@char}}
    \nointerlineskip
    \box0
  }%
}
\makeatother

\newcount\QO
\newcounter{QT}
\newcounter{QTh}
\newcounter{QF}
\newcounter{QFi}
\newcounter{QS}
\newcounter{QSe}
\newcounter{QE}
\newcounter{QN}
\newcounter{QTe}

\begin{document}
	\noindent \hrulefill \\
	MATH-331 Introduction to Real Analysis \hfill Created by P.-O. Paris{\'e}\\
	Homework 07 \hfill Fall 2021\\\vspace*{-0.7cm}
	
	\noindent\hrulefill
	
	\noindent Due date: December, 6${}^{\text{th}}$ 1:20pm \hfill Total: \hspace{0.3cm}/65.
	
\vspace*{0.5cm}

	\bgroup \renewcommand{\arraystretch}{1.5}
\begin{table}[h]
\centering
\begin{tabular}{|m{1.5cm}|>{\centering\arraybackslash}p{0.75cm}|>{\centering\arraybackslash}p{0.75cm}|>{\centering\arraybackslash}p{0.75cm}|>{\centering\arraybackslash}p{0.75cm}|>{\centering\arraybackslash}p{0.75cm}|>{\centering\arraybackslash}p{0.75cm}|>{\centering\arraybackslash}p{0.75cm}|>{\centering\arraybackslash}p{0.75cm}|>{\centering\arraybackslash}p{0.75cm}|>{\centering\arraybackslash}p{0.75cm}|}
\hline
Exercise & 1 (10) & 2 (5) & 3 (10) & 4 (5) & 5 (5) & 6 (10) & 7 (5) & 8 (5) & 9 (5) & 10 (5) \\
\hline
Score & & & & & & & & & &  \\\hline
\end{tabular}
\caption{Scores for each exercises}
\end{table}
\egroup
	
\vspace*{0.5cm}

{\bf Instructions:} You must answer all the questions below and send your solution by email (to \url{parisepo@hawaii.edu}). If you decide to not use {\LaTeX} to hand out your solutions, please be sure that after you scan your copy, it is clear and readable. Make sure that you attached a copy of the homework assignment to your homework. 

\noindent If you choose to use {\LaTeX}, you can use the template available on the course website.

\noindent No late homework will be accepted. No format other than PDF will be accepted. Name your file as indicated in the syllabus.

\section{Writing problems}
For each of the following problems, you will be asked to write a clear and detailed proof. All the exercises below can be solve without using the definition with partitions. Try to go back to homework 6 and use some of the exercises there to solve the following problems. 

You will have the chance to rewrite your solution in your semester project after receiving feedback from me.

%% ---------------------------------------------
\begin{exer}
(10 pts)
Prove that a step function is Riemann integrable on $[a, b]$. Follow the steps below.
\begin{enumerate}[label=\textbf{\alph*)}]
\item Let $I$ be a subinterval of $[a, b]$ and put $\phi = c \chi_I$. Prove that $\phi$ is Riemann integrable and that $\int_a^b \phi = c \ell (I)$. [There are three cases to consider: $I = [u, v]$, $I = (u, v]$, and $I = \{ u \} = [u, u]$.]
\item Prove by induction that if $f_1$, $f_2$, $\ldots$, $f_n$ are Riemann integrable functions on $[a, b]$, then $f_1 + f_2 + \cdots + f_n$ is Riemann integrable and
	\begin{align*}
	\int_a^b (f_1 + f_2 + \cdots + f_n) = \int_a^b f_1 + \int_a^b f_2 + \cdots + \int_a^b f_n .
	\end{align*}
\item Write $\phi = \sum_{k = 1}^n c_k \chi_{I_k}$. Use the second part of this exercise to show that $\phi$ is Riemann integrable.
\end{enumerate}
\end{exer}
\begin{sol}
\begin{enumerate}
\item Suppose that $I = [u, v]$. Since $\phi$ is zero on $[a, u)$, then it is integrable on $[a, c)$ for each $c \in (a, u)$. Also since $\phi$ is bounded, from exercise 5 in homework 5, $\phi$ is integrable on $[a, u]$. Similarly, $\phi$ is integrable on $[v, b]$. Now $\phi$ is just the constant function $c$ on $[u, v]$ which is integrable. Thus from the properties of the Riemann integral, $\phi$ is integrable on $[a, b]$.
\item From the properties of the Riemann integral, we know that $\int_a^b (f_1 + f_2) = \int_a^b f_1 + \int_a^b f_2$ for two Riemann integrable functions $f_1$ and $f_2$. Suppose that the property is true for $n$ Riemann integrable functions $f_1$, $f_2$, $\ldots$, $f_n$. Then $f_1 + f_2 + f_3 + \ldots + f_n$ and $f_n$ are Riemann integrable, so $f_1 + f_2 + \ldots + f_n + f_{n+1}$ is Riemann integrable and
	\begin{align*}
	\int_a^b (f_1 + f_2 + \cdots + f_n + f_{n+1}) &= \int_a^b (f_1 + f_2 + \cdots + f_n) + \int f_{n+1} \\
	&= \int_a^b f_1 + \int_a^b f_2 + \cdots + \int f_{n + 1}
	\end{align*}
where in the last equality, we used the induction hypothesis.
\item Let $f_k := c_k \chi_{I_k}$. From a), each $f_k$ are Riemann integrable on $[a, b]$. From b), their sum is also integrable on $[a, b]$. But their sum is simply $\phi$. Thus $\phi$ is Riemann integrable on $[a, b]$.
\end{enumerate}
\end{sol}

%-----------------------------------------------
\begin{exer}
(5 pts)
Suppose that $f$ is Riemann integrable on $[a, b]$ and that $f$ is nonnegative (means that $f(x) \geq 0$ for $x \in [a, b]$). Let $u, v \in \bR$. Show that if $a \leq u < v \leq b$, then
	\begin{align*}
	\int_u^v f \leq \int_a^b f .
	\end{align*}
[Hint: Use the following property of the Riemann Integral multiple times: $\int_a^b f = \int_a^c f + \int_c^b f$.]
\end{exer}
\begin{sol}
Let $a \leq u \leq v \leq b$. Since $f$ is Riemann integrable on $[a, b]$, it is Riemann integrable on each subinterval of $[a, b]$. In particular, $f$ is Riemann integrable on $[a, u]$, $[u, v]$, and $[v, b]$. This implies that
	\begin{align*}
	\int_a^b f = \int_a^u f + \int_u^v f + \int_v^b f .
	\end{align*}
Since $f(x) \geq 0$, then $\int_a^u f \geq 0$ and $\int_v^b f \geq 0$. Thus, 
	\begin{align*}
	\int_a^b f \geq \int_u^v f 
	\end{align*}
which is exactly what we wanted!
\end{sol}

%-----------------------------------------------
\begin{exer}
(10 pts)
Use the Fundamental Theorem of Calculus to solve the following problems:
	\begin{enumerate}[label=\textbf{\alph*)}]
	\item Suppose that $f$ is continuous on $[a, b]$ and that $f$ is nonnegative on $[a, b]$. Show that if $\int_a^b f = 0$, then $f(x) = 0$ for any $x \in [a, b]$.
	\item Suppose that $f$ and $g$ are continuous on $[a, b]$ such that $\int_a^b f = \int_a^b g$. Show that there exists a point $c \in (a, b)$ such that $f(c) = g(c)$.
	\end{enumerate}
\end{exer}
\begin{sol}
\begin{enumerate}
\item Let $F(x) = \int_a^x f$. Since $f$ is nonnegative on $[a, b]$, this means that for any $x \in [a, b]$, we have
	\begin{align*}
	F(x) = \int_a^x f \geq 0 .
	\end{align*}
This also means that if $x \leq y$, from the previous exercise, we have $F(x) \leq F(y)$. So $F$ is increasing. However, $F(b) = 0$. Thus, we must have that $F(x) = 0$ for any $x \in [a, b]$. From the Fundamental Theorem of Calculus, we know that $F'(x) = f(x)$ for any $x \in (a, b)$ which means that $f(x) = 0$ for any $x \in (a, b)$. By continuity, $f(x) = 0$ for any $x \in [a, b]$.
\item Let $F := \int_a^x (f - g)$. Since $f$ and $g$ are Riemann integrable on $[a, b]$, $F$ is well-defined. By the Fundamental Theorem of Calculus, since $f - g$ is continuous, this implies that $F$ is differentiable and continuous on $(a, b)$ and $[a, b]$ respectively. We see that $F(a) = 0$ and $F(b) = 0$. By Rolle's Theorem, this implies that there exists a $c \in (a, b)$ such that $F'(c) = 0$. But $F'(x) = f(x) - g(x)$ for any $x \in (a, b)$, so $f(c) = g(c)$.
\end{enumerate}
\end{sol}

%-----------------------------------------------
\begin{exer}
(5 pts)
Let $f$ be a continuous function on $[a, b]$. Prove that there exists a number $c \in [a, b]$ such that $f(c) (b - a) = \int_a^b f$.
\end{exer}
\begin{sol}
If $f$ is constant on $[a, b]$, just take any $c \in [a, b]$.

Suppose that $f$ is nonconstant. Then $f([a, b])$ is an interval. We will show that $\frac{1}{b - a} \int_a^b f \in f ([a, b])$. Let $m$ and $M$ be the minimum and maximum of $f$ on $[a, b]$. By the Extreme Value Theorem, we know that these exists and are attained by some points in $[a, b]$, say by $x_m, x_M$. Thus $m, M \in f ([a, b])$. We have
	\begin{align*}
	\int_a^b f \leq (b - a) M 
	\end{align*}
and so $\int_a^b f / (b - a) \leq M$. Also
	\begin{align*}
	\int_a^b f \geq (b - a) m
	\end{align*}
and so $\int_a^b f/ (b -a) \geq m$. Thus 
	\begin{align*}
	m \leq \frac{1}{b-a} \int_a^b f \leq M
	\end{align*}
and so by the Intermediate Value Theorem, there exists a $c \in [x_m, x_M] \subseteq [a, b]$ such that $f (c) = \frac{1}{b-a} \int_a^b f$. 
\end{sol}

%-----------------------------------------------
\begin{exer}
(5 pts)
Suppose that $f$ is Riemann integrable on $[a, b]$ and is strictly increasing there. Prove that there exists a point $c \in (a, b)$ such that 
	\begin{align*}
	\int_a^b f = f(a) (c - a) + f(b) (b - c).
	\end{align*}
[Hint: Define the function $g(x) = f(a) (x - a) + f(b) (b - x)$. Show that $\int_a^b f$ is between the numbers $f(a) (b - a)$ and $f(b) (b - a)$ and use the Intermediate Value Theorem.]
\end{exer}
\begin{sol}
Define $g(x) = f(a) (x - a) + f(b) (b - x)$. We see that $g(a) = f(b) (b -a)$ and $g(b) = f(a) (b - a)$. We will show that $\int_a^b f$ is between these two numbers.

Since $f$ is increasing, $f(x) \leq f(b)$ for any $x \in [a, b]$. So
	\begin{align*}
	\int_a^b f \leq f(b) (b - a) .
	\end{align*}
For the same reason, we have
	\begin{align*}
	\int_a^b f \geq f(a) (b - a) .
	\end{align*}
Thus $\int_a^b f$ is between $f(b) (b - a)$ and $f(a) (b - a)$.
\end{sol}

\section{Homework problems}
Answer all the questions below. Make sure to show your work.

\begin{exer}
(10pts)
\begin{enumerate}[label=\textbf{\alph*)}]
\item Show that the function $f : [0, 1] \ra \bR$ defined by	
	\begin{align*}
	f(x) = \begin{cases}
	1 &\text{, } x \in \bQ \\
	0 &\text{, } x \not\in \bQ
	\end{cases}
	\end{align*}
is not Riemann integrable on $[0, 1]$. [Hint: Use exercise 4 from Homework 6.]
\item Define the two functions $g : [0, 1] \ra \bR$ and $h : [0, 1] \ra \bR$ by $g = \chi_{(0, 1]}$ and
	\begin{align*}
	h (x) = \begin{cases}
	0 & \text{, } x \not\in \bQ \\
	\frac{1}{q} & \text{, } x = p/q \in \bQ .
	\end{cases}
	\end{align*}
Use the first part to show that $g \circ h$ is not Riemann integrable on $[0, 1]$. What can you say about the composition of two Riemann integrable functions in light of this last examples?
\end{enumerate}
\end{exer}
\begin{sol}
\begin{enumerate}
\item We will argue by contradiction. Suppose that $f$ is Riemann integrable. So, from exercise 4 in Homework 6, for any sequence of tagged partitions of $[0, 1]$ $(\cP_n )_{n = 1}^\infty$ with $\lim_{n \ra \infty} \Vert \cP_n \Vert = 0$, we have that $\lim_{n \ra \infty} S (f, \cP_n ) = L$ where $L$ is unique. In other words, the limit $(S (f, \cP_n ))_{n = 1}^\infty$ exists. 

Let $n \geq 1$ be an integer. Define each $\cP_n$ in the following way. Let $x_i := \frac{i}{n}$ where $i = 0, 1, 2, \ldots , 1$. 
If $n$ is even, then let $c_i \in [x_{i-1}, x_i] \cap \bQ$ for $i = 1, 2, \ldots , n$. If $n$ is odd, then let $c_i \in [x_{i-1}, x_i] \cap \bQ^c$ for $i = 1, 2, \ldots , n$. 

Then, if $n$ is even, we have $S(f, \cP_n) = 1$. But it $n$ is odd, we have $S(f, \cP_n) = 0$. So the sequence $(S(f, \cP_n))_{n = 1}^\infty$ doesn't have a limit. This is a contradiction.
\item We see by direct verification that $g \circ h = f$ where $f$ is the function in a). Since $f$ is not Riemann integrable, this means that $g \circ h$ is not Riemann integrable. From the lecture notes, we know that $g$ and $h$ are Riemann integrable. This shows that even though two functions are Riemann integrable, this doesn't mean that the composition is Riemann integrable.
\end{enumerate}
\end{sol}

%------------------------------------------------
\begin{exer}
(5 pts)
Show that if $f$ is continuous on $[a, b]$, then $|f|$ is Riemann integrable on $[a, b]$ and
	\begin{align*}
	\Big| \int_a^b f \Big| \leq \int_a^b |f| .
	\end{align*}
[Hint: There is a clever way to show that $|f|$ is Riemann integrable without using the definition with the partitions.]
\end{exer}
\begin{sol}
We have $-|f| \leq f \leq |f|$. From the properties of the Riemann integral, this means that $-\int_a^b |f| \leq \int_a^b f \leq \int_a^b |f|$. So $|\int_a^b f | \leq \int_a^b |f|$.
\end{sol}

%------------------------------------------------
\begin{exer}
(5 pts)
Find $f'(x)$ if $f (x) = \displaystyle\int_{\sqrt{x}}^{\sqrt[3]{x}} \frac{1}{1 + t^3} \, dt$ where $x \in [0, 1]$.
\end{exer}
\begin{sol}
Let $g(t) = \frac{1}{1 + t^3}$ for $t \in [0, 1]$.

From the property of the Riemann integral and since $\sqrt{x} \leq \sqrt[3]{x}$ for $x \in (0, 1)$, we have
	\begin{align*}
	\int_0^1 g = \int_0^{\sqrt{x}} g + \int_{\sqrt{x}}^{\sqrt[3]{x}} g + \int_{\sqrt[3]{x}}^{1} g .
	\end{align*}
So, by isolating the second term in the right-hand side of the last equality, we get
	\begin{align*}
	f(x) = \int_0^1 g- \int_0^{\sqrt{x}} g - \int_{\sqrt[3]{x}}^1 g = \int_0^{\sqrt[3]{x}} g - \int_0^{\sqrt{x}} g .
	\end{align*}
Let $\alpha (x) = \sqrt{x}$ and $\beta (x) = \sqrt[3]{x}$. Define the function $G (x) = \int_0^x f$. Then since $t \mapsto 1/(1 + t^3)$ is continuous on $[0, 1]$, the function $G$ is differentiable on $(0, 1)$ with $G' = g$ by the Fundamental Theorem of Calculus. 

We see that $f(x) = G(\beta (x)) - G (\alpha (x))$. Taking the derivative, we get $f'(x) = g (\beta (x)) \beta'(x) - g (\alpha (x)) \alpha'(x)$.
\end{sol}

%-------------------------------------------------
\begin{exer}
(5 pts)
Find a function $f : [1, \infty ) \ra \bR$ such that $f(1) = 0$ and $f'(x) = 1 + \sin (x^2 )$ for all $x > 1$.
\end{exer}
\begin{sol}
Let $f(x) = \int_1^\infty (1 + \sin (x^2)) \, dx$. From the Fundamental Theorem of Calculus, that should do the job!
\end{sol}

%------------------------------------------------
\begin{exer}
(5 pts)
By thinking the following sum as a Riemann sum, evaluate
	\begin{align*}
	\lim_{n \ra \infty} \sum_{k = 1}^n \frac{n}{k^2 + n^2} .
	\end{align*}
\end{exer}
\begin{sol}
We see that
	\begin{align*}
	\sum_{k= 1}^n \frac{n}{k^2 + n^2} = \sum_{k= 1}^n \frac{n^2}{k^2 + n^2} \Big( \frac{1}{n} \Big) = \sum_{k = 1}^n \frac{1/n}{1 + ( \frac{k}{n})^2} .
	\end{align*}
Let $f(t) = \frac{1}{1 + t^2}$. Then
	\begin{align*}
	\sum_{k = 1}^n \frac{n}{k^2 + n^2} = S (f, \cP_n )
	\end{align*}
where $\cP_n := \{ (k/n , [ (k-1)/n , k/n]) \, : \, k = 1, 2, \ldots , n \}$ which is a tagged partition of $[0, 1]$. Since $f$ is Riemann integrable on $[0, 1]$, from exercise 4 in Homework 6, we see that
	\begin{align*}
	\lim_{n \ra \infty} S (f, \cP_n ) = \int_0^1 f .
	\end{align*}
Now, $\int_0^1 f = \frac{\pi}{4}$. Thus
	\begin{align*}
	\lim_{n \ra \infty} \sum_{k = 1}^n \frac{n}{k^2 + n^2} = \lim_{n \ra \infty} S (f , \cP_n ) = \pi/4 .
	\end{align*}
\end{sol}


\end{document}