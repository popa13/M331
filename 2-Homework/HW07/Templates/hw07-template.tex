\documentclass[12pt]{article}
\usepackage[utf8]{inputenc}

\usepackage{enumitem}
\usepackage[margin=2cm]{geometry}

\usepackage{amsmath, amsfonts, amssymb}
\usepackage{graphicx}
\usepackage{tikz}
\usepackage{pgfplots}
\usepackage{multicol}

\usepackage{comment}
\usepackage{url}

\usepackage{titlesec}
\titleformat{\section}[frame]
{\normalfont\scshape}
{\thesection}{8pt}{\centering}

\usepackage{array}

\pgfplotsset{compat=1.16}

%\usepackage[margin=1cm]{cloze}

\usepackage[thmmarks]{ntheorem}

% MATH commands
\newcommand{\bC}{\mathbb{C}}
\newcommand{\bR}{\mathbb{R}}
\newcommand{\bN}{\mathbb{N}}
\newcommand{\bZ}{\mathbb{Z}}
\newcommand{\bT}{\mathbb{T}}
\newcommand{\bD}{\mathbb{D}}
\newcommand{\bQ}{\mathbb{Q}}

\newcommand{\cL}{\mathcal{L}}
\newcommand{\cM}{\mathcal{M}}
\newcommand{\cP}{\mathcal{P}}
\newcommand{\cH}{\mathcal{H}}
\newcommand{\cB}{\mathcal{B}}
\newcommand{\cK}{\mathcal{K}}
\newcommand{\cJ}{\mathcal{J}}
\newcommand{\cU}{\mathcal{U}}
\newcommand{\cO}{\mathcal{O}}
\newcommand{\cA}{\mathcal{A}}
\newcommand{\cC}{\mathcal{C}}

\newcommand{\fK}{\mathfrak{K}}
\newcommand{\fM}{\mathfrak{M}}

\newcommand{\ga}{\left\langle}
\newcommand{\da}{\right\rangle}
\newcommand{\oa}{\left\lbrace}
\newcommand{\fa}{\right\rbrace}
\newcommand{\oc}{\left[}
\newcommand{\fc}{\right]}
\newcommand{\op}{\left(}
\newcommand{\fp}{\right)}

\newcommand{\ra}{\rightarrow}
\newcommand{\Ra}{\Rightarrow}

\renewcommand{\Re}{\mathrm{Re}\,}
\renewcommand{\Im}{\mathrm{Im}\,}
\newcommand{\Arg}{\mathrm{Arg}\,}
\newcommand{\Arctan}{\mathrm{Arctan}\,}
\newcommand{\sech}{\mathrm{sech}\,}
\newcommand{\csch}{\mathrm{csch}\,}
\newcommand{\Log}{\mathrm{Log}\,}
\newcommand{\cis}{\mathrm{cis}\,}

\newcommand{\ran}{\mathrm{ran}\,}
\newcommand{\bi}{\mathbf{i}}
\newcommand{\Sp}{\mathrm{span}\,}
\newcommand{\Inv}{\mathrm{Inv}\,}
\newcommand\smallO{
  \mathchoice
    {{\scriptstyle\mathcal{O}}}% \displaystyle
    {{\scriptstyle\mathcal{O}}}% \textstyle
    {{\scriptscriptstyle\mathcal{O}}}% \scriptstyle
    {\scalebox{.7}{$\scriptscriptstyle\mathcal{O}$}}%\scriptscriptstyle
  }
\newcommand{\HOL}{\mathrm{Hol}}
\newcommand{\cl}{\mathrm{clos}}
\newcommand{\ve}{\varepsilon}

\tikzstyle{myboxT} = [draw=black, fill=black!0,line width = 1pt,
    rectangle, rounded corners = 0pt, inner sep=8pt, inner ysep=8pt]
    
\newcommand{\MyC}[1]{\begin{tikzpicture}
\node (boxIntro) at (0,0) {};
\node [myboxT](Intro) at (boxIntro){%
	\begin{minipage}{0.9\textwidth}
	#1
	\end{minipage}};
\end{tikzpicture}}

%%%%  Environnement exer et solutionnaire
{\theorembodyfont{}
\theoremstyle{plain}
\theoremseparator{\textbf{.}}
\theoremsymbol{}
\newtheorem{exer}{\textbf{Exercise}}}

{\theorembodyfont{\color{black}}
\theoremstyle{plain}
\theoremseparator{\textbf{:}}
\theoremsymbol{$\square$}
\newtheorem*{sol}{\textbf{Solution}}}


\renewcommand*{\theexer}{\arabic{exer}}
\renewcommand*{\thesol}{\arabic{sol}}


\newcommand{\headHW}[4]{%
	\noindent \hrulefill \\
	MATH-#1 #2 \\
	#3 #4
	
}

\begin{document}

%%%%%%%%%%%%%%%%%%%%%%%%%%%%%%%%%%%%%%%5
%%%%%%%%%% HEADING HERE %%%%%%%%%%%%%%%%
%%%%%%%%%%%%%%%%%%%%%%%%%%%%%%%%%%%%%%%%%%
	\noindent \hrulefill \\
	MATH-331 Introduction to Real Analysis \hfill YOUR FULL NAME\\
	Homework 07 \hfill Fall 2021\\\vspace*{-0.7cm}
	
%%%%%%%%%%%%%%%%%%%%%%%%%%%%%%%%%%%%%%%%%55
%%%%%%%%%%%%%%%%%%%%%%%%%%%%%%%%%%%%%%%%
	
	\noindent\hrulefill
	
	\noindent Due date: December, 6${}^{\text{th}}$ 1:20pm \hfill Total: \hspace{0.3cm}/65.
	
\vspace*{0.5cm}

	\bgroup \renewcommand{\arraystretch}{1.5}
\begin{table}[h]
\centering
\begin{tabular}{|m{1.5cm}|>{\centering\arraybackslash}p{0.75cm}|>{\centering\arraybackslash}p{0.75cm}|>{\centering\arraybackslash}p{0.75cm}|>{\centering\arraybackslash}p{0.75cm}|>{\centering\arraybackslash}p{0.75cm}|>{\centering\arraybackslash}p{0.75cm}|>{\centering\arraybackslash}p{0.75cm}|>{\centering\arraybackslash}p{0.75cm}|>{\centering\arraybackslash}p{0.75cm}|>{\centering\arraybackslash}p{0.75cm}|}
\hline
Exercise & 1 (10) & 2 (5) & 3 (10) & 4 (5) & 5 (5) & 6 (10) & 7 (5) & 8 (5) & 9 (5) & 10 (5) \\
\hline
Score & & & & & & & & & &  \\\hline
\end{tabular}
\caption{Scores for each exercises}
\end{table}
\egroup
	
\vspace*{0.5cm}

{\bf Instructions:} You must answer all the questions below and send your solution by email (to \url{parisepo@hawaii.edu}). If you decide to not use {\LaTeX} to hand out your solutions, please be sure that after you scan your copy, it is clear and readable. Make sure that you attached a copy of the homework assignment to your homework. 

\noindent If you choose to use {\LaTeX}, you can use the template available on the course website.

\noindent No late homework will be accepted. No format other than PDF will be accepted. Name your file as indicated in the syllabus.


\section{Writing problems}
For each of the following problems, you will be asked to write a clear and detailed proof. All the exercises below can be solve without using the definition with partitions. Try to go back to homework 6 and use some of the exercises there to solve the following problems. 

You will have the chance to rewrite your solution in your semester project after receiving feedback from me.

%% ---------------------------------------------
\begin{exer}
(10 pts)
Prove that a step function is Riemann integrable on $[a, b]$. Follow the steps below.
\begin{enumerate}[label=\textbf{\alph*)}]
\item Let $I$ be a subinterval of $[a, b]$ and put $\phi = c \chi_I$. Prove that $\phi$ is Riemann integrable and that $\int_a^b \phi = c \ell (I)$. [There are three cases to consider: $I = [u, v]$, $I = (u, v]$, and $I = \{ u \} = [u, u]$.]
\item Prove by induction that if $f_1$, $f_2$, $\ldots$, $f_n$ are Riemann integrable functions on $[a, b]$, then $f_1 + f_2 + \cdots + f_n$ is Riemann integrable and
	\begin{align*}
	\int_a^b (f_1 + f_2 + \cdots + f_n) = \int_a^b f_1 + \int_a^b f_2 + \cdots + \int_a^b f_n .
	\end{align*}
\item Write $\phi = \sum_{k = 1}^n c_k \chi_{I_k}$. Use the second part of this exercise to show that $\phi$ is Riemann integrable.
\end{enumerate}
\end{exer}
\begin{sol}

\end{sol}

%-----------------------------------------------
\begin{exer}
(5 pts)
Suppose that $f$ is Riemann integrable on $[a, b]$ and that $f$ is nonnegative (means that $f(x) \geq 0$ for $x \in [a, b]$). Let $u, v \in \bR$. Show that if $a \leq u < v \leq b$, then
	\begin{align*}
	\int_u^v f \leq \int_a^b f .
	\end{align*}
[Hint: Use the following property of the Riemann Integral multiple times: $\int_a^b f = \int_a^c f + \int_c^b f$.]
\end{exer}
\begin{sol}

\end{sol}

%-----------------------------------------------
\begin{exer}
(10 pts)
Use the Fundamental Theorem of Calculus to solve the following problems:
	\begin{enumerate}[label=\textbf{\alph*)}]
	\item Suppose that $f$ is continuous on $[a, b]$ and that $f$ is nonnegative on $[a, b]$. Show that if $\int_a^b f = 0$, then $f(x) = 0$ for any $x \in [a, b]$.
	\item Suppose that $f$ and $g$ are continuous on $[a, b]$ such that $\int_a^b f = \int_a^b g$. Show that there exists a point $c \in (a, b)$ such that $f(c) = g(c)$.
	\end{enumerate}
\end{exer}
\begin{sol}

\end{sol}

%-----------------------------------------------
\begin{exer}
(5 pts)
Let $f$ be a continuous function on $[a, b]$. Prove that there exists a number $c \in [a, b]$ such that $f(c) (b - a) = \int_a^b f$.
\end{exer}
\begin{sol}

\end{sol}

%-----------------------------------------------
\begin{exer}
(5 pts)
Suppose that $f$ is Riemann integrable on $[a, b]$ and is strictly increasing there. Prove that there exists a point $c \in (a, b)$ such that 
	\begin{align*}
	\int_a^b f = f(a) (c - a) + f(b) (b - c).
	\end{align*}
[Hint: Define the function $g(x) = f(a) (x - a) + f(b) (b - x)$. Show that $\int_a^b f$ is between the numbers $f(a) (b - a)$ and $f(b) (b - a)$ and use the Intermediate Value Theorem.]
\end{exer}
\begin{sol}

\end{sol}

\section{Homework problems}
Answer all the questions below. Make sure to show your work.

\begin{exer}
(10pts)
\begin{enumerate}[label=\textbf{\alph*)}]
\item Show that the function $f : [0, 1] \ra \bR$ defined by	
	\begin{align*}
	f(x) = \begin{cases}
	1 &\text{, } x \in \bQ \\
	0 &\text{, } x \not\in \bQ
	\end{cases}
	\end{align*}
is not Riemann integrable on $[0, 1]$. [Hint: Use exercise 4 from Homework 6.]
\item Define the two functions $g : [0, 1] \ra \bR$ and $h : [0, 1] \ra \bR$ by $g = \chi_{(0, 1]}$ and
	\begin{align*}
	h (x) = \begin{cases}
	0 & \text{, } x \not\in \bQ \\
	\frac{1}{q} & \text{, } x = p/q \in \bQ .
	\end{cases}
	\end{align*}
Use the first part to show that $g \circ h$ is not Riemann integrable on $[0, 1]$. What can you say about the composition of two Riemann integrable functions in light of this last examples?
\end{enumerate}
\end{exer}
\begin{sol}

\end{sol}

%------------------------------------------------
\begin{exer}
(5 pts)
Show that if $f$ is continuous on $[a, b]$, then $|f|$ is Riemann integrable on $[a, b]$ and
	\begin{align*}
	\Big| \int_a^b f \Big| \leq \int_a^b |f| .
	\end{align*}
[Hint: There is a clever way to show that $|f|$ is Riemann integrable without using the definition with the partitions.]
\end{exer}
\begin{sol}

\end{sol}

%------------------------------------------------
\begin{exer}
(5 pts)
Find $f'(x)$ if $f (x) = \displaystyle\int_{\sqrt{x}}^{\sqrt[3]{x}} \frac{1}{1 + t^3} \, dt$ where $x \in [0, 1]$.
\end{exer}
\begin{sol}

\end{sol}

%-------------------------------------------------
\begin{exer}
(5 pts)
Find a function $f : [1, \infty ) \ra \bR$ such that $f(1) = 0$ and $f'(x) = 1 + \sin (x^2 )$ for all $x > 1$.
\end{exer}
\begin{sol}

\end{sol}

%------------------------------------------------
\begin{exer}
(5 pts)
By thinking the following sum as a Riemann sum, evaluate
	\begin{align*}
	\lim_{n \ra \infty} \sum_{k = 1}^n \frac{n}{k^2 + n^2} .
	\end{align*}
\end{exer}
\begin{sol}

\end{sol}


\end{document}