\documentclass[12pt]{article}
\usepackage[utf8]{inputenc}

\usepackage{enumitem}
\usepackage[margin=2cm]{geometry}

\usepackage{amsmath, amsfonts, amssymb}
\usepackage{graphicx}
\usepackage{tikz}
\usepackage{pgfplots}
\usepackage{multicol}

\usepackage{comment}
\usepackage{url}

\usepackage{titlesec}
\titleformat{\section}[frame]
{\normalfont\scshape}
{\thesection}{8pt}{\centering}

\usepackage{array}

\pgfplotsset{compat=1.16}

%\usepackage[margin=1cm]{cloze}

\usepackage[thmmarks]{ntheorem}

% MATH commands
\newcommand{\bC}{\mathbb{C}}
\newcommand{\bR}{\mathbb{R}}
\newcommand{\bN}{\mathbb{N}}
\newcommand{\bZ}{\mathbb{Z}}
\newcommand{\bT}{\mathbb{T}}
\newcommand{\bD}{\mathbb{D}}

\newcommand{\cL}{\mathcal{L}}
\newcommand{\cM}{\mathcal{M}}
\newcommand{\cP}{\mathcal{P}}
\newcommand{\cH}{\mathcal{H}}
\newcommand{\cB}{\mathcal{B}}
\newcommand{\cK}{\mathcal{K}}
\newcommand{\cJ}{\mathcal{J}}
\newcommand{\cU}{\mathcal{U}}
\newcommand{\cO}{\mathcal{O}}
\newcommand{\cA}{\mathcal{A}}
\newcommand{\cC}{\mathcal{C}}

\newcommand{\fK}{\mathfrak{K}}
\newcommand{\fM}{\mathfrak{M}}

\newcommand{\ga}{\left\langle}
\newcommand{\da}{\right\rangle}
\newcommand{\oa}{\left\lbrace}
\newcommand{\fa}{\right\rbrace}
\newcommand{\oc}{\left[}
\newcommand{\fc}{\right]}
\newcommand{\op}{\left(}
\newcommand{\fp}{\right)}

\newcommand{\ra}{\rightarrow}
\newcommand{\Ra}{\Rightarrow}

\renewcommand{\Re}{\mathrm{Re}\,}
\renewcommand{\Im}{\mathrm{Im}\,}
\newcommand{\Arg}{\mathrm{Arg}\,}
\newcommand{\Arctan}{\mathrm{Arctan}\,}
\newcommand{\sech}{\mathrm{sech}\,}
\newcommand{\csch}{\mathrm{csch}\,}
\newcommand{\Log}{\mathrm{Log}\,}
\newcommand{\cis}{\mathrm{cis}\,}

\newcommand{\ran}{\mathrm{ran}\,}
\newcommand{\bi}{\mathbf{i}}
\newcommand{\Sp}{\mathrm{span}\,}
\newcommand{\Inv}{\mathrm{Inv}\,}
\newcommand\smallO{
  \mathchoice
    {{\scriptstyle\mathcal{O}}}% \displaystyle
    {{\scriptstyle\mathcal{O}}}% \textstyle
    {{\scriptscriptstyle\mathcal{O}}}% \scriptstyle
    {\scalebox{.7}{$\scriptscriptstyle\mathcal{O}$}}%\scriptscriptstyle
  }
\newcommand{\HOL}{\mathrm{Hol}}
\newcommand{\cl}{\mathrm{clos}}
\newcommand{\ve}{\varepsilon}

\tikzstyle{myboxT} = [draw=black, fill=black!0,line width = 1pt,
    rectangle, rounded corners = 0pt, inner sep=8pt, inner ysep=8pt]
    
\newcommand{\MyC}[1]{\begin{tikzpicture}
\node (boxIntro) at (0,0) {};
\node [myboxT](Intro) at (boxIntro){%
	\begin{minipage}{0.9\textwidth}
	#1
	\end{minipage}};
\end{tikzpicture}}

%%%%  Environnement exer et solutionnaire
{\theorembodyfont{}
\theoremstyle{plain}
\theoremseparator{\textbf{.}}
\theoremsymbol{}
\newtheorem{exer}{\textbf{Exercise}}}

{\theorembodyfont{\color{blue}}
\theoremstyle{plain}
\theoremseparator{\textbf{:}}
\theoremsymbol{$\square$}
\newtheorem*{sol}{\textbf{Solution}}}

{\theorembodyfont{\color{blue}}
\theoremstyle{plain}
\theoremseparator{\textbf{:}}
\theoremsymbol{$\square$}
\newtheorem*{solWP}{\textbf{Solution}}}

{\theorembodyfont{\color{blue}}
\theoremstyle{plain}
\theoremseparator{\textbf{:}}
\theoremsymbol{$\square$}
\newtheorem*{hint}{\textbf{Hints}}}

\renewcommand*{\theexer}{\arabic{exer}}
\renewcommand*{\thesol}{\arabic{sol}}
\renewcommand*{\thesolWP}{\arabic{solWP}}
\renewcommand*{\thehint}{\arabic{hint}}

%%% Ignorer les solutions
\excludecomment{sol}
\excludecomment{solWP}
\excludecomment{hint}

\newcommand{\headHW}[4]{%
	\noindent \hrulefill \\
	MATH-#1 #2 \\
	#3 #4
	
}

\begin{document}
	\noindent \hrulefill \\
	MATH-331 Introduction to Real Analysis \hfill Created by P.-O. Paris{\'e}\\
	Homework 03 \hfill Fall 2021\\\vspace*{-0.7cm}
	
	\noindent\hrulefill
	
	\noindent Due date: October 11${}^{\text{th}}$ 1:20pm \hfill Total: \hspace{0.3cm}/70.
	
\vspace*{0.5cm}

	\bgroup \renewcommand{\arraystretch}{1.5}
\begin{table}[h]
\centering
\begin{tabular}{|m{1.5cm}|>{\centering\arraybackslash}p{0.75cm}|>{\centering\arraybackslash}p{0.75cm}|>{\centering\arraybackslash}p{0.75cm}|>{\centering\arraybackslash}p{0.75cm}|>{\centering\arraybackslash}p{0.75cm}|>{\centering\arraybackslash}p{0.75cm}|>{\centering\arraybackslash}p{0.75cm}|>{\centering\arraybackslash}p{0.75cm}|>{\centering\arraybackslash}p{0.75cm}|>{\centering\arraybackslash}p{0.75cm}|}
\hline
Exercise & 1 (5) & 2 (5) & 3 (5) & 4 (5) & 5 (10) & 6 (10) & 7 (5) & 8 (5) & 9 (5) & 10 (10) \\
\hline
Score & & & & & & & & & &  \\\hline
\end{tabular}
\caption{Scores for each exercises}
\end{table}
\egroup
	
\vspace*{0.5cm}

{\bf Instructions:} You must answer all the questions below and send your solution by email (to \url{parisepo@hawaii.edu}). If you decide to not use {\LaTeX} to hand out your solutions, please be sure that after you scan your copy, it is clear and readable. Make sure that you attached a copy of the homework assignment to your homework. 

\noindent If you choose to use {\LaTeX}, you can use the template available on the course website.

\noindent No late homework will be accepted. No format other than PDF will be accepted. Name your file as indicated in the syllabus.

\section{Writing problems}
For each of the following problems, you will be asked to write a clear and detailed proof. You will have the chance to rewrite your solution in your semester project after receiving feedback from me.

%\begin{exer}
%(10 pts)
%Prove that the sequence $\Big( (1 + 1/n)^n \Big)_{n = 1}^\infty$ converges. [Hint: follow the steps of project 1.1 in the book]. The limit of this sequence is $e$, the Neperian base.
%\end{exer}

%% ---------------------------------------------
\begin{exer}
(5 pts)
Let $(a_n)_{n = 1}^\infty$ be an increasing sequence and $(b_n)_{n = 1}^\infty$ be a decreasing sequence. Let $(c_n)_{n = 1}^\infty$ be the sequence defined by $c_n = b_n - a_n$. Show that if $\lim_{n \ra \infty} c_n = 0$, then the sequences $(a_n)_{n = 1}^\infty$ and $(b_n)_{n = 1}^\infty$ converges and $\lim_{n \ra \infty} a_n = \lim_{n \ra \infty} b_n$.
\end{exer}
\begin{sol}
Suppose that $\lim_{n\ra \infty} c_n = 0$. Since $(a_n)$ is increasing, we have that $a_n \leq a_{n + 1}$ for any $n \geq 1$, so $-a_n \geq -a_{n + 1}$ for any $n \geq 1$. So, $(-a_n)$ is a decreasing sequence. Thus, since $(b_n)$ is decreasing, we get, for any $n \geq 1$,
	\begin{align*}
	c_n = b_n - a_n \geq b_{n + 1} - a_{n + 1} = c_{n + 1} .
	\end{align*}
Thus, $(c_n)$ is decreasing. This implies that, if $m, n \geq 1$ such that $m \leq n$, then $b_m - a_m \geq b_n - a_n = c_n$ and taking $n \ra \infty$, we get that
	\begin{align*}
	b_m - a_m \geq \lim_{n \ra \infty} c_n = 0 .
	\end{align*}
Thus, $a_m \leq b_m$ for any $m \geq 1$.

Since the sequence $(b_n)$ is decreasing, we see that $a_m \leq b_m \leq b_1$ for any $m \geq 1$ and so $(a_n)$ is bounded from above. Since $a_m \geq a_1$ for any $m \geq 1$, the sequence $(a_n)$ is bounded. So by a Theorem from the lecture notes, the sequence $(a_n)$ converges. Also, since $(a_n)$ is increasing, we have $b_m \geq a_m \geq a_1$ for any $m \geq$, so the sequence $(b_n)$ is bounded from below. Since $b_m \leq b_1$ for any $m \geq 1$, we see that the sequence $(b_n)$ is bounded. By a Theorem from the lecture notes, the sequence $(b_n)$ is convergent.

Call $A$ and $B$ the limits of the sequences $(a_n)$ and $(b_n)$. Then, from the sum rules for limits of sequences, we have
	\begin{align*}
	\lim_{n \ra \infty} c_n = \lim_{n \ra \infty} b_n - \lim_{n \ra \infty} a_n = B - A .
	\end{align*}
Since, $\lim_{n \ra \infty} c_n = 0$, then $A = B$.
\end{sol}

%-----------------------------------------------
\begin{exer}
(5 pts)
Let $f: D \subseteq \bR \ra \bR$, and suppose that $x_0$ is an accumulation point of $D$. Suppose that for each sequence $(x_n)_{n = 1}^\infty$ converging to $x_0$ with $x_n \in D\backslash \{ x_0 \}$ for each $n \geq 1$, then the sequence $(f(x_n))_{n = 1}^\infty$ is Cauchy. Show that $f$ has a limit at $x_0$.

[Hint: For two sequences $(x_n)$ and $(y_n)$ that satisfy the assumption, define the sequence $(z_n)$ to be $z_{2n} = x_n$ and $z_{2n - 1} = y_n$. Show that $(f(z_n))$ converges and the sequence $(f(x_n))$ and $(f(y_n))$ converges to the same limit as $(f(z_n))$. Conclude by a theorem in the lecture notes.]
\end{exer}
\begin{sol}
By the assumptions, since $(f(x_n))_{n \geq 1}$ is Cauchy, it is convergent, say to $A (x)$ where $x = (x_n)$. So the limit depends on the choice of the sequence $x$. We will show first that in fact, it does not depend on the choice of the sequence and the values are the same. Let $x = (x_n)_{n \geq 1}$ and $y = (y_n)_{n \geq 1}$ be two sequences satisfying the conditions of the assumptions. Then $\lim_{n \ra \infty} f(y_n) = A(y)$ and $\lim_{n \ra \infty} f(x_n) = A(x)$. Define a new sequence $z = (z_n)$ in the following way
	\begin{itemize}
	\item $z_{2n} = x_n$ for $n \geq 1$.
	\item $z_{2n -1} = y_n$ for $n \geq 1$.
	\end{itemize}
Then the sequence $(z_n)$ satifies the conditions in the assumptions and so the sequence $(f(z_n))$ is Cauchy. It does converge to some limit $A(z)$. However, the sequence $(x_n)$ is a subsequence of $(z_n)$. So, it must also converge to $A(z)$. Then, we must have $A(x) = A(z)$. Also, the sequence $(y_n)$ is a subsequence of $(z_n)$. So, it must converge to $A(z)$. Then, we must also have $A(y) = A(z)$. Thus, $A(x) = A(y)$. 

So, if $(x_n)$ is a sequence that satifies the assumptions, then $(f(x_n))$ converges to some $L \in \bR$ which does not depend on the choice of $(x_n)_{n \geq 0}$. From a Theorem from the lecture notes, we conclude that the fonction has a limit at $x_0$ and $\lim_{x \ra x_0} f(x) = L$.
\end{sol}

%-----------------------------------------------
\begin{exer}
(5 pts)
Prove that if $f : D \subseteq \bR \ra \bR$ has a limit at $x_0 \in \mathrm{acc}\, D$, then the limit is unique.
\end{exer}
\begin{sol}
Suppose that there are two values $L_1$ and $L_2$ with $L_1 \neq L_2$ such that $f(x) \ra L_1$ and $f(x) \ra L_2$ as $x \ra x_0$. Take $\varepsilon := |L_1 - L_2|/2$. Then there is a $\delta_1 > 0$ and $\delta_2 > 0$ such that 
	\begin{itemize}
	\item if $|x - x_0| < \delta$ and $x \in D \backslash \{ x_0 \}$, then $|f(x) - L_1| < |L_1 - L_2|/2$.
	\item if $|x - x_0| < \delta_2$ and $x \in D \backslash \{ x_0 \}$. then $|f(x) - L_2| < |L_1 - L_2|/2$.
	\end{itemize}
let $\delta := \min \{ \delta_1 , \delta_2 \}$ and pick a $x \in D\backslash \{ x_0 \}$ such that $|x - x_0| < \delta$. Then, we have
	\begin{align*}
	|L_1 - L_2| \leq |L_1 - f(x)| + |f(x) - L_2| < |L_1 - L_2|/2 + |L_1 - L_2|/2 = |L_1 - L_2| .
	\end{align*}
So $|L_1 - L_2| < |L_1 - L_2|$, a contradiction. So $L_1 = L_2$.
\end{sol}

%-----------------------------------------------
\begin{exer}
(5 pts)
Suppose $f: D \subseteq \bR \ra \bR$, $g: D \subseteq \bR \ra \bR$ and $h : D \subseteq \bR \ra \bR$ are three functions such that
	\begin{align*}
	f(x) \leq h(x) \leq g(x) \quad (\forall x \in D ) .
	\end{align*}
Suppose that $f$ and $g$ have limits at $x_0$ with $\lim_{x \ra x_0} f(x) = \lim_{x \ra x_0} g(x)$. Prove that $h$ has a limit at $x_0$ and
	$$
	\lim_{x \ra x_0} f(x) = \lim_{x \ra x_0} h(x) = \lim_{x \ra x_0} g(x) .
	$$
\end{exer}
\begin{sol}
Call $L$ the commun limit of $f$ and $g$. We will use the characterization in terms of sequences of the limits of functions. Suppose $(x_n)$ is a sequence such that $x_n \ra x_0$ with $x_n \in D \backslash \{ x_0 \}$ for any $n \geq 1$. Then, we know that $\lim_{n\ra \infty} f(x_n) = \lim_{n \ra \infty} g(x_n) = L$. Also, we have $f(x_n) \leq h(x_n) \leq g(x_n)$ for any $n \geq 1$. From the squeeze Theorem for sequences (see homework 2), we have that $\lim_{n \ra \infty} h(x_n)$ exists and $\lim_{n \ra \infty} h(x_n) = L$. Since this is true for any sequences $(x_n)$, we conclude that $\lim_{x \ra x_0} h(x) = L$.
\end{sol}

%-----------------------------------------------
\begin{exer}
(10 pts)
Let $f : (0, \infty ) \ra \bR$ be a function. We say that $f$ has a limit at $\infty$ if there exists a $L \in \bR$ such that for any $\varepsilon > 0$, there is a real number $M > 0$ such that if $x > M$, then $|f(x) - L| < \varepsilon$. 
	\begin{enumerate}[label=\textbf{\alph*)}]
	\item Show that if $g: (0, \infty ) \ra \bR$ is bounded and $\lim_{x \ra \infty} f(x ) = 0$, then $\lim_{x \ra \infty} f(x) g(x) = 0$.
	\item Let $a > 0$ and suppose that $f: (a, \infty ) \ra \bR$ and define $g : (0, 1/a ) \ra \bR$ by $g(x) = f(1/x)$. Show that $f$ has a limit at $\infty$ if and only if $g$ has a limit at $0$.
	\end{enumerate}
\end{exer}
\begin{sol}
\begin{enumerate}[label=\textbf{\alph*)}]
\item Let $g : (0, \infty ) \ra \bR$ be bounded. Then, there is a $N > 0$ such that $|g(x)|\leq N$ for any $x \in (0, \infty )$. Let $\varepsilon > 0$. Since $f$ has a limit at $\infty$, there is a $M > 0$ such that if $x > M$, then $|f(x)| < \varepsilon/N$. Then, if $x > M$, then
	\begin{align*}
	|f(x) g(x)| \leq |f(x)| N < \varepsilon .
	\end{align*}
So, $\lim_{x \ra \infty} f(x) g(x) = 0$.
\item Suppose $f$ has a limit at $\infty$, say $L$. Then, there is a $M > 0$ such that if $x > M$, then $|f(x) - L| < \varepsilon$. Take $\delta := \min \{ 1/M , 1/a \}$. Then, if $x < \delta$, then $1/x > M$ and so
	\begin{align*}
	| g(x) - L | = |f(1/x) - L| < \varepsilon .
	\end{align*}
Thus, $\lim_{x \ra 0} g(x) = L$.

Suppose now that $\lim_{x \ra 0} g(x) = L$. Then there is a $\delta > 0$ such that $|x| < \delta$ and $0 < x < 1/a$, then $|g(x) - L| < \varepsilon$. Take $M := \frac{1}{\delta}$. Then, if $x > M$, then $\frac{1}{x} < 1/M = \delta$. This implies that
	\begin{align*}
	|f(x) - L| = |g(1/x) - L| < \varepsilon .
	\end{align*}
So $\lim_{x \ra \infty} f(x) = L$.
\end{enumerate}
\end{sol}



\section{Homework problems}
Answer all the questions below. Make sure to show your work.

\begin{exer}
(10pts)
For each of the sequences below, determine its nature (converges or diverges)\footnote{You don't need to compute the limit.}:
	\begin{enumerate}[label=\textbf{\alph*)}]
	\item $(a_n)$ where $a_n = \frac{1}{n} + \frac{1}{n + 1} + \cdots + \frac{1}{2n}$.
	\item $(a_n)$ where $a_n = \frac{1 + 2 + \cdots + n}{n^2}$.
	\end{enumerate}
\end{exer}
\begin{sol}
\begin{enumerate}
\item The sequence $(a_n)$ is decreasing and bounded. 
	\begin{itemize}
	\item Decreasing: We have $a_{n+ 1} \leq a_n$ iff
		\begin{align*}
		\sum_{k = n+1}^{2n+2} \frac{1}{k} \leq \sum_{k = n}^{2n} \frac{1}{k}
		\end{align*}
	which is iff
		\begin{align*}
		\frac{1}{2n + 1} + \frac{1}{2n + 2} \leq \frac{1}{n} .
		\end{align*}
	But, $2n + 1, 2n + 2 \geq 2n$ and so $\frac{1}{2n + 1}, \frac{1}{2n + 2} \leq \frac{1}{2n}$. This implies that
		\begin{align*}
		\frac{1}{2n + 1} + \frac{1}{2n + 1} \leq \frac{2}{2n} = \frac{1}{n}.
		\end{align*}
	So, we conclude that $a_{n+ 1} \leq a_n$.
	\item Bounded: Since $(a_n)$ is decreasing, $a_n \leq a_1 = \frac{3}{2}$. Also, since for any $k \geq n$, $\frac{1}{k} \leq \frac{1}{n}$, we get
		\begin{align*}
		\sum_{k = n}^{2n} \frac{1}{k} \geq \sum_{k = n}^{2n} \frac{1}{2n} = \frac{n + 1}{n} \geq 1.
		\end{align*}
	\end{itemize}
So, since $(a_n)$ is decreasing and bounded, it must converge to a limit.
\item Since $\sum_{k = 1}^n k = \frac{n (n + 1)}{2}$, we get $a_n = \frac{n + 1}{2n}$. The sequence $(a_n)$ is bounded and increasing.
	\begin{itemize}
	\item Increasing: For any $n \geq 1$, we have $1/(n + 1) \leq 1/n$, so
		\begin{align*}
		a_n= (n+ 1)/2n = \frac{1}{2} + \frac{1}{2n} \geq \frac{1}{2} + \frac{1}{2 (n + 1)} = a_{n + 1} .
		\end{align*}
	\item Bounded: Since $(a_n)$ is decreasing, $a_n \leq a_1 = 1$. Also, since all terms in the sequence of positive, we have $a_n \geq 0$. 
	\end{itemize}
Since $a_n$ is decreasing and bounded, it must converge to a limit.

Another way is to use the limit rules. We know that $\frac{n + 1}{2n} = \frac{1}{2} + \frac{1}{2n}$. So, this limit must converge to $\frac{1}{2}$.
\end{enumerate}
\end{sol}

%------------------------------------------------
%\begin{exer}
%Define $f : (-2, 0) \ra \bR$ by $f(x) = \frac{2x^2 + 3x - 2}{x + 2}$. Prove that $f$ has a limit at $x_0 = -2$, and find it.
%\end{exer}

%------------------------------------------------
\begin{exer}
(5 pts)
Define $g: (0, 1) \ra \bR$ by $f(x) = \frac{\sqrt{1 + x} - 1}{x}$. Prove that $g$ has a limit at $0$ and find it.
\end{exer}
\begin{sol}
We have $g(x) = \frac{1}{\sqrt{x + 1} + 1}$. The function $x \mapsto 1$ has a limit at $0$ and $x \mapsto \sqrt{x + 1} - 1$ has a limit at $0$ which is $\sqrt{1} + 1 \neq 0$. So, by the quotient rule, the limit of $g$ exists at $0$ and
	\begin{equation*}
	\lim_{x \ra 0} g(x) = \frac{\lim_{x \ra 0} 1}{\lim_{x \ra 0} \sqrt{1 + x} + 1} = \frac{1}{2} .
	\end{equation*}
\end{sol}

%------------------------------------------------
\begin{exer}
(5 pts)
Suppose that $f: (0, 1) \ra \bR$ has a limit at $x_0 = 1$ and $\lim_{x \ra 1} f(x) = 1$. Compute the value of the limit
	\begin{align*}
	\lim_{x \ra 1} \frac{f(x) (1 - f(x)^2)}{1 - f(x)} .
	\end{align*}
\end{exer}
\begin{sol}
We have $\frac{f(x) (1 - f(x)^2)}{1 - f(x)} = \frac{f(x) (1 + f(x)) (1 - f(x))}{1 - f(x)} = f(x) (1 + f(x))$. So, by the product rule, we get
	\begin{align*}
	\lim_{x \ra 1} \frac{f(x) (1 - f(x)^2)}{1 - f(x)} = \lim_{x \ra 1} f(x) (1 + f(x)) = 2 .
	\end{align*}
\end{sol}

%------------------------------------------------
\begin{exer}
(5 pts)
Prove that if $f: D \ra \bR$ has a limit at $x_0$, then $|f|(x) := |f(x)|$ has a limit at $x_0$.
\end{exer}
\begin{sol}
Let $\varepsilon > 0$. Since $f$ has a limit at $x_0$, say $L$, there is a $\delta > 0$ such that if $|x - x_0| < \delta$ and $x \in D\backslash \{ x_0 \}$, then $|f(x) - L| < \varepsilon$. Using the reverse triangle inequality, we have $||f(x) | - |L|| \leq |f(x) - L|$. So, if $|x - x_0| < \delta$ and $x \in D \backslash \{ x_0 \}$, then $| |f(x)| - |L| | \leq |f(x) - L| < \varepsilon$. So, $\lim_{x \ra x_0} |f(x)| = |L|$.
\end{sol}

%-------------------------------------------------
\begin{exer}
(10 pts)
Using the link between sequences and limits of functions, show the following.
	\begin{enumerate}[label=\textbf{\alph*)}]
	\item If $f(x) = x^n$ ($n \geq 0$), then $\lim_{x \ra x_0} f(x) = x_0^n$ for any $x_0 \in \bR$.
	\item If $x_0 \in [0, \infty )$, then $\lim_{x \ra x_0} \sqrt{x} = \sqrt{x_0}$.
\end{enumerate}	 
\end{exer}
\begin{sol}
\begin{enumerate}[label=\textbf{\alph*)}]
\item Let $(x_n)$ be a sequence such that $x_k \ra x_0$ and $x_k \in \bR \backslash \{ x_0 \}$. Using induction (using the product rule), we can prove that $x_k^n \ra x_0^n$. So, the limit of $(f(x_k))_{k = 1}^\infty$ exists and is $x_0^k$. By a Theorem from the lecture notes, we conclude that the limit exists and must be $x_0^n$.
\item Let $x_0 \in [0, \infty )$ and $(x_n)$ be a sequence such that $x_n \ra x_0$ with $x_n \in [0, \infty ) \backslash \{ x_0 \}$. Then as we see for sequences (see homework 2), we have $\sqrt{x_n} \ra \sqrt{x_0}$. Since $(x_n)$ was arbitrary, from a Theorem on characterization of limits in terms of sequences, we conclude that $\lim_{x \ra x_0} \sqrt{x}$ exists and must be $\sqrt{x_0}$.
\end{enumerate}
\end{sol}

\section{Bonus}
\begin{exer}
Assume that $f : \bR \ra \bR$ such that $f(x + y) = f(x) f(y)$ for all $x, y \in \bR$. Suppose that $f$ has a limit at some point.
	\begin{enumerate}[label=\textbf{\alph*)}]
	\item Show that $f$ has a limit at every point of $\bR$.
	\item Show that either $\lim_{x \ra 0} f(x) = 1$ or $f(x) = 0$ for any $x \in \bR$.
	\end{enumerate}
\end{exer}
\begin{sol}
\begin{enumerate}[label=\textbf{\alph*)}]
\item Let $L$ be the limit of $f$ at the point $x_0$. Remark that the limit of a function $f$ at a point $x_0$ can be rewritten as
	\begin{align*}
	\lim_{h \ra 0} f (x_0 + h)  = L 
	\end{align*}
which means that for any $\varepsilon > 0$, there is a number $\delta > 0$ such that if $|h| < \delta$ and $h \neq 0$, then $|f(x_0 + h) - L| < \varepsilon$.

 Let $h \in \bR$. Then, $f(x_0 + h) = f(x_0) f(h)$. So, if $f(x_0) \neq 0$, we get that $f(h) = \frac{f(x_0 + h)}{f(x_0)}$. We know that the limit as $h \ra 0$ on the right-hand side exists and by the quotient rule, is equal to $L/f(x_0)$. So, the limit of $f$ at $0$ exists and is $L/f(x_0)$. Now, since for any $y \in \bR$, we have $f(y + h) = f(y) f(h)$, we conclude that the limit exists at $y$, and the limit is equal to $L f(y)/f(x_0)$.
 
\item We made the assumption that $f(x_0) \neq 0$. If $f(x_0) = 0$, then for any $x \in \bR$, we have $f(x) = f(x - x_0) f(x_0) = 0$. Thus, $f$ is identically zero on $\bR$. In the case $f(x_0 ) \neq 0$, then we know that the limit at $x = 0$ exists and is equal to $L/f(x_0)$. Also, since $0 = 0 + 0$, we have $f(0) = f(0)^2$ and so $f(0) = 1$ (otherwise $f$ is identically zero). 

Let $n$ be a natural number. Then $f(1) = 1$ if $n = 1$. Also, if $n = 2$, we have $f(1) = f(1/2 + 1/2) = f(1/2)^2$ and so $f(1/2) = \sqrt{f(1)}$. By induction, we can show that $f(1/n) = \sqrt[n]{f(1)}$. For sure, $f(1) > 0$, otherwise, we would have $f(1/2) = \sqrt{f(1)}$ with $f(1) < 0$ (which is impossible for real numbers since the square of any number is positive or zero). This implies that 
	$$
	\lim_{n \ra \infty} f(1/n) = \lim_{n \ra \infty} (f(1))^{1/n} = 1
	$$
by a computation we did in the lecture notes on the sequences (see last example). Since taking any sequence $h_n \ra 0$ results in the same limit, we must have that
	\begin{align*}
	\lim_{x \ra 0} f(x) = 1 = f(0) .
	\end{align*}
\end{enumerate}
\end{sol}

\end{document}