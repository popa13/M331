\documentclass[12pt]{article}
\usepackage[utf8]{inputenc}

\usepackage{enumitem}
\usepackage[margin=2cm]{geometry}

\usepackage{amsmath, amsfonts, amssymb}
\usepackage{graphicx}
\usepackage{tikz}
\usepackage{pgfplots}
\usepackage{multicol}

\usepackage{comment}
\usepackage{url}
\usepackage{calc}

\usepackage{titlesec}
\titleformat{\section}[frame]
{\normalfont\scshape}
{6.\thesection}{8pt}{\centering}

\usepackage{array}

\pgfplotsset{compat=1.16}

%\usepackage[margin=1cm]{cloze}

\usepackage[thmmarks]{ntheorem}

% MATH commands
\newcommand{\bC}{\mathbb{C}}
\newcommand{\bR}{\mathbb{R}}
\newcommand{\bN}{\mathbb{N}}
\newcommand{\bQ}{\mathbb{Q}}
\newcommand{\bZ}{\mathbb{Z}}
\newcommand{\bT}{\mathbb{T}}
\newcommand{\bD}{\mathbb{D}}

\newcommand{\cL}{\mathcal{L}}
\newcommand{\cM}{\mathcal{M}}
\newcommand{\cP}{\mathcal{P}}
\newcommand{\cH}{\mathcal{H}}
\newcommand{\cB}{\mathcal{B}}
\newcommand{\cK}{\mathcal{K}}
\newcommand{\cJ}{\mathcal{J}}
\newcommand{\cU}{\mathcal{U}}
\newcommand{\cO}{\mathcal{O}}
\newcommand{\cA}{\mathcal{A}}
\newcommand{\cC}{\mathcal{C}}

\newcommand{\fK}{\mathfrak{K}}
\newcommand{\fM}{\mathfrak{M}}

\newcommand{\ga}{\left\langle}
\newcommand{\da}{\right\rangle}
\newcommand{\oa}{\left\lbrace}
\newcommand{\fa}{\right\rbrace}
\newcommand{\oc}{\left[}
\newcommand{\fc}{\right]}
\newcommand{\op}{\left(}
\newcommand{\fp}{\right)}

\newcommand{\ra}{\rightarrow}
\newcommand{\Ra}{\Rightarrow}

\renewcommand{\Re}{\mathrm{Re}\,}
\renewcommand{\Im}{\mathrm{Im}\,}
\newcommand{\Arg}{\mathrm{Arg}\,}
\newcommand{\Arctan}{\mathrm{Arctan}\,}
\newcommand{\sech}{\mathrm{sech}\,}
\newcommand{\csch}{\mathrm{csch}\,}
\newcommand{\Log}{\mathrm{Log}\,}
\newcommand{\cis}{\mathrm{cis}\,}

\newcommand{\ran}{\mathrm{ran}\,}
\newcommand{\bi}{\mathbf{i}}
\newcommand{\Sp}{\mathrm{span}\,}
\newcommand{\Inv}{\mathrm{Inv}\,}
\newcommand\smallO{
  \mathchoice
    {{\scriptstyle\mathcal{O}}}% \displaystyle
    {{\scriptstyle\mathcal{O}}}% \textstyle
    {{\scriptscriptstyle\mathcal{O}}}% \scriptstyle
    {\scalebox{.7}{$\scriptscriptstyle\mathcal{O}$}}%\scriptscriptstyle
  }
\newcommand{\HOL}{\mathrm{Hol}}
\newcommand{\cl}{\mathrm{clos}}
\newcommand{\ve}{\varepsilon}

\tikzstyle{myboxT} = [draw=black, fill=black!0,line width = 1pt,
    rectangle, rounded corners = 0pt, inner sep=8pt, inner ysep=8pt]
    
\newcommand{\MyC}[1]{\begin{tikzpicture}
\node (boxIntro) at (0,0) {};
\node [myboxT](Intro) at (boxIntro){%
	\begin{minipage}{0.9\textwidth}
	#1
	\end{minipage}};
\end{tikzpicture}}

%%%%  Environnement exer et solutionnaire
{\theorembodyfont{}
\theoremstyle{plain}
\theoremseparator{\textbf{.}}
\theoremsymbol{}
\newtheorem{exer}{\textbf{Exercise}}}

{\theorembodyfont{\color{blue}}
\theoremstyle{plain}
\theoremseparator{\textbf{:}}
\theoremsymbol{$\square$}
\newtheorem*{sol}{\textbf{Solution}}}

{\theorembodyfont{\color{blue}}
\theoremstyle{plain}
\theoremseparator{\textbf{:}}
\theoremsymbol{$\qed$}
\newtheorem*{solWP}{\textbf{Solution}}}

{\theorembodyfont{\color{blue}}
\theoremstyle{plain}
\theoremseparator{\textbf{:}}
\theoremsymbol{$\qed$}
\newtheorem*{hint}{\textbf{Hints}}}

\renewcommand*{\theexer}{\arabic{exer}}
\renewcommand*{\thesol}{\arabic{sol}}
\renewcommand*{\thesolWP}{\arabic{solWP}}
\renewcommand*{\thehint}{\arabic{hint}}

%%% Ignorer les solutions
\excludecomment{sol}
\excludecomment{solWP}
\excludecomment{hint}

\makeatletter
\DeclareFontFamily{U}{tipa}{}
\DeclareFontShape{U}{tipa}{m}{n}{<->tipa10}{}
\newcommand{\arc@char}{{\usefont{U}{tipa}{m}{n}\symbol{62}}}%

\newcommand{\arc}[1]{\mathpalette\arc@arc{#1}}

\newcommand{\arc@arc}[2]{%
  \sbox0{$\m@th#1#2$}%
  \vbox{
    \hbox{\resizebox{\wd0}{\height}{\arc@char}}
    \nointerlineskip
    \box0
  }%
}
\makeatother

\title{Chapter 6 : Integral \\ Suggested Exercises}
\author{MATH-331 Introduction to Real Analysis \\
Pierre-Olivier Paris{\'e}}
\date{}

\begin{document}
	
	\MyC{%
	\maketitle}
	
\vspace*{0.5cm}
	
You will find some suggested exercises related to the Riemann integral. Try to solve all of them or, at least, understand the solutions of all the problems in this document.

\section{Definition}
\begin{exer}
Let $f(x) = x^2 + x$ defined on the interval $[-1, 2]$. Find the Riemann sum of $f$ corresponding to the following tagged partition of $[-1, 2]$:
	\begin{align*}
	\cP &:= \{ (-9, [-1, -.8]) , (-.7, [-.8, -.3]), (-.1, [-.3, 0]), (.2, [0,0.2]), (.2, [.2, .4]), (.8, [.4, 1]), \\
	& \phantom{:= \{} (1.42, [1, 1.5]), (1.9, [1.5, 2]) \} .
	\end{align*}
\end{exer}

\begin{exer}
Let $\cP$ be the tagged partition of the last exercise. Find a tagged partition $\cP_1$ of $[-1, 2]$ such that $\Vert \cP_1 \Vert \leq \Vert \cP \Vert/3$.
\end{exer}

\begin{exer}
Prove that the number $L$ in the definition of the Riemann integral is unique.
\end{exer}

\begin{exer}
Suppose that $f(x) = k$ for all $x \in [a, b]$ where $k \in \bR$ is a constant. prove that $f$ is Riemann integrable on $[a, b]$ and $\int_a^b f = k (b - a)$.
\end{exer}

\begin{exer}
Prove that the function $f: [0 , 1] \ra \bR$ defined by
	\begin{align*}
	f(x) := \begin{cases}
	0 &\text{, if } x \neq 0 \\
	4 &\text{, if } x = 0 
	\end{cases}
	\end{align*}
is Riemann integrable on $[0, 1]$.
\end{exer}

\begin{exer}
Prove that the function $h(x) = x$ is Riemann integrable on $[0, 1]$.
\end{exer}

\section{Properties}
\begin{exer}
Let $f : [a, b] \ra \bR$ be Riemann integrable on $[a, b]$ and suppose that $|f(x)| \leq M$ for any $x \in [a, b]$. Prove that $| \int_a^b f | \leq M (b-a)$.
\end{exer}

\begin{exer}
Let $f : [a, b] \ra \bR$ and $g : [a, b] \ra \bR$ be two Riemann integrable functions on $[a, b]$.
	\begin{enumerate}[label=\textbf{\alph*)}]
	\item Show that $f + g$ is Riemann integrable and that $\int_a^b (f + g) = \int_a^b f + \int_a^b g$.
	\item Show that if $f(x) \leq g(x)$ for all $x \in [a, b]$, then $\int_a^b f(x) \leq \int_a^b g(x)$.
	\end{enumerate}
\end{exer}

\begin{exer}
Prove that the function $f : \bR \ra \bR$ defined by $f(x) = 1$ if $x \in \bQ$ and $f(x) = 0$ if $x \not\in \bQ$ is not Riemann integrable on $[0, 1]$.
\end{exer}

\begin{exer}
Suppose that $f$ is Riemann integrable on $[a, b]$. Let $(\cP_n )_{n = 1}^\infty$ be a sequence of tagged partitions of $[a, b]$ such that $\Vert \cP_n \Vert \ra 0$ as $n \ra \infty$. Prove that the sequence $(S(f, \cP_n ))_{n = 1}^\infty$ of Riemann sums converges to $\int_a^b f$.
\end{exer}

\begin{exer}
Suppose that $f : [a, b] \ra \bR$ is a bounded function and that $f$ is Riemann integrable on $[a, c]$ for each $c \in (a, b)$. Prove that $f$ is Riemann integrable on $[a, b]$.
\end{exer}

\begin{exer}
Prove that if $f : [a, b] \ra \bR$ is Riemann integrable on $[a, b]$ and if $g : [a, b ] \ra \bR$ is a function such that $g$ differed from $f$ only at one point, then $g$ is Riemann integrable and $\int_a^b f = \int_a^b g$. 
\end{exer}

\section{Types of Riemann Integral functions}
\begin{exer}
Consider the step function $\phi : [0, 3] \ra \bR$ defined by 
	\begin{align*}
	\phi = 4 \chi_{[0, 1]} - 3 \chi_{(1, 1.5)} + 2 \chi_{[1.5, 2.5]} + \chi_{(2.5, 3]} .
	\end{align*}
\end{exer}

\begin{exer}
Let $\phi : [a, b] \ra \bR$ be a step function and suppose that $\psi : [a, b] \ra \bR$ differs from $\phi$ in a finite number of points. Is $\psi$ a step functions? Explain.
\end{exer}

\begin{exer}
Let $f : [a, b] \ra \bR$ be an increasing function and let $\cP$ be a tagged partition of $[a, b]$. Suppose that $\cP_1$ is another tagged partition of $[a, b]$ with exactly the same intervales as $\cP$, only tags are different. Prove that
	\begin{align*}
	|S (f , \cP ) - S (f , \cP_1) | \leq |f (b) - f(a)| \Vert \cP \Vert .
	\end{align*}
\end{exer}

\begin{exer}
Consider the function $f : [0, 6] \ra \bR$ defined by
	\begin{align*}
	f(x) = \begin{cases}
	\sin x & \text{, if } 0 \leq x \leq 3 \\
	10 - x & \text{, if } 3 < x < 4 \\
	1/x & \text{, if } 4 \leq x \leq 6 .
	\end{cases}
	\end{align*}
Explain carefully why $f$ is Riemann integrable on $[0, 6]$.
\end{exer}

\begin{exer}
Suppose that $f$ is continuous and nonnegative on $[a, b]$. Show that if $\int_a^b f = 0$, then $f(x) = 0$ for every $x \in [a, b]$. Give an example to show that the nonnegative hypothesis is necessary.
\end{exer}

\begin{exer}
Suppose that $f$ and $g$ are continuous on $[a, b]$ and that $\int_a^b f = \int_a^b g$. Prove that there exists a point $c \in [a, b]$ such that $f(c) = g(c)$.
\end{exer}

\section{Fundamental Theorem of calculus}
\begin{exer}
Find the derivative of the function $F(x) =\int_{-x^4}^{x^2} t^2 \sin (t^2) \, dt$.
\end{exer}

\begin{exer}
Evaluate $\int_{1/9}^1 \frac{\sin (\pi \sqrt{t}/2 )}{\sqrt{t}} \, dt$.
\end{exer}
	
\begin{exer}
Give an example of a function such that the derivative is not Riemann Integrable.
\end{exer}

\begin{exer}
Suppose that $f$ is continuous and nonnegative on $[a, b]$ and that $\int_a^b f = 0$. Use the Fundamental Theorem of Calculus to prove that $f(x) = 0$ for every $x \in [a, b]$.
\end{exer}

\begin{exer}
Suppse that $f$ and $g$ are continuous functions on $[a, b]$ and that $\int_a^b f = \int_a^b g$. Use the Fundamental Theorem of Calculus to prove that there exists a point $c \in [a, b]$ such that $f(c) = g(c)$.
\end{exer}

\section{Additional Algebraic Properties}
\begin{exer}
Find a function $f : [0, 1] \ra \bR$ that is not Riemann integrable on $[0, 1]$ and a Riemann integrable function $g : [0, 1] \ra \bR$ such that $g \circ f$ is Riemann integrable.
\end{exer}

\begin{exer}
Suppose that $f$ and $g$ are differentiable functions on $[a, b]$ and that $f'$ and $g'$ are Riemann integrable on $[a, b]$. Prove that $f'g$ and $f g'$ are Riemann integrable on $[a, b]$ and that
	\begin{align*}
	\int_a^b f' g = f(b) g(b) - f(a) g(a) - \int_a^b f g' .
	\end{align*}
What is the name of this integration formula?
\end{exer}

\begin{exer}
Use the previous exercise to compute $\int_0^{\pi/3} x \sin x \, dx$.
\end{exer}

\begin{exer}
Let $g : [a, b] \ra [c, d]$ be differentiable on $[a, b]$ and let $f : [c, d] \ra \bR$ be continuous on $[c, d]$. Suppose that $g'$ is Riemann integrable on $[a, b]$. Prove that $(f \circ g)g'$ is Riemann integrable on $[a, b]$ and that
	\begin{align*}
	\int_a^b (f \circ g) g' = \int_{g(a)}^{g(b)} f .
	\end{align*}
What is the name of this integration formula?
\end{exer}

\begin{exer}
Use the previous exercise to evaluate $\int_0^3 \frac{6}{\sqrt{x} (1 + x)} \, dx$.
\end{exer}

\section*{Miscellaneous Problems}

\begin{exer}
Let $f$ be a continuous function on $[a, b]$. Prove that there exists a point $c \in [a, b]$ such that $f(c) (b - a) = \int_a^b f$. [This result is often called the Mean Value Theorem for integrals.]
\end{exer}

\begin{exer}
Suppose that $f$ is a continuous function on $[a, b]$ and that $g$ is a nonnegative Riemann integrable on $[a, b]$. Prove that there exists $c \in [a, b]$ such that $f(c) \int_a^b g = \int_a^b fg$. Show that this result is a generalization of the previous exercise.
\end{exer}

\begin{exer}
Suppose that $f : [a, b] \ra \bR$ is monotone. Prove that there exists a point $c \in (a, b)$ such that
	\begin{align*}
	\int_a^b f = f(a) (c - a) + f(b) (b - c) .
	\end{align*}
\end{exer}

\end{document}