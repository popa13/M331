\documentclass[addpoints, 12pt]{exam}%, answers]
\usepackage[utf8]{inputenc}
\usepackage[T1]{fontenc}

\usepackage{lmodern}
\usepackage{arydshln}
\usepackage[margin=2cm]{geometry}

\usepackage{enumitem}

\usepackage{amsmath, amsthm, amsfonts, amssymb}
\usepackage{graphicx}
\usepackage{tikz}
\usetikzlibrary{arrows,calc,patterns}
\usepackage{pgfplots}
\pgfplotsset{compat=newest}
\usepackage{url}
\usepackage{multicol}
\usepackage{thmtools}

\usepackage{caption}
\usepackage{subcaption}

\usepackage{pifont}

% MATH commands
\newcommand{\bC}{\mathbb{C}}
\newcommand{\bR}{\mathbb{R}}
\newcommand{\bN}{\mathbb{N}}
\newcommand{\bZ}{\mathbb{Z}}
\newcommand{\bT}{\mathbb{T}}
\newcommand{\bD}{\mathbb{D}}

\newcommand{\cL}{\mathcal{L}}
\newcommand{\cM}{\mathcal{M}}
\newcommand{\cP}{\mathcal{P}}
\newcommand{\cH}{\mathcal{H}}
\newcommand{\cB}{\mathcal{B}}
\newcommand{\cK}{\mathcal{K}}
\newcommand{\cJ}{\mathcal{J}}
\newcommand{\cU}{\mathcal{U}}
\newcommand{\cO}{\mathcal{O}}
\newcommand{\cA}{\mathcal{A}}
\newcommand{\cC}{\mathcal{C}}
\newcommand{\cF}{\mathcal{F}}

\newcommand{\fK}{\mathfrak{K}}
\newcommand{\fM}{\mathfrak{M}}

\newcommand{\ga}{\left\langle}
\newcommand{\da}{\right\rangle}
\newcommand{\oa}{\left\lbrace}
\newcommand{\fa}{\right\rbrace}
\newcommand{\oc}{\left[}
\newcommand{\fc}{\right]}
\newcommand{\op}{\left(}
\newcommand{\fp}{\right)}

\newcommand{\ra}{\rightarrow}
\newcommand{\Ra}{\Rightarrow}

\renewcommand{\Re}{\mathrm{Re}\,}
\renewcommand{\Im}{\mathrm{Im}\,}
\newcommand{\Arg}{\mathrm{Arg}\,}
\newcommand{\Arctan}{\mathrm{Arctan}\,}
\newcommand{\sech}{\mathrm{sech}\,}
\newcommand{\csch}{\mathrm{csch}\,}
\newcommand{\Log}{\mathrm{Log}\,}
\newcommand{\cis}{\mathrm{cis}\,}

\newcommand{\ran}{\mathrm{ran}\,}
\newcommand{\bi}{\mathbf{i}}
\newcommand{\Sp}{\mathrm{span}\,}
\newcommand{\Inv}{\mathrm{Inv}\,}
\newcommand\smallO{
  \mathchoice
    {{\scriptstyle\mathcal{O}}}% \displaystyle
    {{\scriptstyle\mathcal{O}}}% \textstyle
    {{\scriptscriptstyle\mathcal{O}}}% \scriptstyle
    {\scalebox{.7}{$\scriptscriptstyle\mathcal{O}$}}%\scriptscriptstyle
  }
\newcommand{\HOL}{\mathrm{Hol}}
\newcommand{\cl}{\mathrm{clos}}
\newcommand{\ve}{\varepsilon}

\DeclareMathOperator{\dom}{dom}

%%%%%% Définitions Theorems and al.
%\declaretheoremstyle[preheadhook = {\vskip0.2cm}, mdframed = {linewidth = 2pt, backgroundcolor = yellow}]{myThmstyle}
%\declaretheoremstyle[preheadhook = {\vskip0.2cm}, postfoothook = {\vskip0.2cm}, mdframed = {linewidth = 1.5pt, backgroundcolor=green}]{myDefstyle}
%\declaretheoremstyle[bodyfont = \normalfont , spaceabove = 0.1cm , spacebelow = 0.25cm, qed = $\blacktriangle$]{myRemstyle}

%\declaretheorem[ style = myThmstyle, name=Th\'eor\`eme]{theorem}
%\declaretheorem[style =myThmstyle, name=Proposition]{proposition}
%\declaretheorem[style = myThmstyle, name = Corollaire]{corollary}
%\declaretheorem[style = myThmstyle, name = Lemme]{lemma}
%\declaretheorem[style = myThmstyle, name = Conjecture]{conjecture}

%\declaretheorem[style = myDefstyle, name = D\'efinition]{definition}

%\declaretheorem[style = myRemstyle, name = Remarque]{remark}
%\declaretheorem[style = myRemstyle, name = Remarques]{remarks}

\newtheorem{theorem}{Théorème}
\newtheorem{corollary}{Corollaire}
\newtheorem{lemma}{Lemme}
\newtheorem{proposition}{Proposition}
\newtheorem{conjecture}{Conjecture}

\theoremstyle{definition}

\newtheorem{definition}{Définition}[section]
\newtheorem{example}{Exemple}[section]
\newtheorem{remark}{\textcolor{red}{Remarque}}[section]
\newtheorem{exer}{\textbf{Exercice}}[section]


\tikzstyle{myboxT} = [draw=black, fill=black!0,line width = 1pt,
    rectangle, rounded corners = 0pt, inner sep=8pt, inner ysep=8pt]

\begin{document}
	\noindent \hrulefill \\
	MATH-331 Intro. to Real Analysis \hfill Pierre-Olivier Paris{\'e}\\
	Team test 03 \hfill Fall 2021, 12/03/2021\\\vspace*{-0.7cm}
	
	\noindent\hrulefill
	
\vspace*{1cm}

\noindent\makebox[\textwidth]{\textbf{Name of the members of the team:}\enspace \hrulefill}
\noindent\makebox[\textwidth]{\hrulefill}
\makebox[\textwidth]{\textbf{Team name (if any):}\enspace\hrulefill}

\vspace*{1cm}

\vspace*{1cm}

{\bf Instructions:} You must answer all the questions in teams of $3$ and hand out one copy per team. There are \underline{\textsc{2 questions}} on this test. The scores for each questions is on the last page of the test.

You are allowed to use the lecture notes only. No other tools such as a cell-phone, a calculator, or a laptop. Only your pen and eraser. The space between the questions are there to write the final versions of your answers.

\vspace*{5cm}

Be the better version of yourself!

\qformat{\rule{0.3\textwidth}{.4pt} \begin{large}{\textsc{Question}} \thequestion \end{large} \hrulefill \hspace*{0.1cm} \textbf{(\totalpoints\hspace*{0.1cm} pts)}}

\newpage

\begin{questions}

\question
Let $f : (a, b] \ra \bR$ be a function where $a < b$.

	\begin{parts}
	\part[5]
	How would you define the Riemann integral of $f$ on $(a, b]$? Explain in details your definition.
	\begin{solution}
	First, we have to make sure that the function is Riemann integrable on each $[c, b]$ where $a < c < b$. This is well-defined because the Riemann integral on closed intervals were defined in the lecture notes. Now, we define the integral of $f$ on $(a, b]$ by taking the limit as $c$ goes to $a$ of the integral of $f$ on $[c, b]$. So the definition will be: $f : (a, b] \ra \bR$ is Riemann integrable on $(a, b]$ if
		\begin{itemize}
		\item $f$ is Riemann integrable on $[c, b]$ for any $c \in (a, b)$.
		\item the $\lim_{c \ra a^+} \int_c^b f$ exists.
		\end{itemize}
	We then define the integral of $f$ from $a$ to $b$ by
		\begin{align*}
		\int_a^b f = \lim_{c \ra a^+} \int_c^b .
		\end{align*}
	\end{solution}
	
	\part[5]
	Find a function $f : (0, 1] \ra \bR$ that is Riemann integrable on $(0, 1]$ (with respect to your definition) but is unbounded on $(0, 1]$.
	\begin{solution}
	Take $f(x) = \frac{1}{\sqrt{x}}$ defined on $(0, 1]$. The integral on $[c, 1]$ (for $0 < c < 1$) is
		\begin{align*}
		\int_c^1 \frac{1}{\sqrt{x}} \, dx = \left. 2 x^{1/2} \right|_c^1 = 2 (1 - \sqrt{c}) .
		\end{align*}
	So, as $c \ra 0^+$, we get
		\begin{align*}
		\int_0^1 \frac{1}{\sqrt{x}} \, dx = \lim_{c \ra 0^+} 2 (1 - \sqrt{c}) = 2 .
		\end{align*}
	We see that $f(x) = 1/\sqrt{x}$ is unbounded on $(0, 1]$ because $\lim_{x \ra 0^+} f(x) = +\infty$.
	\end{solution}
	\end{parts}
	
	
\newpage

\question[10]
Find the limit of the sequence $(a_n)_{n =1}^\infty$ if
	\begin{align*}
	a_n = \sum_{k = 1}^n \frac{k}{k^2 + n^2}.
	\end{align*}

	\begin{solution}
	We have
		\begin{align*}
		a_n = \sum_{k = 1} \frac{1}{n} \Big( \frac{k/n}{1 + (k/n)^2} \Big) .
		\end{align*}
	This, as $n \ra \infty$, represents the integral from $0$ to $1$ of the function $f(x) = \frac{x}{1 + x^2}$. So,
		\begin{align*}
		\lim_{n \ra \infty} a_n = \int_0^1 \frac{x}{1 + x^2} \, dx = (1/2) \log (2) .
		\end{align*}
	\end{solution}
	
	
\end{questions}


\vfill
\noindent\hrulefill \hspace{0.1cm} \textsc{Scores table}\hspace{0.1cm} \hrulefill

\begin{center}
\gradetable[h][questions]
\end{center}

\end{document}