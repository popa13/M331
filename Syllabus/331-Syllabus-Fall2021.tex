\documentclass[12pt]{amsart}

\usepackage{xcolor}
\usepackage[hidelinks]{hyperref}
\usepackage[margin=1.1in, tmargin=1in, bmargin=1in]{geometry}

\pagestyle{myheadings}
%\markleft{Syllabus for Math 331, Spring 2020}
%\markright{Syllabus for Math 331, Spring 2020}

\newcommand{\spacer}{\vspace{.2cm}}
\newcommand{\svs}{\vspace{.1cm}}

\newcommand{\red}[1]{\textcolor{red}{#1}}
\definecolor{gold}{rgb}{0.80,0.68,0.00}\newcommand{\gold}[1]{\textcolor{gold}{#1}}

\begin{document}

\noindent\hrulefill
\begin{center}
\large{\textsc{Math 331 Introduction to Real Analysis} } \\
\textsc{Fall 2021 Syllabus}
\end{center}
\hrulefill \\
Lecture: MWF 12:30--1:20pm \hfill HOLMES 247

\spacer

\noindent\textbf{Instructor:} Pierre-Olivier Parise (email: \texttt{parisepo@hawaii.edu})\\
Office: Physical Science Building (PSB) 302\\
Office hours: MW 1:30--3:30pm

\noindent\hrulefill

\section*{Course description}
This course is intended to introduce the student to the fundamentals of real analysis. The content to be covered includes axioms of real numbers, sequences, limits of functions, continuity, differentiation, the Riemann integral, as well as infinite series and sequences/series of functions (if time permits).

{\bf WI course:}
This course is part of the five(5) WI courses you will take in the program. This means that you will be evaluated on your proviciency of writing correct mathematical proofs. Some questions from the homework and team tests will receive feedback from me and you will be allowed to rewrite the proof correctly in your semester project. Overall, during the semester, you will be inspected to write at least 4,000 words (which is equivalent to 16 pages). The semester project is estimated to be 4 pages long.

\noindent{\bf Prerequisites:}
A grade of C or better in Math 243 or 253A, and Math 321 or consent.

\section*{Lectures}
There will be lectures each Mondays, Wednesdays, and Fridays, 12:30--1:20pm taking place at HIG 110. All students are expected to attend and participate in each lecture and take their own personal notes. Some material may be presented during lectures that is not in the book or in different order.

A student who misses a lecture is responsible for any assignments and/or announcements made. Unavoidable absences should be explained to the instructor. Office hours will not be utilized to re-teach material presented in a class missed by a student. 

\section*{Course material}

\noindent\textbf{Textbook:} Edward D. Gaughan, \emph{Introduction to Analysis}, 5th Ed. Access to the textbook may be required, as some homework questions will be assigned out of it. You may have access to the book online via the UH library website.

\noindent{\bf Lecture notes:} Attend lectures regularly and take your own notes.

\noindent{\bf Course website:} \url{https://mathopo.ca/courses-website/math-331/math-331}\\
All the information about the course (like the schedule and important dates) is posted on the course webiste.

\section*{Grading components}
%The students will be required to scan and upload their solutions to the homework and team tests on Gradescope (\texttt{http://gradescope.com}, entry code GEK6Y4).
The grading components are separated into 5 parts: Midterms, Final exam, Homework, Team tests, and Semester project. All homework must be send to me by email on the due date.
\begin{table}[ht]
\begin{tabular}{c|c|c}
Evaluation & Number & \% average \\ \hline\hline
Midterms & 3 & 30\% \\\hline
Final & 1 & 20\% \\\hline
Homework & 7 & 20\% \\\hline
Team Tests & 3 & 10\% \\\hline
S. Project & 1 & 20\% \\\hline\hline
Total & 15 & 100\%
\end{tabular}
\end{table}

\begin{enumerate}
\item {\bf Midterm exams:} There will be two(2) closed book and closed notes midterms exams. The first one on October, $1$ 2021 and the second one on November, $12$ 2021. You may also check the Schedule on the course website.
Each exam will have a duration of 50min and will be done during class (in-person). No extra time will be allowed. Midterms are not cummulative. This will be at least one question to test your mathematical proof writing. %You will get detailed feedback on these questions. Then you may add your corrected version of the proof to your semester project. 
This component counts for $30\%$ of the course average (at least $10\%$ will count for writing correct mathematical proofs).
\item {\bf Final exam:} There will be one closed book and closed notes final exam as scheduled on the 13, December 2021 12--2pm. This exam is cummulative and will be in-person. It is not possible to do the exam earlier. This part will count for $30\%$ of the course average.
\item{\bf Team tests:} There will be three(3) team tests in-person. You will work in team of two(2) on a set of problems (usually 2 problems and there will be a problem $x$ focusing on your mathematical proof-writing). The duration of the test will be 30 minutes. For the remaining 20 minutes, I will give the answers to the questions on the board and you will have to take your personal notes. %Your corrected copy will be uploaded on gradescope. 
Based on my feedback, you may rewrite your proof of problem $x$ and add it to your semester project. This part will count for $10\%$ of the course average (at least $5\%$ for evaluating skills to write correct mathematical proofs).
\item{\bf Homework:} There will be homework each two weeks. There will be assigned on Mondays (starting on 30, August 2021) and due on the next Monday before 1:20pm (see the schedule on the course website). The assigments will be posted on the course website each Monday's morning. You should have access to the book because some of the exercices will be extracted from it. This part will count for $20\%$ of the course average (at least $5\%$ for evaluating skills to write correct mathematical proofs). 
\item{\bf Semester project:} There will be a semester project. This project will be an opportunity to write-up correct proofs of the writing problems of the homework and the team tests. Even if your proof is correct, I will provide feedback to enhance the clarity of the exposition. I highly recommand {\LaTeX} to write your semester project. Here are some references that may help you:
	\begin{itemize}
	\item A {\LaTeX} tutorial: \url{http://math.hawaii.edu/wordpress/latex/}
	\item I recommand to use {\TeX}Maker to produce your source file.
	\item There is also an internet based options: \url{https://www.overleaf.com/}
	\item I will also provide a .tex template that you can use
	\end{itemize}
	For the structure of your project, you must provide the statement of the questions and the correct proof directly after (see the template offered on the course website). This project is espected to be a 4-pages document. The semester project must be sent to my email address. \textbf{You must follow the following template to name your file: LASTNAME{\_}FIRSTNAME.tex. If you don't respect this template, you will loose five(5) points (out of 100 points) and you will be asked to send your file again}. This part will count for $20\%$ of the overall average.
\end{enumerate}
An overall of $40\%$ ($5\% + 5\% + 20\%$) is dedicated to evaluate your skills at writing correct mathematical proofs.

\section*{Missed assignment policies}

\noindent{\bf Policies for exams:} Attendance on the exams is compulsory; otherwise, a grade of zero will be recorded. Any student who has an excused, documented conflict with a test time must inform their instructor \textbf{within two weeks before the midterm's date} when possible. Late requests will either be denied or will result in an automatic deduction from the exam score. 

For those students with an excused absence for a midterm, there will be a make-up exam which must be taken within two working days of the scheduled exam time (before or after). Conflicts arising from work or social obligations, or from personal travel plans do \textbf{not} qualify as excused absences. By registering for this course, you are agreeing to take all exams at the scheduled times.

\noindent{\bf Policies for homework:} No late homework will be accepted. A note of zero(0) will be assigned for a late homework.

\noindent{\bf Academic integrity:}
All students are expected to abide by the university's Conduct Code. Academic sanctions for dishonesty may include receiving an F in the assignment or receiving an F in the class. There may be additional administrative sanctions.
\newline {\url{https://www.hawaii.edu/policy/index.php?action=home&policySection=ep}}

\section*{Classroom policies}
Please refrain from using electronic items, including calculators, cell phones, music players, tablets, laptops, etc., during class, except for note-taking.
Please arrive, be seated and ready to start each class on time. If you have a valid reason to leave early, please sit near the exit to minimize disruption.

\section*{Sources of help}
All students are  encouraged to come to \textit{office hours} to discuss homework questions or material from class. If the advertised times do not suit you, please email me to set an appointment. It will be my pleasure to accomodate you.

{\bf KOKUA:} I am happy to work with you and the KOKUA Program (Office for Students with Disabilities), if you need course accommodations due to a disability. KOKUA can be reached at (808) 956-7511 or (808) 956-7612 (voice/text) in room 013 of the Queen Lili`uokalani Center for Student Services. All course modifications must be arranged through KOKUA. You are encouraged to start this process as early as possible.\svs

\section*{Concerns}
If at any time during the semester you have any questions or concerns about the class, please contact me during regularly scheduled office hours or via email to make an appointment. You may also contact the following people:
\spacer

\noindent {\bf Director of Undergraduate Studies}\\
Mirjana Jovovic \\
Email: \texttt{undergrad-dir@math.hawaii.edu}

\svs
\noindent {\bf Associate Chair}\\
Bj{\o}rn Kjos-Hanssen \\
Email: \texttt{assoc-chair@math.hawaii.edu}

\end{document}

